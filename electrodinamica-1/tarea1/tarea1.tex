\documentclass{article}
\usepackage{fvextra}
\usepackage{csquotes}

% Paquetes matemáticos y tipográficos
\usepackage{cancel}
\usepackage{mathrsfs}
\usepackage{amssymb}
\usepackage{amsmath}
\usepackage{amsfonts}
\usepackage{mathtools}
\usepackage{bigints}

% Paquetes necesarios
\usepackage{minted}
\usepackage{xcolor}

% Configura minted
\definecolor{codebg}{rgb}{0.1, 0.1, 0.1}
\setminted{
    style=monokai,
    bgcolor=codebg,
    linenos,
    breaklines,
    fontsize=\small
}

% Para la terminal
\usepackage{tcolorbox}
\tcbuselibrary{listingsutf8}
\usepackage{listings}
\usepackage{bera} % Fuente monoespaciada bonita (opcional)

% Permite referencias personalizadas
\usepackage{nameref}

% Numeración de ecuaciones por sección
\numberwithin{equation}{section}

% Hipervinculos
\usepackage[colorlinks=true,
            linkcolor=black,
            urlcolor=black,
            citecolor=black,
            filecolor=black,
            pdfborder={0 0 0}]{hyperref}

% Idioma en español
\usepackage[spanish]{babel}

% Manejo de imágenes
\usepackage{graphicx} 
\graphicspath{ {images/} }

% Configuración de márgenes
\usepackage[a4paper, left=1.5cm, right=1.5cm, top=20mm, bottom=20mm]{geometry}

% Tipografía mejorada
\usepackage{lmodern}

% Estilo de títulos con punto después del número
\usepackage{titlesec}
\titleformat{\section}{\huge\bfseries}{\thesection.}{1em}{}  % Título más grande

% Encabezados sin pie de página
\usepackage{fancyhdr}
\pagestyle{fancy}
\fancyhf{}
\fancyhead[L]{\textit{Tarea 1, 2025-2}}
\fancyhead[R]{Electrodinámica I}

% Mejor separación de párrafos
\setlength{\parindent}{0pt}
\setlength{\parskip}{5pt}

% Evita hifenaciones excesivas
\sloppy

% Configuración del índice
\usepackage{tocloft}
\setcounter{tocdepth}{2}

\begin{document}

% Portada
\begin{titlepage}
    \centering
    \vspace*{3cm} % Ajuste en la posición vertical
    % Logo centrado
    \includegraphics[width=0.6\textwidth]{UdeC_azul_centrado.png} 
    
    \vspace{1cm}
    \thispagestyle{empty} % Sin número en la portada

    % Título de la tarea
    {\Huge \textbf{Tarea 01} \par}
    
    \vspace{0.5cm}
    {\Huge \textbf{Electrodinámica I} \par}
    \vspace{1.5cm}

    % Nombre del autor
    {\Large José Ignacio Rosas Sepúlveda \par}
    \vspace{1cm}
    
    % Fechas de la tarea
    {\Large Septiembre 2025 \par}
    \vfill
\end{titlepage}

% Índice
\tableofcontents
\newpage

%%%%%%%%%%%%%%%%%%%%%%%%%%%%%%%%%%%%%%%%%%%%%%%%%%%%%%%%%%%%%%%%%%%%%%%%%%%%%%%%%%%%%%%%%%%%%%%%%%%%%%%%%%%%%%%%%%%%%%%%%%%%%%%%%%%%%%%%%%%%%%%%%%%%%%%%%%%%%%%%%%%%%%%%%%%%%%%%%%%%%%%%%%%%%%%%%%%%%%%

\section{Problema 1}
Considere un plano finito, de lados $a$ y $b$, carga neta $Q$ y densidad superficial de carga constante.  
Use un sistema de coordenadas cartesiano tal que el plano de carga quede contenido en el plano $xy$, eligiendo el origen en el centro del plano.  
De este modo, un punto arbitrario sobre el plano de carga tiene coordenadas
\begin{equation*}
    \vec{x}' = x'\,\hat{x} + y'\,\hat{y} \,.
\end{equation*}

%%%%%%%%%%%%%%%%%%%%%%%%%%%%%%%%%%%%%%%%%%%%%%%%%%%%%%%%%%%%%%%%%%%%%%%%%%%%%%%%%%%%%%%%%%%%%%%%%%%%%%%%%%%%%%%%%%%%%%%%%%%%%%%%%%%%%%%%%%%%%%%%%%%%%%%%%%%%%%%%%%%%%%%%%%%%%%%%%%%%%%%%%%%%%%%%%%%%%%%

\subsection{Campo eléctrico en un punto arbitrario}

Calcule el campo eléctrico $\vec{E}(x,y,z)$ en un punto arbitrario 
    \begin{equation}\label{1.1_posicion_espacio}
        \vec{x} = x\,\hat{x} + y\,\hat{y} + z\,\hat{z} \,.
    \end{equation}

\textbf{Solución:} 

%%%%%%%%%%%%%%%%%%%%%%%%%%%%%%%%%%%%%%%%%%%%%%%%%%%%%%%%%%%%%%%%%%%%%%%%%%%%%%%%%%%%%%%%%%%%%%%%%%%%%%%%%%%%%%%%%%%%%%%%%%%%%%%%%%%%%%%%%%%%%%%%%%%%%%%%%%%%%%%%%%%%%%%%%%%%%%%%%%%%%%%%%%%%%%%%%%%%%%%

Consideremos una placa rectangular centrada en el origen y contenida en el plano $xy$. Un punto genérico de la placa es
    \begin{equation}\label{1.1_posicion_plano}
        \vec{x}' = x'\,\hat{x} + y'\,\hat{y}\,,\qquad 
        x' \in \!\left[-\tfrac{a}{2},\,\tfrac{a}{2}\right],\quad
        y' \in \!\left[-\tfrac{b}{2},\,\tfrac{b}{2}\right].
    \end{equation}
y denotamos por $S=\left[-\tfrac{a}{2},\,\tfrac{a}{2}\right]\times\left[-\tfrac{b}{2},\,\tfrac{b}{2}\right]$ la superficie de la placa (véase la figura \ref{fig:rectangulo_S_vector}). 
Como la densidad superficial es uniforme (constante) en $S$, la carga total se define como
    \begin{align*}
        Q &= \iint_{S}\sigma\,dS \\[4pt]
        &= \sigma\,(\text{área de }S) \\[4pt]
        &=\sigma\,a\,b, 
    \end{align*}
de donde se sigue:
    \begin{equation*}
        \sigma = \frac{Q}{a\,b}\,.
    \end{equation*}
En lo que sigue trabajaremos con $\sigma$ para mantener las expresiones compactas.

El campo eléctrico en $\vec{x}$ está dado por
    \begin{equation}\label{1.1_campo_electrico}
        \vec{E}(\vec{x})
        = \frac{1}{4\pi\epsilon_0}\iint_{S}
        \frac{\sigma(\vec{x}')\big(\vec{x}-\vec{x}'\big)}
        {\|\vec{x}-\vec{x}'\|^{3}}\,dS\,,
    \end{equation}
donde $dS=dx'\,dy'$ y, en nuestro caso, $\sigma(\vec{x}')=\sigma$ (constante) para todo $\vec{x}'\in S$.


Sustituyendo \eqref{1.1_posicion_plano} y \eqref{1.1_posicion_espacio} en \eqref{1.1_campo_electrico} y usando que $\sigma$ es constante,
    \begin{equation*}
        \vec{E}(\vec{x})
        = \frac{\sigma}{4\pi\epsilon_0}
          \iint_{S}
          \frac{(x-x')\,\hat{x} + (y-y')\,\hat{y} + z\,\hat{z}}
               {\big[(x-x')^{2} + (y-y')^{2} + z^{2}\big]^{3/2}}
          \,dS\,.
    \end{equation*}
Separando por componentes,
    \begin{equation*}
        \vec{E}(\vec{x})=E_x(\vec{x})\,\hat{x}+E_y(\vec{x})\,\hat{y}+E_z(\vec{x})\,\hat{z}\,,
    \end{equation*}
con
    \begin{align*}
        E_x(\vec{x})
        &=\frac{\sigma}{4\pi\epsilon_0}\iint_{S}
          \frac{x-x'}{\big[(x-x')^{2}+(y-y')^{2}+z^{2}\big]^{3/2}}\,dS,\\[4pt]
        E_y(\vec{x})
        &=\frac{\sigma}{4\pi\epsilon_0}\iint_{S}
          \frac{y-y'}{\big[(x-x')^{2}+(y-y')^{2}+z^{2}\big]^{3/2}}\,dS,\\[4pt]
        E_z(\vec{x})
        &=\frac{\sigma}{4\pi\epsilon_0}\iint_{S}
          \frac{z}{\big[(x-x')^{2}+(y-y')^{2}+z^{2}\big]^{3/2}}\,dS.
    \end{align*}
Fijando los límites de S, tenemos:
    \begin{align}
         E_x(\vec{x})
        &= \frac{\sigma}{4\pi\epsilon_0}\int_{y'=-\frac{b}{2}}^{y'=\frac{b}{2}}
           \int_{x'=-\frac{a}{2}}^{x'=\frac{a}{2}}
           \frac{x-x'}{\big[(x-x')^{2}+(y-y')^{2}+z^{2}\big]^{3/2}}
           \,dx'\,dy', \label{1.1_Ex}\\[6pt]
        E_y(\vec{x})
        &= \frac{\sigma}{4\pi\epsilon_0}\int_{x'=-\frac{a}{2}}^{x'=\frac{a}{2}}
           \int_{y'=-\frac{b}{2}}^{y'=\frac{b}{2}}
           \frac{y-y'}{\big[(x-x')^{2}+(y-y')^{2}+z^{2}\big]^{3/2}}
           \,dy'\,dx', \label{1.1_Ey}\\[6pt]
        E_z(\vec{x})
        &= \frac{\sigma \,z}{4\pi\epsilon_0}\int_{y'=-\frac{b}{2}}^{y'=\frac{b}{2}}
           \int_{x'=-\frac{a}{2}}^{x'=\frac{a}{2}}
           \frac{1}{\big[(x-x')^{2}+(y-y')^{2}+z^{2}\big]^{3/2}}
           \,dx'\,dy'. \label{1.1_Ez}
    \end{align}
%%%%%%%%%%%%%%%%%%%%%%%%%%%%%%%%%%%%%%%%%%%%%%%%%%%% CAMPO ELECTRICO EN LA COORDENADA X %%%%%%%%%%%%%%%%%%%%%%%%%%%%%%%%%%%%%%%%%%%%%%%%%%%%

Con el fin de obtener la componente en la dirección $\hat{x}$ del campo eléctrico $\vec{E}$, procedemos a evaluar la integral doble indicada en la ecuación \eqref{1.1_Ex}. Para ello resulta útil emplear la siguiente antiderivada (véase Apéndice~\ref{app:antiderivadas}, ec.~\eqref{eq:app:int-u-over}):
    \begin{equation*}
        -\int \frac{u}{\left( u^2+\rho^2\right)
        ^{\frac{3}{2}}}du=\frac{1}{\sqrt{\rho^2+u^2}}+C,
    \end{equation*}
donde $x-x'=u \;\; \Rightarrow \;\; dx'=-du$ y $\rho^2=(y-y')^2+z^2$. Aplicando el cambio de variable y la sustitución indicada, la integral se transforma en:
    \begin{align*}
        E_x(\vec{x})
        &=\frac{\sigma}{4\pi\epsilon_0}
        \int_{y'=-\frac{b}{2}}^{y'=\frac{b}{2}}  
        \int_{x'=-\frac{a}{2}}^{x'=\frac{a}{2}} \frac{x-x'}{ \left[ (x-x')^2+(y-y')^2+z^2)\right]^{\frac{3}{2}}}dx'dy' \\[6pt]
        &=\frac{\sigma}{4\pi\epsilon_0}
        \int_{y'=-\frac{b}{2}}^{y'=\frac{b}{2}}  
        \left[ \frac{1}{\sqrt{(x-x')^2+(y-y')^2+z^2}}\right]_{x'=-\frac{a}{2}}^{x'=\frac{a}{2}}dy' \\[6pt]
        &=\frac{\sigma}{4\pi\epsilon_0}
        \int_{y'=-\frac{b}{2}}^{y'=\frac{b}{2}}   
        \left( \frac{1}{\sqrt{(x-\frac{a}{2})^2+(y-y')^2+z^2}}-\frac{1}{\sqrt{(x+\frac{a}{2})^2+(y-y')^2+z^2}}\right)dy' \\[6pt]
        &=\frac{\sigma}{4\pi\epsilon_0} \left(
        \int_{y'=-\frac{b}{2}}^{y'=\frac{b}{2}} 
        \frac{1}{\sqrt{(x-\frac{a}{2})^2+(y-y')^2+z^2}} dy'
        -\int_{y'=-\frac{b}{2}}^{y'=\frac{b}{2}} 
        \frac{1}{\sqrt{(x+\frac{a}{2})^2+(y-y')^2+z^2}}dy' \right) ,
    \end{align*}
El problema se reduce, entonces, a evaluar el par de integrales obtenidas en el paso anterior. Para ello se recurre a la antiderivada (véase Apéndice~\ref{app:antiderivadas}, ec.~\eqref{eq:app:int-asinh})
    \begin{equation*}
        \int \frac{dt}{\sqrt{t^2+\alpha^2}}=\operatorname{asinh}\!\left(\frac{t}{\alpha}\right)+C,
    \end{equation*}
complementando con el cambio de variable $t=y-y' \;\; \Rightarrow \;\; dy'=-dt$ y la definición de los parámetros auxiliares:
    \begin{equation*}
        \alpha_+=\sqrt{(x+\tfrac{a}{2})^2+z^2}\,, \qquad
        \alpha_-=\sqrt{(x-\tfrac{a}{2})^2+z^2}\,.
    \end{equation*}
De este modo, las integrales en la variable $y'$ se resuelven de la siguiente manera:
    \begin{align*}
        \int_{y'=-\frac{b}{2}}^{y'=\frac{b}{2}}
        \frac{1}{\sqrt{(x-\frac{a}{2})^2+(y-y')^2+z^2}}dy'
        &=
        \left[-\operatorname{asinh}\!\left(\frac{y-y'}{\sqrt{(x-\tfrac{a}{2})^2+z^2}}\right)\right]_{y'=-\frac{b}{2}}^{y'=\frac{b}{2}}\\
        &=\operatorname{asinh}\!\left(\frac{y+\frac{b}{2}}{\sqrt{(x-\tfrac{a}{2})^2+z^2}}\right)
        -\operatorname{asinh}\!\left(\frac{y-\frac{b}{2}}{\sqrt{(x-\tfrac{a}{2})^2+z^2}}\right),
        \\[10pt]
        \int_{y'=-\frac{b}{2}}^{y'=\frac{b}{2}}
        \frac{1}{\sqrt{(x+\frac{a}{2})^2+(y-y')^2+z^2}}dy'
        &=
        \left[-\operatorname{asinh}\!\left(\frac{y-y'}{\sqrt{(x+\tfrac{a}{2})^2+z^2}}\right)\right]_{y'=-\frac{b}{2}}^{y'=\frac{b}{2}}\\
        &=\operatorname{asinh}\!\left(\frac{y+\frac{b}{2}}{\sqrt{(x+\tfrac{a}{2})^2+z^2}}\right)
        -\operatorname{asinh}\!\left(\frac{y-\frac{b}{2}}{\sqrt{(x+\tfrac{a}{2})^2+z^2}}\right).
    \end{align*}
En consecuencia, la componente en la dirección $\hat{x}$ del campo eléctrico $\vec{E}$ adopta la forma:
    \begin{align*}
        E_x(\vec{x})
        &=\frac{\sigma}{4\pi\epsilon_0}\left[
        \operatorname{asinh}\!\left(\frac{y+\frac{b}{2}}{\sqrt{(x-\tfrac{a}{2})^2+z^2}}\right)
        -\operatorname{asinh}\!\left(\frac{y-\frac{b}{2}}{\sqrt{(x-\tfrac{a}{2})^2+z^2}}\right)\right.\\[6pt]
        &\quad-
        \left.
        \operatorname{asinh}\!\left(\frac{y+\frac{b}{2}}{\sqrt{(x+\tfrac{a}{2})^2+z^2}}\right)
        +\operatorname{asinh}\!\left(\frac{y-\frac{b}{2}}{\sqrt{(x+\tfrac{a}{2})^2+z^2}}\right)
        \right].
    \end{align*}

%%%%%%%%%%%%%%%%%%%%%%%%%%%%%%%%%%%%%%%%%%%%%%%%%%%% CAMPO ELECTRICO EN LA COORDENADA Y %%%%%%%%%%%%%%%%%%%%%%%%%%%%%%%%%%%%%%%%%%%%%%%%%%%%
De manera análoga al procedimiento desarrollado para la componente $\hat{x}$, la componente en la dirección $\hat{y}$ del campo eléctrico $\vec{E}$ se obtiene evaluando la integral doble de la ecuación \eqref{1.1_Ey}. Para ello se recurre a la siguiente antiderivada (véase Apéndice~\ref{app:antiderivadas}, ec.~\eqref{eq:app:int-u-over}): 
    \begin{equation*}
        -\int \frac{v}{\big(v^2+\rho^2\big)^{3/2}}\,dv=\frac{1}{\sqrt{v^2+\rho^2}}+C,
    \end{equation*}
con el cambio de variable $y-y'=v \;\; \Rightarrow \;\; dy'=-dv$ y la sustitución $\rho^2=(x-x')^2+z^2$. Bajo estas consideraciones, la integral se reescribe como:"
    \begin{align*}
        E_y(\vec{x})
        &=\frac{\sigma}{4\pi\epsilon_0}
        \int_{x'=-\frac{a}{2}}^{x'=\frac{a}{2}}
        \int_{y'=-\frac{b}{2}}^{y'=\frac{b}{2}}
       \frac{y-y'}{\big[(x-x')^{2}+(y-y')^{2}+z^{2}\big]^{3/2}}
       \,dy'\,dx' \\[6pt]
        &=\frac{\sigma}{4\pi\epsilon_0}
        \int_{x'=-\frac{a}{2}}^{x'=\frac{a}{2}}
        \left[ \frac{1}{\sqrt{(x-x')^2+(y-y')^2+z^2}}\right]_{y'=-\frac{b}{2}}^{y'=\frac{b}{2}}dx' \\[6pt]
        &=\frac{\sigma}{4\pi\epsilon_0}
        \int_{x'=-\frac{a}{2}}^{x'=\frac{a}{2}}
        \left( \frac{1}{\sqrt{(x-x')^2+(y-\frac{b}{2})^2+z^2}}-\frac{1}{\sqrt{(x-x')^2+(y+\frac{b}{2})^2+z^2}}\right)dx' \\[6pt]
        &=\frac{\sigma}{4\pi\epsilon_0} \left(
        \int_{x'=-\frac{a}{2}}^{x'=\frac{a}{2}}
        \frac{1}{\sqrt{(x-x')^2+(y-\frac{b}{2})^2+z^2}} dx'
        -\int_{x'=-\frac{a}{2}}^{x'=\frac{a}{2}}
        \frac{1}{\sqrt{(x-x')^2+(y+\frac{b}{2})^2+z^2}}dx' \right) ,
    \end{align*}
El cálculo se reduce, entonces, a la evaluación del par de integrales resultantes. Para ello se emplea la antiderivada (véase Apéndice~\ref{app:antiderivadas}, ec.~\eqref{eq:app:int-asinh})
    \begin{equation*}
        \int \frac{dw}{\sqrt{w^2+\beta^2}}=\operatorname{asinh}\!\left(\frac{w}{\beta}\right)+C,
    \end{equation*}
en combinación con el cambio de variable $w=x-x'\;\; \Rightarrow \;\; dx'=-dw$, y definiendo los parámetros auxiliares:
    \begin{equation*}
        \beta+=\sqrt{(y+\tfrac{b}{2})^2+z^2}\,, \qquad
        \beta_-=\sqrt{(y-\tfrac{b}{2})^2+z^2}\,.
    \end{equation*}
En consecuencia, las integrales en la variable $x'$ se evalúan del siguiente modo:
    \begin{align*}
    \int_{x'=-\frac{a}{2}}^{x'=\frac{a}{2}}
    \frac{1}{\sqrt{(y-\frac{b}{2})^2+(x-x')^2+z^2}}dx'
    &=\left[-\operatorname{asinh}\!\left(\frac{x-x'}{\sqrt{(y-\tfrac{b}{2})^2+z^2}}\right)\right]_{x'=-\frac{a}{2}}^{x'=\frac{a}{2}}\\[6pt]
    &=\operatorname{asinh}\!\left(\frac{x+\tfrac{a}{2}}{\sqrt{(y-\tfrac{b}{2})^2+z^2}}\right)
    -\operatorname{asinh}\!\left(\frac{x-\tfrac{a}{2}}{\sqrt{(y-\tfrac{b}{2})^2+z^2}}\right),\\[10pt]
    \int_{x'=-\frac{a}{2}}^{x'=\frac{a}{2}}
    \frac{1}{\sqrt{(y+\frac{b}{2})^2+(x-x')^2+z^2}}dx'
    &=\left[-\operatorname{asinh}\!\left(\frac{x-x'}{\sqrt{(y+\tfrac{b}{2})^2+z^2}}\right)\right]_{x'=-\frac{a}{2}}^{x'=\frac{a}{2}}\\[6pt]
    &=\operatorname{asinh}\!\left(\frac{x+\tfrac{a}{2}}{\sqrt{(y+\tfrac{b}{2})^2+z^2}}\right)
    -\operatorname{asinh}\!\left(\frac{x-\tfrac{a}{2}}{\sqrt{(y+\tfrac{b}{2})^2+z^2}}\right)\,.
    \end{align*}
Por lo tanto, la componente en la dirección $\hat{y}$ del campo eléctrico $\vec{E}$ queda expresada como:
    \begin{align*}
        E_y(\vec{x})
        &=\frac{\sigma}{4\pi\epsilon_0}\left[
        \operatorname{asinh}\!\left(\frac{x+\tfrac{a}{2}}{\sqrt{(y-\tfrac{b}{2})^2+z^2}}\right)
        -\operatorname{asinh}\!\left(\frac{x-\tfrac{a}{2}}{\sqrt{(y-\tfrac{b}{2})^2+z^2}}\right)
        \right. \\[6pt]
        &\quad-
        \left.
        \operatorname{asinh}\!\left(\frac{x+\tfrac{a}{2}}{\sqrt{(y+\tfrac{b}{2})^2+z^2}}\right)
        +\operatorname{asinh}\!\left(\frac{x-\tfrac{a}{2}}{\sqrt{(y+\tfrac{b}{2})^2+z^2}}\right)
        \right].
    \end{align*}
 
%%%%%%%%%%%%%%%%%%%%%%%%%%%%%%%%%%%%%%%%%%%%%%%%%%%% CAMPO ELECTRICO EN LA COORDENADA Z %%%%%%%%%%%%%%%%%%%%%%%%%%%%%%%%%%%%%%%%%%%%%%%%%%%%
Finalmente, la componente en la dirección $\hat{z}$ del campo eléctrico $\vec{E}$ se determina a partir de la integral doble de la ecuación \eqref{1.1_Ez}. Para su evaluación resulta conveniente emplear la siguiente antiderivada (véase Apéndice~\ref{app:antiderivadas}, ec.~\eqref{eq:app:int-one-over}):
    \begin{equation*}
        \int \frac{1}{ (u^2+\rho^2)^{\frac{3}{2}}}du=\frac{1}{\rho^2}\frac{u}{\sqrt{\rho^2+u^2}}+C,
    \end{equation*}
introduciendo el cambio de variable $x-x'=u \;\; \Rightarrow \;\; dx'=-du$ y efectuando la sustitución $\rho^2=(y-y')^2+z^2$:
    \begin{align*}
        E_z(\vec{x})
        &=\frac{\sigma \,z}{4\pi\epsilon_0}
        \int_{y'=-\frac{b}{2}}^{y'=\frac{b}{2}}
               \int_{x'=-\frac{a}{2}}^{x'=\frac{a}{2}}
               \frac{1}{\big[(x-x')^{2}+(y-y')^{2}+z^{2}\big]^{3/2}}
               \,dx'\,dy' \\[6pt]
        &= \frac{\sigma \,z}{4\pi\epsilon_0}
        \int_{y'=-\frac{b}{2}}^{y'=\frac{b}{2}}
        \left[-\frac{1}{(y-y')^2+z^2}\frac{x-x'}{\sqrt{(x-x')^2+(y-y')^2+z^2}} \right]_{x'=-\frac{a}{2}}^{x'=\frac{a}{2}} dy'\\
        &=\frac{\sigma \,z}{4\pi\epsilon_0}
        \int_{y'=-\frac{b}{2}}^{y'=\frac{b}{2}}
        \left( \frac{1}{(y-y')^2+z^2}\frac{x+\frac{a}{2}}{\sqrt{(x+\frac{a}{2})^2+(y-y')^2+z^2}} 
        -
        \frac{1}{(y-y')^2+z^2}\frac{x-\frac{a}{2}}{\sqrt{(x-\frac{a}{2})^2+(y-y')^2+z^2}} \right)dy' \\
        &=\frac{\sigma \,z}{4\pi\epsilon_0} \left(
        \int_{y'=-\frac{b}{2}}^{y'=\frac{b}{2}}
        \frac{1}{(y-y')^2+z^2}\frac{x+\frac{a}{2}}{\sqrt{(x+\frac{a}{2})^2+(y-y')^2+z^2}} dy' \right.\\[6pt]
        &\left. \quad
        -\int_{y'=-\frac{b}{2}}^{y'=\frac{b}{2}} 
        \frac{1}{(y-y')^2+z^2}\frac{x-\frac{a}{2}}{\sqrt{(x-\frac{a}{2})^2+(y-y')^2+z^2}} dy'\right). 
    \end{align*}
El problema se reduce, en este punto, a la evaluación del par de integrales resultantes. Para ello se hace uso de la antiderivada (véase Apéndice~\ref{app:antiderivadas}, ec.~\eqref{eq:app:int-A-over})
    \begin{equation*}
        \int\frac{A}{(s^2+z^2)\,\sqrt{A^2+s^2+z^2}}ds
        = \frac{1}{z}\,
        \arctan\!\left(\frac{A\,s}{z\,\sqrt{A^2+s^2+z^2}}\right) +C,
    \end{equation*}
junto con el cambio de variable $s=y-y' \;\; \Rightarrow \;\; dy'=-ds$, y la introducción de los parámetros auxiliares:
    \begin{equation*}
        A_+=x+\frac{a}{2}\,, \qquad
        A_-=x-\frac{a}{2}\,.
    \end{equation*}
De este modo, las integrales en la variable $y'$ se pueden expresar como:
    \begin{align*}
        \int_{y'=-\frac{b}{2}}^{y'=\frac{b}{2}} \frac{1}{(y-y')^2+z^2}\frac{x+\frac{a}{2}}{\sqrt{(x+\frac{a}{2})^2+(y-y')^2+z^2}} dy'
        &=\left[ -\frac{1}{z}\,\arctan\!\left(\frac{(x+\frac{a}{2})\,(y-y')}{z\,\sqrt{(x+\frac{a}{2})^2+(y-y')^2+z^2}}\right)\right]_{y'=-\frac{b}{2}}^{\,y'=\frac{b}{2}} \\[6pt]
        &=\frac{1}{z}\,\arctan\!\left(\frac{(x+\frac{a}{2})\,(y+\frac{b}{2})}{z\,\sqrt{(x+\frac{a}{2})^2+(y+\frac{b}{2})^2+z^2}}\right) \\[6pt]
        &- \frac{1}{z}\,\arctan\!\left(\frac{(x+\frac{a}{2})\,(y-\frac{b}{2})}{z\,\sqrt{(x+\frac{a}{2})^2+(y-\frac{b}{2})^2+z^2}}\right),\\[10pt]
        \int_{y'=-\frac{b}{2}}^{y'=\frac{b}{2}} \frac{1}{(y-y')^2+z^2}\frac{x-\frac{a}{2}}{\sqrt{(x-\frac{a}{2})^2+(y-y')^2+z^2}} dy'
        &=\left[ -\frac{1}{z}\,\arctan\!\left(\frac{(x-\frac{a}{2})\,(y-y')}{z\,\sqrt{(x-\frac{a}{2})^2+(y-y')^2+z^2}}\right)\right]_{y'=-\frac{b}{2}}^{\,y'=\frac{b}{2}} \\[6pt]
        &=\frac{1}{z}\,\arctan\!\left(\frac{(x-\frac{a}{2})\,(y+\frac{b}{2})}{z\,\sqrt{(x-\frac{a}{2})^2+(y+\frac{b}{2})^2+z^2}}\right) \\[6pt]
        &- \frac{1}{z}\,\arctan\!\left(\frac{(x-\frac{a}{2})\,(y-\frac{b}{2})}{z\,\sqrt{(x-\frac{a}{2})^2+(y-\frac{b}{2})^2+z^2}}\right).
    \end{align*}
En consecuencia, la componente en la dirección $\hat{z}$ del campo eléctrico $\vec{E}$ se obtiene en la forma:
    \begin{align*}
        E_z(\vec{x})
        &=\frac{\sigma}{4\pi\epsilon_0}\left[
        \arctan\!\left(\frac{(x+\frac{a}{2})\,(y+\frac{b}{2})}{z\,\sqrt{(x+\frac{a}{2})^2+(y+\frac{b}{2})^2+z^2}}\right)
        -\arctan\!\left(\frac{(x+\frac{a}{2})\,(y-\frac{b}{2})}{z\,\sqrt{(x+\frac{a}{2})^2+(y-\frac{b}{2})^2+z^2}}\right)
        \right. \\[6pt]
        &\quad-
        \left.
        \arctan\!\left(\frac{(x-\frac{a}{2})\,(y+\frac{b}{2})}{z\,\sqrt{(x-\frac{a}{2})^2+(y+\frac{b}{2})^2+z^2}}\right)
        +\arctan\!\left(\frac{(x-\frac{a}{2})\,(y-\frac{b}{2})}{z\,\sqrt{(x-\frac{a}{2})^2+(y-\frac{b}{2})^2+z^2}}\right)
        \right].
    \end{align*}

%%%%%%%%%%%%%%%%%%%%%%%%%%%%%%%%%%%%%%%%%%%%%%%%%%%%%%%%%%%%%%%%%%%%%%%

En conclusión, el campo eléctrico generado por una placa rectangular de densidad superficial de carga constante $\sigma$, evaluado en un punto arbitrario del espacio, se expresa explícitamente como una función de la posición $\vec{x}$:
    \begin{align}
        \vec{E}(\vec{x})
        &= \frac{\sigma}{4\pi\epsilon_0} \left[
        \operatorname{asinh}\!\left(\frac{y+\frac{b}{2}}{\sqrt{(x-\tfrac{a}{2})^2+z^2}}\right)
        -\operatorname{asinh}\!\left(\frac{y-\frac{b}{2}}{\sqrt{(x-\tfrac{a}{2})^2+z^2}}\right)
        \right. \notag\\[6pt]
        &\quad-
        \left.
        \operatorname{asinh}\!\left(\frac{y+\frac{b}{2}}{\sqrt{(x+\tfrac{a}{2})^2+z^2}}\right)
        +\operatorname{asinh}\!\left(\frac{y-\frac{b}{2}}{\sqrt{(x+\tfrac{a}{2})^2+z^2}}\right)
        \right]\,\hat{x} \notag\\[6pt]
        &\quad +
        \frac{\sigma}{4\pi\epsilon_0} \left[
        \operatorname{asinh}\!\left(\frac{x+\tfrac{a}{2}}{\sqrt{(y-\tfrac{b}{2})^2+z^2}}\right)
        -\operatorname{asinh}\!\left(\frac{x-\tfrac{a}{2}}{\sqrt{(y-\tfrac{b}{2})^2+z^2}}\right)
        \right. \notag\\[2pt]
        &\quad-
        \left.
        \operatorname{asinh}\!\left(\frac{x+\tfrac{a}{2}}{\sqrt{(y+\tfrac{b}{2})^2+z^2}}\right)
        +\operatorname{asinh}\!\left(\frac{x-\tfrac{a}{2}}{\sqrt{(y+\tfrac{b}{2})^2+z^2}}\right)
        \right]\,\hat{y} \notag\\[6pt]
        &\quad +
        \frac{\sigma}{4\pi\epsilon_0} 
        \left[
        \arctan\!\left(\frac{(x+\frac{a}{2})\,(y+\frac{b}{2})}{z\,\sqrt{(x+\frac{a}{2})^2+(y+\frac{b}{2})^2+z^2}}\right)
        -\arctan\!\left(\frac{(x+\frac{a}{2})\,(y-\frac{b}{2})}{z\,\sqrt{(x+\frac{a}{2})^2+(y-\frac{b}{2})^2+z^2}}\right)
        \right. \notag\\[2pt]
        &\quad-
        \left.
        \arctan\!\left(\frac{(x-\frac{a}{2})\,(y+\frac{b}{2})}{z\,\sqrt{(x-\frac{a}{2})^2+(y+\frac{b}{2})^2+z^2}}\right)
        +\arctan\!\left(\frac{(x-\frac{a}{2})\,(y-\frac{b}{2})}{z\,\sqrt{(x-\frac{a}{2})^2+(y-\frac{b}{2})^2+z^2}}\right)
        \right]\,\hat{z}. \label{eq:campo-rectangular}
    \end{align}

La expresión \eqref{eq:campo-rectangular} constituye la forma cerrada del campo eléctrico para una placa finita de lados $a$ y $b$, válida en todo el espacio. Este resultado generaliza casos particulares de interés y, en el límite $a,b \to \infty$, conduce al conocido campo uniforme de una lámina infinita. Dicho análisis se llevará a cabo con mayor detalle en la Sección~\ref{sec:limite-infinito}.

\begin{figure}[htbp]
  \centering
  \includegraphics[width=.88\linewidth]{fig_rectangulo_S.png}
  \caption{Región rectangular $S=\left[-\tfrac{a}{2},\,\tfrac{a}{2}\right]\times\left[-\tfrac{b}{2},\,\tfrac{b}{2}\right]$ contenida en el plano $xy$ ($z=0$), con sus vértices marcados. Se muestra un punto $\vec{x}'=x'\,\hat{x}+y'\,\hat{y}\in S$ y un vector arbitrario $\vec{x}=x\,\hat{x}+y\,\hat{y}+z\,\hat{z}$ aplicado en el origen. Los ejes coordenados se incluyen como guía visual.}
  \label{fig:rectangulo_S_vector}
\end{figure}

\clearpage
%%%%%%%%%%%%%%%%%%%%%%%%%%%%%%%%%%%%%%%%%%%%%%%%%%%%%%%%%%%%%%%%%%%%%%%%%%%%%%%%%%%%%%%%%%%%%%%%%%%%%%%%%%%%%%%%%%%%%%%%%%%%%%%%%%%%%%%%%%%%%%%%%%%%%%%%%%%%%%%%%%%%%%%%%%%%%%%%%%%%%%%%%%%%%%%%%%%%%%%

\subsection{Campo eléctrico en el plano \texorpdfstring{$xz$}{xz}}

Grafique el campo eléctrico en el plano $xz$. Adapte el código disponible \href{https://github.com/gfrubi/electrodinamica/blob/master/notebooks/Ejemplo-quiver-y-odenint.ipynb}{aquí}.


\textbf{Solución:}

%%%%%%%%%%%%%%%%%%%%%%%%%%%%%%%%%%%%%%%%%%%%%%%%%%%%%%%%%%%%%%%%%%%%%%%%%%%%%%%%%%%%%%%%%%%%%%%%%%%%%%%%%%%%%%%%%%%%%%%%%%%%%%%%%%%%%%%%%%%%%%%%%%%%%%%%%%%%%%%%%%%%%%%%%%%%%%%%%%%%%%%%%%%%%%%%%%%%%%%
% Se fijan constantes físicas usadas en todo el script (permivitidad del vacío y constante de Coulomb k=1/(4πϵ0)).

Empleando las librerías \texttt{numpy} y \texttt{matplotlib.pyplot}, se establecen las constantes de \textbf{permitividad del vacío} $\epsilon_0$ y la constante de Coulomb $\frac{1}{4\pi\epsilon_0}$ de forma:
\begin{center}
\begin{minted}{python}
# ==============================
# Constantes y parámetros globales
# ==============================

EPS0 = 8.8541878128e-12                    # permitividad del vacío (SI)
constante_de_coulomb = 1.0/(4.0*np.pi*EPS0)  # constante de Coulomb
\end{minted}
\end{center}

% Se implementan las componentes analíticas del campo para la placa finita (expresiones cerradas deducidas en la sección previa).

A continuación, se implementan en \texttt{python} las expresiones analíticas de las componentes $E_x$, $E_y$ y $E_z$ para una placa rectangular finita de lados $a$ y $b$:
\begin{center}
\begin{minted}{python}
def Ex(x, y, z, a, b, sigma):
    """
    Componente Ex del campo eléctrico de una placa rectangular centrada en el origen
    y contenida en el plano z=0 (lados a×b, densidad uniforme sigma).

    Args:
        x (array_like): Coordenada x (escalar o arreglo broadcast-compatible).
        y (array_like): Coordenada y (escalar o arreglo broadcast-compatible).
        z (array_like): Coordenada z (escalar o arreglo broadcast-compatible).
        a (float): Longitud en x del lado de la placa (m).
        b (float): Longitud en y del lado de la placa (m).
        sigma (float): Densidad superficial de carga (C/m^2).

    Returns:
        np.ndarray: Componente Ex (V/m), con misma forma que el broadcast de (x,y,z).
    """
    return constante_de_coulomb*sigma*(np.arcsinh( (b/2.0 - y) / np.sqrt( (x - a/2.0)**2 + z**2) ) - np.arcsinh( (-b/2.0 - y) / np.sqrt((x - a/2.0)**2 + z**2))
            - np.arcsinh( (b/2.0 - y) / np.sqrt((x + a/2.0)**2 + z**2) ) + np.arcsinh( (-b/2.0 - y) / np.sqrt((x + a/2.0)**2 + z**2) ))
\end{minted}
\end{center}

\begin{center}
\begin{minted}{python}
def Ey(x, y, z, a, b, sigma):
    """
    Componente Ey del campo eléctrico de una placa rectangular centrada en el origen
    y contenida en el plano z=0 (lados a×b, densidad uniforme sigma).

    Args:
        x (array_like): Coordenada x (escalar o arreglo broadcast-compatible).
        y (array_like): Coordenada y (escalar o arreglo broadcast-compatible).
        z (array_like): Coordenada z (escalar o arreglo broadcast-compatible).
        a (float): Longitud en x del lado de la placa (m).
        b (float): Longitud en y del lado de la placa (m).
        sigma (float): Densidad superficial de carga (C/m^2).

    Returns:
        numpy.ndarray: Componente Ey (V/m), con misma forma que el broadcast de (x,y,z).
    """
    return constante_de_coulomb*sigma*((np.arcsinh( (a/2.0 - x) / np.sqrt((y - b/2.0)**2 + z**2) ) + np.arcsinh( (a/2.0 + x) / np.sqrt((y - b/2.0)**2 + z**2) ))
            - (np.arcsinh( (a/2.0 - x) / np.sqrt((y + b/2.0)**2 + z**2) ) + np.arcsinh( (a/2.0 + x) / np.sqrt((y + b/2.0)**2 + z**2) )))
\end{minted}
\end{center}

\begin{center}
\begin{minted}{python}
def Ez(x, y, z, a, b, sigma):
    """
    Componente Ez del campo eléctrico de una placa rectangular centrada en el origen
    y contenida en el plano z=0 (lados a×b, densidad uniforme sigma).
    Implementación por esquinas con arctan; usa z_safe para estabilidad numérica en z\aprox0.

    Args:
        x (array_like): Coordenada x (escalar o arreglo broadcast-compatible).
        y (array_like): Coordenada y (escalar o arreglo broadcast-compatible).
        z (array_like): Coordenada z (escalar o arreglo broadcast-compatible).
        a (float): Longitud en x del lado de la placa (m).
        b (float): Longitud en y del lado de la placa (m).
        sigma (float): Densidad superficial de carga (C/m^2).

    Returns:
        numpy.ndarray: Componente Ez (V/m), con misma forma que el broadcast de (x,y,z).

    Notas (comentarios con LaTeX):
        - Se usa $z_{\\text{safe}}=\\max(|z|,10^{-12})$ para evitar división por cero ($z\\to 0$).
        - También se regula la raíz con $\epsilon$ en cada esquina.
        - La componente normal presenta salto en $z=0$: $\\Delta E_z = \\sigma/\\epsilon_0$.
    """
    z_safe = np.where(np.abs(z) < 1e-12, 1e-12, z)
    return constante_de_coulomb*sigma*( np.arctan( ( (x+a/2.0) * (y+b/2.0) ) / ( z_safe * np.sqrt( (x+a/2.0)**2 + (y+b/2.0)**2 + z**2 ) ))
          - np.arctan(( (x+a/2.0) * (y-b/2.0) ) / ( z_safe * np.sqrt( (x+a/2.0)**2 + (y-b/2.0)**2 + z**2 ) ))
          - np.arctan(( (x-a/2.0) * (y+b/2.0) ) / ( z_safe * np.sqrt( (x-a/2.0)**2 + (y+b/2.0)**2 + z**2 ) ))
          + np.arctan(( (x-a/2.0) * (y-b/2.0) ) / ( z_safe * np.sqrt( (x-a/2.0)**2 + (y-b/2.0)**2 + z**2 ) )) )
\end{minted}
\end{center}

% Función auxiliar para el corte en el plano xz (y=0): devuelve sólo (E_x, E_z) que son las componentes visibles en el quiver.

Para el corte $y=0$ (plano $xz$), se define una función auxiliar que retorna únicamente las componentes visibles en el gráfico de flechas:
\begin{center}
\begin{minted}{python}
def E_xz(x, z, a, b, sigma):
    """
    Campo en el plano xz (y=0) para una placa rectangular (a×b) con densidad sigma.

    Args:
        x (array_like): Coordenada x (escalar o arreglo broadcast-compatible).
        z (array_like): Coordenada z (escalar o arreglo broadcast-compatible).
        a (float): Longitud en x del lado de la placa (m).
        b (float): Longitud en y del lado de la placa (m).
        sigma (float): Densidad superficial de carga (C/m^2).

    Returns:
        tuple[np.ndarray, np.ndarray]: Par (E_x, E_z) en V/m, evaluado en y=0.
    """
    # --- Componentes en el plano xz (fijando y=0) ---
    coordenada_en_x = Ex(x, 0, z, a, b, sigma)
    coordenada_en_z = Ez(x, 0, z, a, b, sigma)
    return coordenada_en_x, coordenada_en_z
\end{minted}
\end{center}

% Rutina de graficación: malla, evaluación, normalización para dirección y codificación por color de la intensidad. Se dibuja la proyección de la placa en z=0.

Finalmente, se genera el \texttt{quiver} sobre una malla rectangular en $(x,z)$: se evalúa $(E_x,E_z)$, se normalizan los vectores para resaltar la dirección y se colorean por la magnitud $\|\vec{E}\|$ en el plano, incorporando además la proyección de la placa sobre $z=0$:
\begin{center}
\begin{minted}{python}
def plot_Exz(a=1.0, b=0.6, sigma=1.0, n=20, max=2):
    """
    Quiver 2D del campo eléctrico en el plano xz (y=0) para una placa de lados a×b
    con densidad superficial sigma. Las flechas muestran dirección (normalizadas)
    y el color codifica la magnitud plana $\\sqrt{E_x^2 + E_z^2}$.

    Args:
        a (float): Longitud en x del lado de la placa (m).
        b (float): Longitud en y del lado de la placa (m).
        sigma (float): Densidad superficial de carga (C/m^2).
        n (int): Número de puntos por eje de la grilla (resolución).
        max_ext (float): Semiextensión del dominio en x y z; se grafica en $[-\\text{max\\_ext},\\text{max\\_ext}]$.

    Returns:
        - Guarda 'campo2d-normalizado-colores.svg' en el directorio actual.
        - Para visualizar en pantalla, descomentar 'plt.show()'.
    """

    # --- Malla de evaluación (plano xz con y=0) ---
    X = np.linspace(-max, max, n)
    Z = np.linspace(-max, max, n)
    x, z = np.meshgrid(X, Z)  # todas las combinaciones (x,z) de la grilla
    
    # --- Campo eléctrico en y=0 (componentes x y z) ---
    # Nota: E_xz debe devolver dos arrays (E_x, E_z) evaluados en la grilla.
    E_x, E_z = E_xz(x, z, a, b, sigma)

    # --- Normalización de vectores para mostrar dirección ---
    # Empleamos la magnitud plana para normalizar y evitar flechas desproporcionadas.
    E_magnitud = np.hypot(E_x, E_z) + 1e-16  # el término 1e-16 previene división por cero
    E_x_normalizado = E_x / E_magnitud
    E_z_normalizado = E_z / E_magnitud

    # --- Gráfico quiver: dirección normalizada y color por magnitud ---
    plt.figure(figsize=(7, 5))
    plt.quiver(
        x, z,                           # posiciones en la grilla
        E_x_normalizado, E_z_normalizado,  # componentes normalizadas (dirección)
        E_magnitud,                     # magnitud (para el color)
        pivot='middle',
        cmap='viridis'
    )

    # --- Ventana gráfica y rótulos ---
    plt.xlim(-1.1*max, 1.1*max)
    plt.ylim(-1.1*max, 1.1*max)
    plt.xlabel("Coordenada x (m)")  
    plt.ylabel("Coordenada z (m)")  
    plt.title('$\\vec{E}(x, y=0, z)$')
    plt.colorbar(label="Magnitud del campo eléctrico ($V/m$)")  # barra de color asociada a la magnitud

    # --- Placa (proyección en el plano xz) ---
    # Se dibuja como un segmento en z=0, desde x=-a/2 hasta x=a/2.
    plt.hlines(0, -a/2, a/2, linewidth=3, color="purple")

    # --- Exportar a archivo vectorial (SVG) ---
    plt.savefig('campo2d-normalizado-colores.svg')
    #plt.show()
    plt.close()
    return 'campo2d-normalizado-colores.svg'
\end{minted}
\end{center}

Por último, llamamos a esta función con los parámetros de ejemplo 
$a=2.0\, [\text{m}]$, $b=1.6\, [\text{m}]$ y $\sigma=5.0 \,[\text{C/m}^2]$, de forma:

\begin{center}
\begin{minted}{python}
plot_Exz(a=2.0, b=1.6, sigma=5.0)
\end{minted}
\end{center}

Así se obtiene la figura solicitada; véase la figura~\ref{plot_2d}.

\begin{figure}
    \centering
    \includegraphics[width=1\linewidth]{campo_Exz_2D.png}
    \caption{Mapa vectorial del campo eléctrico en el plano $xz$ para $y=0$ generado por una placa rectangular finita de lados $a=2.0\,\text{m}$ (eje $x$) y $b=1.6\,\text{m}$ (eje $y$), con densidad superficial uniforme $\sigma=5.0\,\text{C/m}^2$. Las flechas indican la \emph{dirección} del campo (vectores normalizados), mientras que el color representa su \emph{intensidad} $\|\vec{E}\|$ en el plano. La línea gruesa en $z=0$ corresponde a la proyección de la placa (paralela al eje $x$) en el corte considerado.}
    \label{plot_2d}
\end{figure}


\clearpage
%%%%%%%%%%%%%%%%%%%%%%%%%%%%%%%%%%%%%%%%%%%%%%%%%%%%%%%%%%%%%%%%%%%%%%%%%%%%%%%%%%%%%%%%%%%%%%%%%%%%%%%%%%%%%%%%%%%%%%%%%%%%%%%%%%%%%%%%%%%%%%%%%%%%%%%%%%%%%%%%%%%%%%%%%%%%%%%%%%%%%%%%%%%%%%%%%%%%%%%

\subsection{Campo eléctrico en 3D}

Con la ayuda de la IA generativa de su agrado (ChatGPT, Gemini, Copilot, Manus, etc.) cree un código que grafique el campo eléctrico en tres dimensiones.

\textbf{Solución:}

Para organizar la evaluación del campo, primero se define una función \texttt{E\_field} que actúa como \emph{envolvente}: dada una placa rectangular de lados $a\times b$ y densidad superficial uniforme $\sigma$, retorna el vector \((E_x,E_y,E_z)\) en un punto o malla \((x,y,z)\). Esta función centraliza las llamadas a las expresiones analíticas de las componentes deducidas en la sección anterior.

\begin{center}
\begin{minted}{python}
def E_field(x, y, z, a, b, sigma):
    """
    Campo eléctrico vectorial $(E_x,E_y,E_z)$ de una placa rectangular (a×b, sigma)
    centrada en el origen y contenida en z=0, evaluado en $(x,y,z)$.

    Args:
        x, y, z (array_like): Coordenadas de evaluación (m).
        a (float): Longitud en x del lado de la placa (m).
        b (float): Longitud en y del lado de la placa (m).
        sigma (float): Densidad superficial de carga (C/m^2).

    Returns:
        tuple[np.ndarray, np.ndarray, np.ndarray]:
            $(E_x, E_y, E_z)$ en V/m, con forma igual al broadcast de $(x,y,z)$.
    """
    # --- Componentes del campo (usa integrales cerradas definidas arriba) ---
    coordenada_en_x = Ex(x, y, z, a, b, sigma)
    coordenada_en_y = Ey(x, y, z, a, b, sigma)
    coordenada_en_z = Ez(x, y, z, a, b, sigma)
    return coordenada_en_x, coordenada_en_y, coordenada_en_z
\end{minted}
\end{center}

A continuación, se implementa la rutina de visualización tridimensional \texttt{plot\_3d}. La versión actual realiza:

\begin{enumerate}
    \item Genera una malla 3D centrada en torno a la placa, con límites
    \(z_{\min}=a\,\texttt{zmin\_ratio}\) y \(z_{\max}=a\,\texttt{zmax\_ratio}\).
    Por defecto, \(z\in[-0.5\,a,\,0.5\,a]\), lo que permite muestrear el campo a ambos lados del plano \(z=0\).
    (Si se desea evitar muestrear exactamente en \(z=0\), basta con elegir \texttt{zmin\_ratio}\(>0\) o \texttt{zmax\_ratio}\(<0\).)
    \item Evalúa \(\vec E\) en toda la malla mediante \texttt{E\_field}, calcula \(|\vec E|\) y \emph{normaliza} cada vector para que todas las flechas tengan la misma longitud; la \emph{intensidad} se codifica por color con el mapa \texttt{viridis} (se incluye barra de color). Para evitar divisiones por cero en la normalización, se añade un término numérico pequeño a \(|\vec E|\).
    \item Dibuja la placa como un rectángulo semitransparente en \(z=0\) (referencia geométrica) y fija una longitud global de flecha \( \texttt{length}=\texttt{scale\_factor}\,\max(a,b)\) (por defecto, \(0.05\,\max(a,b)\)).
    \item Ajusta proporciones, límites y punto de vista (uso de \texttt{ax.set\_box\_aspect} y \texttt{view\_init}) para facilitar la lectura volumétrica; activa una rejilla tenue.
    \item Exporta la figura a \texttt{campo3d-normalizado-colores.svg}; para visualizar en pantalla puede descomentarse \texttt{plt.show()}.
\end{enumerate}

\newpage

\begin{center}
\begin{minted}{python}
def plot_3d(a=1.0, b=0.6, sigma=1.0, Nx=15, Ny=15, Nz=10,  # Aumentado Nx, Ny, Nz
                     zmin_ratio=-0.5, zmax_ratio=0.5, scale_factor=0.05):  # scale_factor reducido a 0.05
    """
    Visualiza con flechas (quiver 3D) el campo eléctrico de una placa rectangular
    centrada en el plano z=0, usando la función E_field(x, y, z, a, b, sigma).
    La longitud de todas las flechas es constante; el color codifica |E|.

    Args:
        a (float): Semilado en x de la placa (m). La placa se extiende de -a/2 a +a/2.
        b (float): Semilado en y de la placa (m). La placa se extiende de -b/2 a +b/2.
        sigma (float): Densidad superficial de carga (C/m^2); se pasa a E_field.
        Nx (int): Número de puntos de muestreo en x para la malla 3D.
        Ny (int): Número de puntos de muestreo en y para la malla 3D.
        Nz (int): Número de puntos de muestreo en z para la malla 3D.
        zmin_ratio (float): Límite inferior del dominio en z en múltiplos de 'a'
            (z_min = a * zmin_ratio).
        zmax_ratio (float): Límite superior del dominio en z en múltiplos de 'a'
            (z_max = a * zmax_ratio).
        scale_factor (float): Factor adimensional que fija la longitud global de las
            flechas como length = scale_factor * max(a, b).

    Details:
        - La magnitud |E| se mapea a colores con el colormap 'viridis' (barra de color incluida).
        - Las direcciones de las flechas provienen de normalizar (E_x, E_y, E_z).
        - Se dibuja un rectángulo semitransparente en z=0 indicando la placa.
   
    Requirements:
        - E_field debe devolver (E_x, E_y, E_z) con la misma forma que (x, y, z).
        - Depende de numpy y matplotlib (mplot3d).

    Returns:
        - Guarda 'campo3d-normalizado-colores.svg' en el directorio actual.
        - Para visualizar en pantalla, descomentar 'plt.show()'.
    """

     # --- Malla 3D centrada (más puntos para mayor densidad de flechas) ---
    X = np.linspace(-a, a, Nx)
    Y = np.linspace(-b, b, Ny)
    Z = np.linspace(a*zmin_ratio, a*zmax_ratio, Nz)
    x, y, z = np.meshgrid(X, Y, Z, indexing='ij')

     # --- Campo y magnitud (para normalizar dirección y colorear) ---
    E_x, E_y, E_z = E_field(x, y, z, a, b, sigma)
    E_magnitud = np.sqrt(E_x**2 + E_y**2 + E_z**2) + 1e-16

    # Dirección normalizada
    E_x_normalizado = E_x / E_magnitud
    E_y_normalizado = E_y / E_magnitud
    E_z_normalizado = E_z / E_magnitud

    # --- Longitud de flechas (constante) ---
    length_quiver = scale_factor * max(a, b)

    # --- Colores por |E| ---
    import matplotlib.cm as cm
    import matplotlib.colors as mcolors
    cmap = cm.get_cmap('viridis')
    norm = mcolors.Normalize(vmin=np.min(E_magnitud), vmax=np.max(E_magnitud))
    colores = cmap(norm(E_magnitud.ravel())).reshape(E_magnitud.shape + (4,))

    # --- Figura y quiver 3D ---
    fig = plt.figure(figsize=(12, 10))
    ax = fig.add_subplot(111, projection='3d')
    ax.quiver(x, y, z, E_x_normalizado, E_y_normalizado, E_z_normalizado,
                  length=length_quiver,
                  normalize=False,
                  colors=colores.reshape(-1, 4),
                  linewidth=1.5, alpha=0.8)  

    # --- Barra de color (magnitud del campo) ---
    # Creamos un mapeador "falso" para la barra de color
    from matplotlib.cm import ScalarMappable
    sm = ScalarMappable(norm=norm, cmap=cmap)
    sm.set_array([])
    cbar = fig.colorbar(sm, ax=ax, shrink=0.55, aspect=20)
    cbar.set_label("Magnitud del campo eléctrico ($V/m$)")

    # --- Placa en z=0 (rectángulo semitransparente) ---
    x_placa = np.array([[-a/2, a/2], [-a/2, a/2]])
    y_placa = np.array([[-b/2, -b/2], [b/2, b/2]])
    z_placa = np.zeros_like(x_placa)
    ax.plot_surface(x_placa, y_placa, z_placa, alpha=0.35, color='purple')

    # --- Ajuste de límites para que las flechas no se salgan ---
    max_range = max(a, b, a*(zmax_ratio - zmin_ratio))
    ax.set_xlim(-2.2*a/2, 2.2*a/2)
    ax.set_ylim(-2.2*b/2, 2.2*b/2)
    ax.set_zlim(-1.3*max_range/2, 1.3*max_range/2)

    # --- Ajuste de ejes y perspectiva ---
    ax.set_xlabel("Coordenada x (m)", labelpad=10)
    ax.set_ylabel("Coordenada y (m)", labelpad=10)
    ax.set_zlabel("Coordenada z (m)", labelpad=10)
    ax.set_title('Campo eléctrico 3D alrededor de placa cargada', pad=20)
    ax.set_box_aspect([1, b/a, (zmax_ratio - zmin_ratio)])
    ax.grid(True, alpha=0.3)
    ax.view_init(elev=20, azim=45)

    # --- Exportar a archivo vectorial (SVG) ---
    plt.tight_layout()
    plt.savefig('campo3d-normalizado-colores.svg', bbox_inches='tight')
    #plt.show()
    plt.close()
    return 'campo3d-normalizado-colores.svg'
\end{minted}
\end{center}

Por último, en la llamada de ejemplo, los parámetros \texttt{Nx}, \texttt{Ny} y \texttt{Nz} controlan la resolución de muestreo de la malla; \texttt{zmin\_ratio} y \texttt{zmax\_ratio} determinan la ventana de visualización en \(z\). El parámetro \texttt{scale\_factor} regula el tamaño visual de las flechas sin alterar su dirección ni la codificación por color.

\begin{center}
\begin{minted}{python}
plot_3d(a=1.0, b=0.6, sigma=1.0, Nx=15, Ny=15, Nz=10, zmin_ratio=-0.5, zmax_ratio=0.5)
\end{minted}
\end{center}

Así se obtiene la figura solicitada; véase la figura~\ref{plot_3d}.

\begin{figure}
    \centering
    \includegraphics[width=1\linewidth]{campo_3D_normalizado.png}
    \caption{Visualización tridimensional del campo eléctrico generado por una placa rectangular finita de lados $a$ (eje $x$) y $b$ (eje $y$) con densidad superficial uniforme $\sigma$. Las flechas representan la \emph{dirección} del campo (vectores normalizados), mientras que la codificación de color corresponde a su \emph{magnitud} total $\|\vec{E}\|$. La superficie semitransparente en $z=0$ indica la ubicación de la placa.}
    \label{plot_3d}
\end{figure}

\clearpage

%%%%%%%%%%%%%%%%%%%%%%%%%%%%%%%%%%%%%%%%%%%%%%%%%%%%%%%%%%%%%%%%%%%%%%%%%%%%%%%%%%%%%%%%%%%%%%%%%%%%%%%%%%%%%%%%%%%%%%%%%%%%%%%%%%%%%%%%%%%%%%%%%%%%%%%%%%%%%%%%%%%%%%%%%%%%%%%%%%%%%%%%%%%%%%%%%%%%%%%
\subsection{Límite de plano infinito}\label{sec:limite-infinito}

Verifique explícitamente que el campo eléctrico obtenido se reduce al de un plano infinito (que puede calcularse fácilmente mediante la ley de Gauss) en el límite apropiado.

\textbf{Solución.}

%%%%%%%%%%%%%%%%%%%%%%%%%%%%%%%%%%%%%%%%%%%%%%%%%%%%%%%%%%%%%%%%%%%%%%%%%%%%%%%%%%%%%%%%%%%%%%%%%%%%%%%%%%%%%%%%%%%%%%%%%%%%%%%%%%%%%%%%%%%%%%%%%%%%%%%%%%%%%%%%%%%%%%%%%%%%%%%%%%%%%%%%%%%%%%%%%%%%%%%

Consideremos, en la expresión del campo eléctrico \eqref{eq:campo-rectangular}, el caso particular $a=b$. Se obtiene
    \begin{align*}
        \vec{E}(\vec{x})
        &= \frac{\sigma}{4\pi\epsilon_0} \left[
        \operatorname{asinh}\!\left(\frac{y+\frac{a}{2}}{\sqrt{(x-\tfrac{a}{2})^2+z^2}}\right)
        -\operatorname{asinh}\!\left(\frac{y-\frac{a}{2}}{\sqrt{(x-\tfrac{a}{2})^2+z^2}}\right)
        \right. \\[6pt]
        &\quad-
        \left.
        \operatorname{asinh}\!\left(\frac{y+\frac{a}{2}}{\sqrt{(x+\tfrac{a}{2})^2+z^2}}\right)
        +\operatorname{asinh}\!\left(\frac{y-\frac{a}{2}}{\sqrt{(x+\tfrac{a}{2})^2+z^2}}\right)
        \right]\,\hat{x} \\[6pt]
        &\quad +
        \frac{\sigma}{4\pi\epsilon_0} \left[
        \operatorname{asinh}\!\left(\frac{x+\tfrac{a}{2}}{\sqrt{(y-\tfrac{a}{2})^2+z^2}}\right)
        -\operatorname{asinh}\!\left(\frac{x-\tfrac{a}{2}}{\sqrt{(y-\tfrac{a}{2})^2+z^2}}\right)
        \right. \\[2pt]
        &\quad-
        \left.
        \operatorname{asinh}\!\left(\frac{x+\tfrac{a}{2}}{\sqrt{(y+\tfrac{a}{2})^2+z^2}}\right)
        +\operatorname{asinh}\!\left(\frac{x-\tfrac{a}{2}}{\sqrt{(y+\tfrac{a}{2})^2+z^2}}\right)
        \right]\,\hat{y}\\[6pt]
        &\quad +
        \frac{\sigma}{4\pi\epsilon_0} 
        \left[
        \arctan\!\left(\frac{(x+\frac{a}{2})\,(y+\frac{a}{2})}{z\,\sqrt{(x+\frac{a}{2})^2+(y+\frac{a}{2})^2+z^2}}\right)
        -\arctan\!\left(\frac{(x+\frac{a}{2})\,(y-\frac{a}{2})}{z\,\sqrt{(x+\frac{a}{2})^2+(y-\frac{a}{2})^2+z^2}}\right)
        \right. \\[2pt]
        &\quad-
        \left.
        \arctan\!\left(\frac{(x-\frac{a}{2})\,(y+\frac{a}{2})}{z\,\sqrt{(x-\frac{a}{2})^2+(y+\frac{a}{2})^2+z^2}}\right)
        +\arctan\!\left(\frac{(x-\frac{a}{2})\,(y-\frac{a}{2})}{z\,\sqrt{(x-\frac{a}{2})^2+(y-\frac{a}{2})^2+z^2}}\right)
        \right]\,\hat{z}.
    \end{align*}

A continuación, en cada término de los argumentos de las funciones $\operatorname{asinh}$ y $\arctan$, factorizamos $a$ y simplificamos $\frac{a}{a}=1$:
    \begin{align*}
        \vec{E}(\vec{x})
        &= \frac{\sigma}{4\pi\epsilon_0} \left[
        \operatorname{asinh}\!\left(\frac{\frac{y}{a}+\frac{1}{2}}{\sqrt{\left(\frac{x}{a}-\frac{1}{2}\right)^2+\left(\frac{z}{a}\right)^2}}\right)
        -\operatorname{asinh}\!\left(\frac{\frac{y}{a}-\frac{1}{2}}{\sqrt{\left(\frac{x}{a}-\frac{1}{2}\right)^2+\left(\frac{z}{a}\right)^2}}\right)
        \right. \\[6pt]
        &\quad-
        \left.
        \operatorname{asinh}\!\left(\frac{\frac{y}{a}+\frac{1}{2}}{\sqrt{\left(\frac{x}{a}+\frac{1}{2}\right)^2+\left(\frac{z}{a}\right)^2}}\right)
        +\operatorname{asinh}\!\left(\frac{\frac{y}{a}-\frac{1}{2}}{\sqrt{\left(\frac{x}{a}+\frac{1}{2}\right)^2+\left(\frac{z}{a}\right)^2}}\right)
        \right]\,\hat{x} \\[6pt]
        &\quad +
        \frac{\sigma}{4\pi\epsilon_0} \left[
        \operatorname{asinh}\!\left(\frac{\frac{x}{a}+\frac{1}{2}}{\sqrt{\left(\frac{y}{a}-\frac{1}{2}\right)^2+\left(\frac{z}{a}\right)^2}}\right)
        -\operatorname{asinh}\!\left(\frac{\frac{x}{a}-\frac{1}{2}}{\sqrt{\left(\frac{y}{a}-\frac{1}{2}\right)^2+\left(\frac{z}{a}\right)^2}}\right)
        \right. \\[2pt]
        &\quad-
        \left.
        \operatorname{asinh}\!\left(\frac{\frac{x}{a}+\frac{1}{2}}{\sqrt{\left(\frac{y}{a}+\frac{1}{2}\right)^2+\left(\frac{z}{a}\right)^2}}\right)
        +\operatorname{asinh}\!\left(\frac{\frac{x}{a}-\frac{1}{2}}{\sqrt{\left(\frac{y}{a}+\frac{1}{2}\right)^2+\left(\frac{z}{a}\right)^2}}\right)
        \right]\,\hat{y} \\[6pt]
        &\quad +
        \frac{\sigma}{4\pi\epsilon_0} 
        \left[
        \arctan\!\left(\frac{\tfrac{a}{z}\left(\tfrac{x}{a}+\tfrac{1}{2}\right)\!\left(\tfrac{y}{a}+\tfrac{1}{2}\right)}{\sqrt{\left(\tfrac{x}{a}+\tfrac{1}{2}\right)^2+\left(\tfrac{y}{a}+\tfrac{1}{2}\right)^2+\left(\tfrac{z}{a}\right)^2}}\right)
        -\arctan\!\left(\frac{\tfrac{a}{z}\left(\tfrac{x}{a}+\tfrac{1}{2}\right)\!\left(\tfrac{y}{a}-\tfrac{1}{2}\right)}{\sqrt{\left(\tfrac{x}{a}+\tfrac{1}{2}\right)^2+\left(\tfrac{y}{a}-\tfrac{1}{2}\right)^2+\left(\tfrac{z}{a}\right)^2}}\right)
        \right. \\[2pt]
        &\quad-
        \left.
        \arctan\!\left(\frac{\tfrac{a}{z}\left(\tfrac{x}{a}-\tfrac{1}{2}\right)\!\left(\tfrac{y}{a}+\tfrac{1}{2}\right)}{\sqrt{\left(\tfrac{x}{a}-\tfrac{1}{2}\right)^2+\left(\tfrac{y}{a}+\tfrac{1}{2}\right)^2+\left(\tfrac{z}{a}\right)^2}}\right)
        +\arctan\!\left(\frac{\tfrac{a}{z}\left(\tfrac{x}{a}-\tfrac{1}{2}\right)\!\left(\tfrac{y}{a}-\tfrac{1}{2}\right)}{\sqrt{\left(\tfrac{x}{a}-\tfrac{1}{2}\right)^2+\left(\tfrac{y}{a}-\tfrac{1}{2}\right)^2+\left(\tfrac{z}{a}\right)^2}}\right)
        \right]\,\hat{z}.
    \end{align*}

Tomando ahora el límite $a\to\infty$,
    \begin{align*}
        \lim_{a\to \infty}\vec{E}(\vec{x})
        &= \lim_{a\to \infty}\frac{\sigma}{4\pi\epsilon_0} \left[
        \operatorname{asinh}\!\left(\frac{\frac{y}{a}+\frac{1}{2}}{\sqrt{\left(\frac{x}{a}-\frac{1}{2}\right)^2+\left(\frac{z}{a}\right)^2}}\right)
        -\operatorname{asinh}\!\left(\frac{\frac{y}{a}-\frac{1}{2}}{\sqrt{\left(\frac{x}{a}-\frac{1}{2}\right)^2+\left(\frac{z}{a}\right)^2}}\right)
        \right. \\[6pt]
        &\quad-
        \left.
        \operatorname{asinh}\!\left(\frac{\frac{y}{a}+\frac{1}{2}}{\sqrt{\left(\frac{x}{a}+\frac{1}{2}\right)^2+\left(\frac{z}{a}\right)^2}}\right)
        +\operatorname{asinh}\!\left(\frac{\frac{y}{a}-\frac{1}{2}}{\sqrt{\left(\frac{x}{a}+\frac{1}{2}\right)^2+\left(\frac{z}{a}\right)^2}}\right)
        \right]\,\hat{x} \\[6pt]
        &\quad +\lim_{a\to \infty}
        \frac{\sigma}{4\pi\epsilon_0} \left[
        \operatorname{asinh}\!\left(\frac{\frac{x}{a}+\frac{1}{2}}{\sqrt{\left(\frac{y}{a}-\frac{1}{2}\right)^2+\left(\frac{z}{a}\right)^2}}\right)
        -\operatorname{asinh}\!\left(\frac{\frac{x}{a}-\frac{1}{2}}{\sqrt{\left(\frac{y}{a}-\frac{1}{2}\right)^2+\left(\frac{z}{a}\right)^2}}\right)
        \right. \\[2pt]
        &\quad-
        \left.
        \operatorname{asinh}\!\left(\frac{\frac{x}{a}+\frac{1}{2}}{\sqrt{\left(\frac{y}{a}+\frac{1}{2}\right)^2+\left(\frac{z}{a}\right)^2}}\right)
        +\operatorname{asinh}\!\left(\frac{\frac{x}{a}-\frac{1}{2}}{\sqrt{\left(\frac{y}{a}+\frac{1}{2}\right)^2+\left(\frac{z}{a}\right)^2}}\right)
        \right]\,\hat{y} \\[6pt]
        &\quad +
        \lim_{a\to \infty}\frac{\sigma}{4\pi\epsilon_0} 
        \left[
        \arctan\!\left(\frac{\tfrac{a}{z}\left(\tfrac{x}{a}+\tfrac{1}{2}\right)\!\left(\tfrac{y}{a}+\tfrac{1}{2}\right)}{\sqrt{\left(\tfrac{x}{a}+\tfrac{1}{2}\right)^2+\left(\tfrac{y}{a}+\tfrac{1}{2}\right)^2+\left(\tfrac{z}{a}\right)^2}}\right)
        -\arctan\!\left(\frac{\tfrac{a}{z}\left(\tfrac{x}{a}+\tfrac{1}{2}\right)\!\left(\tfrac{y}{a}-\tfrac{1}{2}\right)}{\sqrt{\left(\tfrac{x}{a}+\tfrac{1}{2}\right)^2+\left(\tfrac{y}{a}-\tfrac{1}{2}\right)^2+\left(\tfrac{z}{a}\right)^2}}\right)
        \right. \\[2pt]
        &\quad-
        \left.
        \arctan\!\left(\frac{\tfrac{a}{z}\left(\tfrac{x}{a}-\tfrac{1}{2}\right)\!\left(\tfrac{y}{a}+\tfrac{1}{2}\right)}{\sqrt{\left(\tfrac{x}{a}-\tfrac{1}{2}\right)^2+\left(\tfrac{y}{a}+\tfrac{1}{2}\right)^2+\left(\tfrac{z}{a}\right)^2}}\right)
        +\arctan\!\left(\frac{\tfrac{a}{z}\left(\tfrac{x}{a}-\tfrac{1}{2}\right)\!\left(\tfrac{y}{a}-\tfrac{1}{2}\right)}{\sqrt{\left(\tfrac{x}{a}-\tfrac{1}{2}\right)^2+\left(\tfrac{y}{a}-\tfrac{1}{2}\right)^2+\left(\tfrac{z}{a}\right)^2}}\right) 
        \right]\,\hat{z}\\[6pt]
        &= \frac{\sigma}{4\pi\epsilon_0}\left[
        \operatorname{asinh}(1)-\operatorname{asinh}(-1)
        -\operatorname{asinh}(1)+\operatorname{asinh}(-1)
        \right]\hat{x} \\[6pt]
        &\quad+
        \frac{\sigma}{4\pi\epsilon_0}\left[
        \operatorname{asinh}(1)-\operatorname{asinh}(-1)
        -\operatorname{asinh}(1)+\operatorname{asinh}(-1)
        \right]\hat{y} \\[6pt]
        &\quad+
        \frac{\sigma}{4\pi\epsilon_0}\left[
        \operatorname{sgn}(z)\,\frac{\pi}{2}
        -\operatorname{sgn}(z)\!\left(-\frac{\pi}{2}\right)
        -\operatorname{sgn}(z)\!\left(-\frac{\pi}{2}\right)
        +\operatorname{sgn}(z)\,\frac{\pi}{2}
        \right]\hat{z} \\[6pt]
        &= 0\,\hat{x}+0\,\hat{y}+\frac{\sigma}{2\epsilon_0}\,\operatorname{sgn}(z)\,\hat{z}.
    \end{align*}

En conclusión, el campo eléctrico de la placa infinita ubicada en el plano $xy$ posee componentes nulas en las direcciones $x$ e $y$ (de acuerdo con la simetría del sistema) y sólo una componente normal no nula. En forma compacta,
    \begin{equation*}
        \lim_{a\to \infty}\vec{E}(\vec{x})=0\,\hat{x}+0\,\hat{y}+\frac{\sigma}{2\epsilon_0}\,\operatorname{sgn}(z)\,\hat{z},
    \end{equation*}
donde la función signo se define por
    \begin{align*}
        \operatorname{sgn}(z) &=
        \begin{cases}
            +1, & z > 0, \\[6pt]
            0,  & z = 0, \\[6pt]
            -1, & z < 0.
        \end{cases}
    \end{align*}
    
Verifiquemos el mismo resultado con la ley de Gauss. Sea una superficie gaussiana cilíndrica de volumen $V$, cuyo eje es perpendicular al plano y lo corta en el punto medio de su altura (véase la Fig.~\ref{1.4_gaussian}). Denotemos por $S_1$ y $S_2$ las tapas (superficies planas superior e inferior) y por $S_3$ la superficie lateral del cilindro; llamemos $A_1=\text{área}(S_1)$ y $A_2=\text{área}(S_2)$. Entonces
    \begin{equation}\label{1.4_ley_de_gauss}
        \oint_{\partial V} \vec{E}\cdot d\vec{S} = \frac{Q_{enc}}{\epsilon_0}\,.
    \end{equation}
El flujo total es la suma de los flujos por las tres superficies del cilindro:
    \begin{align*}
        \oint_{\partial V} \vec{E}\cdot d\vec{A} 
        &= \int_{S_1} \vec{E}_1\cdot d\vec{A}_1+\int_{S_2} \vec{E}_2\cdot d\vec{A}_2+\int_{S_3} \vec{E}_3\cdot d\vec{A}_3 \\[4pt]
        &= E_1 \int_{S_1} dA_1+ {E}_2\int_{S_2} dA_2+0 \\[4pt]
        &= E_1 A_1+ E_2A_2 \\[4pt]
        &= E A+ EA \\[4pt]
        &=2EA\,.
    \end{align*}
En la superficie lateral $S_3$ el campo eléctrico es tangencial, por lo que $\vec{E}_3 \cdot d\vec{A}_3 = 0$. En contraste, sobre las tapas $S_1$ y $S_2$ el campo es normal y uniforme debido a la simetría del sistema. En este caso, sus magnitudes coinciden, de modo que $E_1 = E_2 =: E$. Cabe señalar que esta igualdad solo se cumple porque el cilindro gaussiano se ha escogido de manera que corta al plano en el punto medio de su altura; de lo contrario, las magnitudes de campo en ambas tapas no serían idénticas. Además, las áreas de las tapas son iguales, $A_1 = A_2 = A$, y los vectores normales externos apuntan en direcciones opuestas, $\pm \hat{z}$. En consecuencia, la contribución neta de flujo a través de las tapas es  
$E A + E A = 2EA$, tal como se muestra en el desarrollo anterior. Por lo tanto,
    \begin{equation}\label{1.4_flujo_calculado}
        \oint_{\partial V} \vec{E}\cdot d\vec{A} =2EA.
    \end{equation}
La carga encerrada por $V$ se relaciona con la densidad superficial constante $\sigma$ y el área $A$ de las tapas mediante
    \begin{equation*}
        \sigma=\frac{Q_{enc}}{A}\,,
    \end{equation*}
de donde
    \begin{equation}\label{1.4_carga_encerrada}
        Q_{enc}=\sigma A.
    \end{equation}
Sustituyendo \eqref{1.4_flujo_calculado} y \eqref{1.4_carga_encerrada} en \eqref{1.4_ley_de_gauss}, se obtiene
    \begin{equation*}
        2EA=\frac{\sigma A}{\epsilon_0}.
    \end{equation*}
Simplificando $A$ y despejando $E$, concluimos
    \begin{equation*}
        E=\frac{\sigma }{2\epsilon_0}.
    \end{equation*}
Como el campo es perpendicular al plano $xy$, su dirección es la del eje $z$ y su sentido viene dado por el signo de $z$. En consecuencia, queda verificado que para la placa infinita con densidad de carga uniforme,
    \begin{equation*}
        \vec{E}(\vec{x})=0\,\hat{x}+0\,\hat{y}+\frac{\sigma}{2\epsilon_0}\,\operatorname{sgn}(z)\,\hat{z}.
    \end{equation*}
    \begin{figure}
        \centering
        \includegraphics[width=0.31\linewidth]{Superficie_gaussiana_y_campo_eléctrico.png}
        \caption{Superficie gaussiana cilíndrica que intersecta el plano $xy$ a mitad de altura. El campo es normal a las tapas y nulo sobre la superficie lateral.}
        \label{1.4_gaussian}
    \end{figure}
    
\clearpage

%%%%%%%%%%%%%%%%%%%%%%%%%%%%%%%%%%%%%%%%%%%%%%%%%%%%%%%%%%%%%%%%%%%%%%%%%%%%%%%%%%%%%%%%%%%%%%%%%%%%%%%%%%%%%%%%%%%%%%%%%%%%%%%%%%%%%%%%%%%%%%%%%%%%%%%%%%%%%%%%%%%%%%%%%%%%%%%%%%%%%%%%%%%%%%%%%%%%%%%

\section{Problema 2}
Considere un gran bloque de material conductor, que tiene en su interior una cavidad esférica vacía 
de radio $R$. En esta cavidad está ubicada una carga puntual $q$ a una distancia $d$ del centro.  
Usando el método de las imágenes, determine:

%%%%%%%%%%%%%%%%%%%%%%%%%%%%%%%%%%%%%%%%%%%%%%%%%%%%%%%%%%%%%%%%%%%%%%%%%%%%%%%%%%%%%%%%%%%%%%%%%%%%%%%%%%%%%%%%%%%%%%%%%%%%%%%%%%%%%%%%%%%%%%%%%%%%%%%%%%%%%%%%%%%%%%%%%%%%%%%%%%%%%%%%%%%%%%%%%%%%%%%

\subsection{Potencial eléctrico}

El potencial eléctrico en todo punto del espacio.

\textbf{Solución:}

%%%%%%%%%%%%%%%%%%%%%%%%%%%%%%%%%%%%%%%%%%%%%%%%%%%%%%%%%%%%%%%%%%%%%%%%%%%%%%%%%%%%%%%%%%%%%%%%%%%%%%%%%%%%%%%%%%%%%%%%%%%%%%%%%%%%%%%%%%%%%%%%%%%%%%%%%%%%%%%%%%%%%%%%%%%%%%%%%%%%%%%%%%%%%%%%%%%%%%%

Consideremos un sistema de coordenadas esféricas $(r,\theta,\varphi)$ cuyo origen $O$ coincide con el centro de la cavidad esférica, donde $r$ es la coordenada radial, $\theta$ la coordenada colatitudinal y $\varphi$ es la coordenada azimutal, es decir:
\begin{equation*}
    r\in[0,\infty),\qquad\theta\in[0,\pi],\qquad\varphi\in[0,2\pi).
\end{equation*}

Sin pérdida de generalidad, alineamos el eje $z$ con la recta que une el centro de la cavidad con la carga real $q$. Así, la posición de la carga real $q$ es $\vec{r}_{1}=d\,\hat{z}$, con $0<d<R$. Introducimos una carga imagen $q'$ situada en $\vec{r}_2=d'\,\hat{z}$, claramente colineal con $q$ (véase la figura \ref{fig:cavidad_q_y_qimagen}). Empleando el principio de superposición, el potencial total en un punto $\vec{r}$ arbitrario del espacio queda definido por:
    \begin{equation}\label{2.1_superposition}
        \Phi(\vec{r})=\frac{1}{4\pi\epsilon_0}\left( \frac{q}{\|\vec{r}-\vec{r}_1\|}+\frac{q'}{\|\vec{r}-\vec{r}_2\|} \right),
    \end{equation}
donde $\epsilon_0$ es la permitividad del vacío.
Suponemos que el bloque conductor está conectado a tierra, por lo que el potencial en todo punto del metal satisface $\Phi(\vec{r})=0$ . En particular, en la superficie de la cavidad esférica ($r=R$), el potencial debe ser cero para todos los ángulos $\theta$ y $\varphi$. Así, se identifica una condición de borde que caracteriza al sistema:
    \begin{equation}\label{2.1_boundary_condition}
        \Phi(r=R,\theta,\varphi)=0, \qquad \forall\varphi\in[0,2\pi),\,\forall\theta\in[0,\pi].
    \end{equation}
Evaluando \eqref{2.1_superposition} en $\vec{r}=R\,\hat{r}$ e imponiendo \eqref{2.1_boundary_condition}, se obtiene:
    \begin{equation*}
        0=\frac{1}{4\pi\epsilon_0}\left( \frac{q}{\|R\,\hat{r}-\vec{r}_1\|}+\frac{q'}{\|R\,\hat{r}-\vec{r}_2\|} \right).
    \end{equation*}
Sustituyendo $\vec{r}_1=d\,\hat{z}$ y $\vec{r}_2=d'\,\hat{z}$, se obtiene:
    \begin{equation}\label{2.1_boundary}
        0=\frac{1}{4\pi\epsilon_0}\left( \frac{q}{\|R\,\hat{r}-d\,\hat{z}\|}+\frac{q'}{\|R\,\hat{r}-d'\,\hat{z}\|} \right).
    \end{equation}
Recordemos que el vector radial unitario $\hat{r}$ en coordenadas esféricas se puede expresar como una combinación lineal de los vectores unitarios en coordenadas cartesianas como:
    \begin{equation*}
        \hat{r}=\cos\varphi\sin\theta\,\hat{x}+\sin\varphi\sin\theta\,\hat{y}+\cos\theta\,\hat{z}.
    \end{equation*}
Así, podemos expresar el vector $\vec{r}=R\,\hat{r}$ de forma:
    \begin{equation*}
        R\,\hat{r}=R\cos\varphi\sin\theta\,\hat{x}+R\sin\varphi\sin\theta\,\hat{y}+R\cos\theta\,\hat{z}.
    \end{equation*}
Ahora calculamos las normas de las diferencias vectoriales $R\,\hat{r}-d\,\hat{z}$:
    \begin{align*}
        \left\|R\,\hat{r}-d\,\hat{z}\right\|       &=\left\|R\cos\varphi\sin\theta\,\hat{x}+R\sin\varphi\sin\theta\,\hat{y}+R\cos\theta\,\hat{z}-d\,\hat{z}\right\|\\[6pt]
        &=\left\|R\cos\varphi\sin\theta\,\hat{x}+R\sin\varphi\sin\theta\,\hat{y}+(R\cos\theta-d)\,\hat{z}\right\|\\[6pt]
        &=\sqrt{ \left(R\cos\varphi\sin\theta\right)^2+\left(R\sin\varphi\sin\theta\right)^2+\left(R\cos\theta-d\right)^2}\\[6pt]
        &=\sqrt{ R^2\sin^2\theta\left(\cos^2\varphi+\sin^2\varphi\right)+R^2\cos^2\theta-2Rd\cos\theta+d^2}\\[6pt]
        &=\sqrt{ R^2\sin^2\theta+R^2\cos^2\theta-2Rd\cos\theta+d^2}\\[6pt]
        &=\sqrt{ R^2\left(\sin^2\theta+\cos^2\theta\right)-2Rd\cos\theta+d^2}\\[6pt]
        &=\sqrt{ R^2-2Rd\cos\theta+d^2}.
    \end{align*}
Por lo tanto:
\begin{equation}\label{2.1_magnitud_d}
    \left\|R\,\hat{r}-d\,\hat{z}\right\|=\sqrt{ R^2+d^2-2Rd\cos\theta}.
\end{equation}
Este resultado coincide con la ley de cosenos. De manera análoga, para el caso de $\|R\,\hat{r}-d'\,\hat{z}\|$ se obtiene:
\begin{equation}\label{2.1_magnitud_d'}
    \left\|R\,\hat{r}-d'\,\hat{z}\right\|=\sqrt{ R^2+d'^2-2Rd'\cos\theta}.
\end{equation}

Sustituyendo \eqref{2.1_magnitud_d} y \eqref{2.1_magnitud_d'} en \eqref{2.1_boundary}, se obtiene:
    \begin{equation*}
        0=\frac{1}{4\pi\epsilon_0}\left( \frac{q}{\sqrt{ R^2+d^2-2Rd\cos\theta}}+\frac{q'}{\sqrt{ R^2+d'^2-2Rd'\cos\theta}} \right).
    \end{equation*}
Por consiguiente,
    \begin{equation*}
        \frac{q}{\sqrt{ R^2+d^2-2Rd\cos\theta}}=-\frac{q'}{\sqrt{ R^2+d'^2-2Rd'\cos\theta}}, \qquad \forall \,\theta\in[0,\pi].
    \end{equation*}
Expresando las cargas en el miembro izquierdo y las raíces en el derecho, se obtiene que la relación entre las cargas y las raíces debe ser igual a una constante real $k>0$  (no confundir con el factor de Coulomb). La positividad de ambas raíces implica además que $q$ y $q'$ tienen signo opuesto para así mantener la coherencia en la ecuación.
    \begin{equation*}
        -\frac{q}{q'}=\frac{\sqrt{ R^2+d^2-2Rd\cos\theta}}{\sqrt{ R^2+d'^2-2Rd'\cos\theta}}=k ,\qquad k\in\mathbb{R}^+.
    \end{equation*}

De esta relación se obtienen dos ecuaciones:
    \begin{align}
        -\frac{q}{q'}&=k, \label{2.1_relacion_cargas}\\[6pt]
        \frac{\sqrt{ R^2+d^2-2Rd\cos\theta}}{\sqrt{ R^2+d'^2-2Rd'\cos\theta}}&=k.\label{2.1_relacion_raices}
    \end{align}
Elevando al cuadrado \eqref{2.1_relacion_raices}, desarrollamos:
    \begin{align*}
        \frac{R^2+d^2-2Rd\cos\theta}{R^2+d'^2-2Rd'\cos\theta}&=k^2  \\[6pt]
        \Rightarrow \quad R^2+d^2-2Rd\cos\theta&= k^2\left(R^2+d'^2-2Rd'\cos\theta\right).
    \end{align*}
Igualando los coeficientes de $\cos\theta$ y los términos independientes, se obtiene:
    \begin{itemize}
        \item \textbf{Coeficientes de $\cos\theta$: }
            \begin{align}
                -2Rd= -2k^2Rd'\quad 
                &\Rightarrow \quad k^2 = \frac{d}{d'} \notag \\[4pt]
                &\Rightarrow \quad k = \sqrt{ \frac{d}{d'}},\label{2.1_constante_k}
            \end{align}
        \item \textbf{Términos constantes: }
            \begin{equation}\label{2.1_terminos_constantes}
                R^2 + d^2 = k^2\,(R^2 + d'^2).
            \end{equation}
    \end{itemize}
Sustituyendo \eqref{2.1_constante_k} en \eqref{2.1_terminos_constantes}, se deduce:
    \begin{align*}
        &\qquad R^2 + d^2 = \left(\sqrt{\frac{d}{d'}}\right)^2\,(R^2 + d'^2) \\[2pt]
        &\Rightarrow \quad R^2 + d^2 = \frac{d}{d'}\,(R^2 + d'^2) \\[4pt]
        &\Rightarrow \quad (R^2 + d^2)\,d' = d\,(R^2 + d'^2)\\[6pt]
        &\Rightarrow \quad d\,d'^2 - (R^2 + d^2)\,d' + dR^2 = 0.
    \end{align*}
Dividiendo por $d$ (donde $d\neq 0$), se obtiene la ecuación cuadrática:
    \begin{equation*}
        d'^2 - \frac{R^2 + d^2}{d}\,d' + R^2 = 0,
    \end{equation*}
cuya solución general viene dada por:
    \begin{equation*}
        d' = \frac{R^2 + d^2}{2d} \,\pm\, \frac{1}{2d}\sqrt{(R^2 + d^2)^2 - 4R^2d^2}.
    \end{equation*}
Donde el discriminante de esta ecuación cuadrática puede reescribirse como:
    \begin{equation*}
        (R^2 + d^2)^2 - 4R^2d^2 = (R^2 - d^2)^2.
    \end{equation*}
De este modo, la ecuación cuadrática queda expresada de la forma:
    \begin{equation*}
        d' = \frac{1}{2d}\Big[(R^2 + d^2) \pm (R^2 - d^2)\Big].
    \end{equation*}
Las soluciones son:
    \begin{equation*}
        d' = \frac{1}{2d}(2R^2) = \frac{R^2}{d}, \qquad
        d' = \frac{1}{2d}(2d^2) = d.
    \end{equation*}
Obsérvese que, si $d'=d$, entonces $k=1$ y $q'=-q$, lo que anularía la fuente neta dentro de la cavidad; por tanto, se descarta esa raíz. En consecuencia, la raíz admisible es 
    \begin{equation}\label{2.1_d'}
        d' = \frac{R^2}{d}, 
    \end{equation}
donde se sigue que la carga imagen se ubica fuera de la cavidad (dentro del conductor), ya que
    \begin{equation*}
        0<d<R\quad\Rightarrow \quad 1<\frac{R}{d}
        \Rightarrow \quad R<\frac{R^2}{d}=d',
    \end{equation*}
por tanto, se verifica que, en efecto, la carga imagen no se encuentra en la cavidad esférica.


Así, con \eqref{2.1_d'}, el vector de posición de la carga imagen es:
    \begin{equation}\label{2.1_vector_posicion_carga_imagen}
        \vec{r}_2=d'\,\hat{z}=\frac{R^2}{d}\,\hat{z}.
    \end{equation}
Sustituyendo \eqref{2.1_d'} en \eqref{2.1_constante_k}:
    \begin{align*}
         k &= \sqrt{ \frac{d}{ \left(  \frac{R^2}{d}\right) }}\\[4pt]
         &=\sqrt{ \frac{d^2}{  R^2 }} \\[4pt]
         &=\frac{d}{R}.
    \end{align*}
Se obtiene:
    \begin{equation}\label{2.1_constante_k_explicita}
        k=\frac{d}{R}.
    \end{equation}
Sustituyendo \eqref{2.1_constante_k_explicita} en \eqref{2.1_relacion_cargas}:
    \begin{equation*}
        -\frac{q}{q'}=\frac{d}{R},
    \end{equation*}
despejando, se obtiene para la carga imagen:
    \begin{equation}\label{2.1_carga_imagen}
        q'=-\frac{R}{d}\,q.
    \end{equation}
Por lo tanto, la coordenada de la posición de la carga imagen, para el caso particular donde esta es colineal con la carga real en el eje $z$, y la magnitud de la primera quedan determinadas por:
    \begin{equation*}
        q'=-\frac{R}{d}\,q , \qquad d' = \frac{R^2}{d}.
    \end{equation*}
Por el teorema de unicidad del problema de Dirichlet, la solución construida es única en la cavidad $0<r<R$ (salvo en $\vec r_1$); en consecuencia, puede evaluarse en cualquier punto de dicha región:
    \begin{equation*}
        \vec r(r,\theta,\varphi)=r\sin\theta\cos\varphi\,\hat{x}+r\sin\theta\sin\varphi\,\hat{y}+r\cos\theta\,\hat{z}.
    \end{equation*}
Para el potencial eléctrico en todo el espacio, consideraremos la ecuación \eqref{2.1_superposition}, donde para un punto dentro de la cavidad ($\|\vec{r}\|<R$), queda dado por:
    \begin{equation}\label{2.1_potencial_cavidad_esferica}
        \Phi(\vec{r}) = \frac{1}{4\pi\epsilon_0}\left(
        \frac{q}{\left\|\vec{r}-d\,\hat{z}\right\|}
        + \frac{q'}{\left\|\vec{r}-\tfrac{R^2}{d}\,\hat{z}\right\|}
        \right), \qquad\|\vec{r}\|<R, \; \vec{r}\neq \vec{r}_1  .  
    \end{equation}
donde $\vec{r}_1=d\,\hat{z}$ es la posición de la carga real y $\vec{r}_2=\tfrac{R^2}{d}\,\hat{z}$ la de la carga imagen. Nótese que, si $\vec{r}=\vec{r}_1$, se reproduce la singularidad, justamente en la posición de la carga real, por lo que se excluye el punto $\vec{r}=\vec{r}_1$.

Por otra parte, para un punto $\vec{r}$ en el espacio fuera de la cavidad esférica ($\|\vec{r}\|\geq R$), es decir, en el material conductor, el potencial es nulo porque el sistema está conectado a tierra
    \begin{equation*}
        \Phi(\vec{r})=0, \qquad  \|\vec{r}\|\ge R.
    \end{equation*}

Reuniendo los resultados anteriores, el potencial queda definido por tramos:
    \begin{equation}\label{2.1_potencial_en_todo_el_espacio}
        \Phi(\vec{r})= \left\{
        \begin{aligned}
            \frac{1}{4\pi\epsilon_0}\left(
            \frac{q}{\left\|\vec{r}-d\,\hat{z}\right\|}
            + \frac{q'}{\left\|\vec{r}-\tfrac{R^2}{d}\,\hat{z}\right\|}
            \right)&, \quad \text{si} \quad 0<\|\vec{r}\|<R\,,\, \vec{r}\neq \vec{r}_1 ,  \\
            0\qquad \qquad \qquad&, \quad \text{si} \quad \|\vec{r}\|\geq R.
        \end{aligned} \right.
    \end{equation}

Obsérvese que la continuidad de $\Phi$ en $r=R$ se satisface por construcción, ya que la condición de frontera \eqref{2.1_boundary_condition} impone $\Phi(R,\theta,\varphi)=0$ para todo ángulo.

Cabe mencionar que si el bloque fuese aislado y eléctricamente neutro (en lugar de estar conectado a tierra), la solución interior seguiría describiéndose mediante $q$ y $q'$ como en \eqref{2.1_potencial_cavidad_esferica}, pero el potencial en el conductor sería una constante calibrable $\Phi_0\neq 0$ y aparecería una carga efectiva $+q$ distribuida sobre la superficie exterior. En este trabajo consideramos exclusivamente el caso a tierra.

\begin{figure}[htbp]
  \centering
  \includegraphics[width=.88\linewidth]{fig_cavidad_esferica.png}
  \caption{Cavidad esférica de radio $R$ con eje $z$ vertical. La carga real $q$ se ubica en $d\,\hat{z}$ con $0<d<R$, mientras que la carga imagen se coloca sobre el mismo eje en $d'\,\hat{z}$ con $d'=R^{2}/d$, colineal con $q$. La esfera representa la superficie conductora de la cavidad; la línea segmentada enfatiza la colinealidad sobre el eje $z$.}
  \label{fig:cavidad_q_y_qimagen}
\end{figure}

\clearpage
%%%%%%%%%%%%%%%%%%%%%%%%%%%%%%%%%%%%%%%%%%%%%%%%%%%%%%%%%%%%%%%%%%%%%%%%%%%%%%%%%%%%%%%%%%%%%%%%%%%%%%%%%%%%%%%%%%%%%%%%%%%%%%%%%%%%%%%%%%%%%%%%%%%%%%%%%%%%%%%%%%%%%%%%%%%%%%%%%%%%%%%%%%%%%%%%%%%%%%%

\subsection{Campo eléctrico}

El campo eléctrico en todo punto del espacio.

\textbf{Solución:}

%%%%%%%%%%%%%%%%%%%%%%%%%%%%%%%%%%%%%%%%%%%%%%%%%%%%%%%%%%%%%%%%%%%%%%%%%%%%%%%%%%%%%%%%%%%%%%%%%%%%%%%%%%%%%%%%%%%%%%%%%%%%%%%%%%%%%%%%%%%%%%%%%%%%%%%%%%%%%%%%%%%%%%%%%%%%%%%%%%%%%%%%%%%%%%%%%%%%%%%

Para calcular el campo eléctrico generado por la carga $q$ al interior de la cavidad esférica, recordemos que en electrostática el campo $\vec{E}$ es conservativo y se relaciona con el potencial escalar $\Phi$ como el \emph{negativo} de su gradiente:
    \begin{equation}\label{2.2_E=-nablaPhi}
        \vec{E}=-\vec{\nabla}\Phi.
    \end{equation}

Donde el gradiente en coordenadas esféricas está dado por:
    \begin{equation}\label{2.2_gradiante}
    \vec{\nabla}=\hat{r}\,\frac{\partial}{\partial r}
    +\hat{\theta}\,\frac{1}{r}\frac{\partial}{\partial \theta}
    +\hat{\varphi}\,\frac{1}{r\sin\theta}\frac{\partial}{\partial \varphi}.
\end{equation}

Considerando el potencial eléctrico obtenido en \eqref{2.1_potencial_en_todo_el_espacio}, es claro que para todo punto $\vec r$ tal que $\|\vec r\|>R$, según \eqref{2.2_E=-nablaPhi}, se tiene
    \begin{equation}\label{2.2_campo_electrico_en_conductor}
        \vec{E}(\vec{r})=\vec 0, \qquad \|\vec r\|> R.
    \end{equation}
Esto implica que el campo eléctrico generado por la carga real $q$ sólo puede ser no nulo dentro de la cavidad esférica, es decir, en puntos del espacio donde $0<\|\vec{r}\|\leq R$ y $\vec{r}\neq \vec{r}_1$.
Para comprobar esto consideramos la ecuación \eqref{2.1_potencial_cavidad_esferica}, donde, como ya hemos demostrado en \eqref{2.1_magnitud_d}, las magnitudes de los vectores de diferencia $\vec r - d\,\hat z$ y $\vec r - d'\,\hat z$ están dadas por:
    \begin{align}
        \left\|\vec{r}-d\,\hat{z}\right\|&=\sqrt{ r^2+d^2-2rd\cos\theta},\label{2.2_magnitud_d}\\[6pt]
        \left\|\vec{r}-d'\,\hat{z}\right\|&=\sqrt{ r^2+d'^2-2rd'\cos\theta}.\label{2.2_magnitud_d'}
    \end{align}
Por hacer más compactas las expresiones en los cálculos siguientes se expresará $\frac{R^2}{d}$ como $d'$, lo cual es válido ya que se demostró en \eqref{2.1_d'}.

Procedemos entonces a calcular el gradiente de \eqref{2.1_potencial_cavidad_esferica}:
    \begin{align*}
        \vec{\nabla}\Phi(\vec{r})&=\vec{\nabla}\left[ \frac{1}{4\pi\epsilon_0}\left(
            \frac{q}{\left\|\vec{r}-d\,\hat{z}\right\|}
            + \frac{q'}{\left\|\vec{r}-d'\,\hat{z}\right\|}
            \right) \right]\\[6pt]
            &=\frac{1}{4\pi\epsilon_0}\left[
            \vec{\nabla}\left( \frac{q}{\left\|\vec{r}-d\,\hat{z}\right\|}
            +  \frac{q'}{\left\|\vec{r}-d'\,\hat{z}\right\|}\right)
            \right].
    \end{align*}
Sustituyendo \eqref{2.2_magnitud_d} y \eqref{2.2_magnitud_d'} se obtiene:
    \begin{equation}\label{2.2_nabla_potencial}
        \vec{\nabla}\Phi(\vec{r})=\frac{1}{4\pi\epsilon_0}\left[
        \vec{\nabla}\left( \frac{q}{\sqrt{ r^2+d^2-2rd\cos\theta}}
        + \frac{q'}{\sqrt{ r^2+d'^2-2rd'\cos\theta}}\right)
        \right].
    \end{equation}

Sustituyendo el gradiente en coordenadas esféricas \eqref{2.2_gradiante} en \eqref{2.2_nabla_potencial}, se tiene:
    \begin{align*}
        \vec{\nabla}\Phi(\vec{r})&=\frac{1}{4\pi\epsilon_0}\left[
        \frac{\partial}{\partial r}\left( \frac{q}{\sqrt{ r^2+d^2-2rd\cos\theta}}
        + \frac{q'}{\sqrt{ r^2+d'^2-2rd'\cos\theta}}\right)\hat{r} \right.\\[6pt]
        &\quad+\frac{1}{r\sin\theta}\frac{\partial}{\partial \varphi}\left( \frac{q}{\sqrt{ r^2+d^2-2rd\cos\theta}}
        + \frac{q'}{\sqrt{ r^2+d'^2-2rd'\cos\theta}}\right)\hat{\varphi}\\[6pt]
        &\left.\quad+\frac{1}{r}\frac{\partial}{\partial \theta}\left( \frac{q}{\sqrt{ r^2+d^2-2rd\cos\theta}}
        + \frac{q'}{\sqrt{ r^2+d'^2-2rd'\cos\theta}}\right)\hat{\theta} \right].
    \end{align*}
Como el potencial eléctrico no tiene dependencia en la coordenada azimutal $\varphi$, es decir
\begin{equation*}
    \frac{\partial \Phi}{\partial \varphi}(\vec{r})=0,
\end{equation*}
el problema para determinar el gradiente del potencial eléctrico se reduce a calcular solo dos derivadas parciales, en este caso con respecto a la coordenada radial $r$ y colatitudinal $\theta$, es decir:
    \begin{align}
        \vec{\nabla}\Phi(\vec{r})&=\frac{1}{4\pi\epsilon_0}\left[
        \frac{\partial}{\partial r}\left( \frac{q}{\sqrt{ r^2+d^2-2rd\cos\theta}}
        + \frac{q'}{\sqrt{ r^2+d'^2-2rd'\cos\theta}}\right)\hat{r} \right. \notag \\[6pt]
        &\left.\quad+\frac{1}{r}\frac{\partial}{\partial \theta}\left( \frac{q}{\sqrt{ r^2+d^2-2rd\cos\theta}}
        + \frac{q'}{\sqrt{ r^2+d'^2-2rd'\cos\theta}}\right)\hat{\theta} \right]. \label{2.2_nabla_potencial_explicito}
    \end{align}

Calculamos entonces dichas derivadas parciales:

Para la derivada parcial con respecto a la coordenada radial $r$:
\begin{align*}
    \frac{\partial}{\partial r}\left( \frac{q}{\sqrt{ r^2+d^2-2rd\cos\theta}}
        + \frac{q'}{\sqrt{ r^2+d'^2-2rd'\cos\theta}}\right)
        &=q\frac{\partial}{\partial r}\left( \frac{1}{\sqrt{ r^2+d^2-2rd\cos\theta}}\right)
        + q'\frac{\partial}{\partial r}\left(\frac{1}{\sqrt{ r^2+d'^2-2rd'\cos\theta}}\right) \\[6pt]
        &=q\left[ \frac{-\frac{1}{2}(2r-2d\cos\theta)}{\left(r^2+d^2-2rd\cos\theta\right)^{\frac{3}{2}}}\right]
        + q'\left[\frac{-\frac{1}{2}(2r-2d'\cos\theta)}{\left( r^2+d'^2-2rd'\cos\theta \right)^{\frac{3}{2}}}\right] \\[6pt]
        &=\frac{-q(r-d\cos\theta)}{\left(r^2+d^2-2rd\cos\theta\right)^{\frac{3}{2}}}
        + \frac{-q'(r-d'\cos\theta)}{\left( r^2+d'^2-2rd'\cos\theta \right)^{\frac{3}{2}}}.
\end{align*}
Por lo tanto:
\begin{equation}\label{2.2_partial_r}
    \frac{\partial}{\partial r}\left( \frac{q}{\sqrt{ r^2+d^2-2rd\cos\theta}}
        + \frac{q'}{\sqrt{ r^2+d'^2-2rd'\cos\theta}}\right)= \frac{-q(r-d\cos\theta)}{\left(r^2+d^2-2rd\cos\theta\right)^{\frac{3}{2}}}
        + \frac{-q'(r-d'\cos\theta)}{\left( r^2+d'^2-2rd'\cos\theta \right)^{\frac{3}{2}}}.
\end{equation}

Para la derivada parcial con respecto a la coordenada colatitudinal $\theta$:
\begin{align*}
    \frac{\partial}{\partial \theta}\left( \frac{q}{\sqrt{ r^2+d^2-2rd\cos\theta}}
        + \frac{q'}{\sqrt{ r^2+d'^2-2rd'\cos\theta}}\right)
        &=q\frac{\partial}{\partial \theta}\left( \frac{1}{\sqrt{ r^2+d^2-2rd\cos\theta}}\right)
        + q'\frac{\partial}{\partial \theta}\left(\frac{1}{\sqrt{ r^2+d'^2-2rd'\cos\theta}}\right) \\[6pt]
        &=q\left[ \frac{-\frac{1}{2}(2dr\sin\theta)}{\left(r^2+d^2-2rd\cos\theta\right)^{\frac{3}{2}}}\right]
        + q'\left[\frac{-\frac{1}{2}(2d'r\sin\theta)}{\left( r^2+d'^2-2rd'\cos\theta \right)^{\frac{3}{2}}}\right] \\[6pt]
        &=\frac{-qdr\sin\theta}{\left(r^2+d^2-2rd\cos\theta\right)^{\frac{3}{2}}}
        + \frac{-q'd'r\sin\theta}{\left( r^2+d'^2-2rd'\cos\theta \right)^{\frac{3}{2}}}. 
\end{align*}
Por lo tanto:
\begin{equation}\label{2.2_partial_theta}
    \frac{\partial}{\partial \theta}\left( \frac{q}{\sqrt{ r^2+d^2-2rd\cos\theta}}
        + \frac{q'}{\sqrt{ r^2+d'^2-2rd'\cos\theta}}\right)=\frac{-qdr\sin\theta}{\left(r^2+d^2-2rd\cos\theta\right)^{\frac{3}{2}}}
        + \frac{-q'd'r\sin\theta}{\left( r^2+d'^2-2rd'\cos\theta \right)^{\frac{3}{2}}}.
\end{equation}

Sustituyendo \eqref{2.2_partial_r} y \eqref{2.2_partial_theta} en \eqref{2.2_nabla_potencial_explicito}:
    \begin{align*}
        \vec{\nabla}\Phi(\vec{r})&=\frac{1}{4\pi\epsilon_0}\left\{
        \left[ \frac{-q(r-d\cos\theta)}{\left(r^2+d^2-2rd\cos\theta\right)^{\frac{3}{2}}}
        + \frac{-q'(r-d'\cos\theta)}{\left( r^2+d'^2-2rd'\cos\theta \right)^{\frac{3}{2}}}\right]\hat{r} \right. \notag \\[6pt]
        &\left.\quad+ \frac{1}{r}\left[ \frac{-qdr\sin\theta}{\left(r^2+d^2-2rd\cos\theta\right)^{\frac{3}{2}}}
        + \frac{-q'd'r\sin\theta}{\left( r^2+d'^2-2rd'\cos\theta \right)^{\frac{3}{2}}}\right]\hat{\theta} \right\} \\[6pt]
        &=-\frac{1}{4\pi\epsilon_0}\left\{
        \left[ \frac{q(r-d\cos\theta)}{\left(r^2+d^2-2rd\cos\theta\right)^{\frac{3}{2}}}
        + \frac{q'(r-d'\cos\theta)}{\left( r^2+d'^2-2rd'\cos\theta \right)^{\frac{3}{2}}}\right]\hat{r} \right. \notag \\[6pt]
        &\left.\quad+ \left[ \frac{qd\sin\theta}{\left(r^2+d^2-2rd\cos\theta\right)^{\frac{3}{2}}}
        + \frac{q'd'\sin\theta}{\left( r^2+d'^2-2rd'\cos\theta \right)^{\frac{3}{2}}}\right]\hat{\theta} \right\} \\[6pt]
        &=-\frac{1}{4\pi\epsilon_0}\left\{
        \frac{q\left[(r-d\cos\theta)\,\hat{r}+d\sin\theta\, \hat{\theta}\right]}{\left(r^2+d^2-2rd\cos\theta\right)^{\frac{3}{2}}}
        + \frac{q'\left[(r-d'\cos\theta)\,\hat{r}+d'\sin\theta\, \hat{\theta}\right] }{\left( r^2+d'^2-2rd'\cos\theta \right)^{\frac{3}{2}}}\right\}.
    \end{align*}
Sustituyendo \eqref{2.2_magnitud_d} y \eqref{2.2_magnitud_d'}, se obtiene:
    \begin{equation}\label{2.2_nabla_potencial_calculado_proto}
         \vec{\nabla}\Phi(\vec{r})=-\frac{1}{4\pi\epsilon_0}\left\{
            \frac{q\left[(r-d\cos\theta)\,\hat{r}+d\sin\theta\, \hat{\theta}\right]}{\|\vec{r}-d\,\hat{z}\|^3}
            + \frac{q'\left[(r-d'\cos\theta)\,\hat{r}+d'\sin\theta\, \hat{\theta}\right] }{\|\vec{r}-d'\,\hat{z}\|^3}\right\}.
    \end{equation}
Recordemos que el vector radial unitario $\hat{r}$ y el vector colatitudinal unitario $\hat{\theta}$ pueden ser expresados en el sistema de coordenadas cartesianas como: 
    \begin{align}
    \hat{r}&=\cos\varphi\sin\theta\,\hat{x}+\sin\varphi\sin\theta\,\hat{y}+\cos\theta\,\hat{z}, \label{2.2_radial_unitario}\\[6pt]
        \hat{\theta}&=\cos\varphi\cos\theta\,\hat{x}+\sin\varphi\cos\theta\,\hat{y}-\sin\theta\,\hat{z}.\label{2.2_colatitudinal_unitario}
    \end{align}
Sustituyendo \eqref{2.2_radial_unitario} y \eqref{2.2_colatitudinal_unitario} en la expresión $(r-d\cos\theta)\,\hat{r}+d\sin\theta\, \hat{\theta}$:
    \begin{align*}
        (r-d\cos\theta)\,\hat{r}+d\sin\theta\, \hat{\theta}
        &=(r-d\cos\theta)(\cos\varphi\sin\theta\,\hat{x}+\sin\varphi\sin\theta\,\hat{y}+\cos\theta\,\hat{z})+d\sin\theta\, (\cos\varphi\cos\theta\,\hat{x}+\sin\varphi\cos\theta\,\hat{y}-\sin\theta\,\hat{z}) \\[4pt]
        &=(r\cos\varphi\sin\theta-d\cos\theta\cos\varphi\sin\theta)\,\hat{x}+(r\sin\varphi\sin\theta-d\cos\theta\sin\varphi\sin\theta)\,\hat{y}+(r\cos\theta-d\cos^2\theta)\,\hat{z} \\[4pt]
        &\quad +\, d\sin\theta\cos\varphi\cos\theta\,\hat{x} +d\sin\theta\sin\varphi\cos\theta\,\hat{y} -d\sin^2\theta\,\hat{z} \\[4pt]
        &=r\cos\varphi\sin\theta\,\hat{x}+r\sin\varphi\sin\theta\,\hat{y}+r\cos\theta\,\hat{z}+(d\sin\theta\cos\varphi\cos\theta-d\cos\theta\cos\varphi\sin\theta)\,\hat{x} \\[4pt]
        &\quad\,+(d\sin\theta\sin\varphi\cos\theta-d\cos\theta\sin\varphi\sin\theta)\,\hat{y}-(d\cos^2\theta+d\sin^2\theta)\,\hat{z} \\[4pt]
        &=r\,(\cos\varphi\sin\theta\,\hat{x}+\sin\varphi\sin\theta\,\hat{y}+\cos\theta\,\hat{z})+0\,\hat{x}+0\,\hat{y}-d\,(\cos^2\theta+\sin^2\theta)\,\hat{z} \\[4pt]
        &=r\,\hat{r}-d\,\hat{z} \\[4pt]
        &=\vec{r}-d\,\hat{z}.
    \end{align*}
Por lo tanto:
    \begin{equation}\label{2.2_diferencia_punto_carga_real}
        (r-d\cos\theta)\,\hat{r}+d\sin\theta\, \hat{\theta}=\vec{r}-d\,\hat{z}.
    \end{equation}
De manera análoga, se puede demostrar que:
    \begin{equation}\label{2.2_diferencia_punto_carga_imagen}
        (r-d'\cos\theta)\,\hat{r}+d'\sin\theta\, \hat{\theta}=\vec{r}-d'\,\hat{z}.
    \end{equation}
Sustituyendo \eqref{2.2_diferencia_punto_carga_real} y \eqref{2.2_diferencia_punto_carga_imagen} en \eqref{2.2_nabla_potencial_calculado_proto}, se compacta el gradiente del potencial eléctrico de forma:
    \begin{equation}\label{2.2_nabla_potencial_calculado}
         \vec{\nabla}\Phi(\vec{r})=-\frac{1}{4\pi\epsilon_0}\left(
            q\,\frac{\vec{r}-d\,\hat{z}}{\|\vec{r}-d\,\hat{z}\|^3}
            + q'\,\frac{\vec{r}-d'\,\hat{z}}{\|\vec{r}-d'\,\hat{z}\|^3}\right).
    \end{equation}
Sustituyendo \eqref{2.2_nabla_potencial_calculado} en \eqref{2.2_E=-nablaPhi}, se obtiene:
    \begin{equation}\label{2.2_campo_electrico}
         \vec{E}(\vec{r})=\frac{1}{4\pi\epsilon_0}\left(
            q\,\frac{\vec{r}-d\,\hat{z}}{\|\vec{r}-d\,\hat{z}\|^3}
            + q'\,\frac{\vec{r}-d'\,\hat{z}}{\|\vec{r}-d'\,\hat{z}\|^3}\right),
    \end{equation}
donde $\vec{r}$ es un vector posición con $0<\|\vec{r}\|\leq R$, entendiéndose el valor en la superficie como el límite interior: $\ \vec E(R^-,\theta,\varphi):=\lim_{r\to R^-}\vec E(r,\theta,\varphi)$.
 Conviene recordar que la carga imagen $q'$ es un artificio matemático para imponer la condición de contorno; no obstante, el campo interior calculado coincide con el campo físico de la configuración original. Así, sustituyendo \eqref{2.1_carga_imagen} y \eqref{2.1_d'} en \eqref{2.2_campo_electrico},
        \begin{equation*}
         \vec{E}(\vec{r})=\frac{q}{4\pi\epsilon_0}\left(
            \frac{\vec{r}-d\,\hat{z}}{\|\vec{r}-d\,\hat{z}\|^3}
            - \frac{R}{d}\,\frac{\vec{r}-\tfrac{R^2}{d}\,\hat{z}}{\left\|\vec{r}-\tfrac{R^2}{d}\,\hat{z}\right\|^3}\right),
    \end{equation*}
se obtiene el campo eléctrico generado por la carga real $q$ en la cavidad esférica de radio $R$ a una distancia $d$ del centro de esta, ubicada sobre el eje $z$ en el sistema de referencia considerado para analizar el problema. Nótese que el campo depende exclusivamente de los parámetros que caracterizan la configuración del sistema.

En resumen, el campo eléctrico asociado al sistema estará dado por:

\begin{equation}\label{2.2_campo_electrico_en_todo_el_espacio}
    \vec{E}(\vec{r})= \left\{
    \begin{aligned}
        \frac{q}{4\pi\epsilon_0 }\left(
            \frac{\vec{r}-d\,\hat{z}}{\|\vec{r}-d\,\hat{z}\|^3}
            - \frac{R}{d}\,\frac{\vec{r}-\tfrac{R^2}{d}\,\hat{z}}{\left\|\vec{r}-\tfrac{R^2}{d}\,\hat{z}\right\|^3}\right)&, \quad \text{si} \quad 0<\|\vec{r}\|\leq R\,,\, \vec{r}\neq \vec{r}_1 , \\
        \vec{0}\qquad \qquad \qquad&, \quad \text{si} \quad \|\vec{r}\|> R.
    \end{aligned} \right.
\end{equation}
En la superficie de la cavidad se emplea el valor interior $\ \vec E(R^-,\theta)=\lim_{r\to R^-}\vec E(r,\theta)$, mientras que en el metal $\ \vec E(R^+,\theta)=\vec 0$.

\clearpage
%%%%%%%%%%%%%%%%%%%%%%%%%%%%%%%%%%%%%%%%%%%%%%%%%%%%%%%%%%%%%%%%%%%%%%%%%%%%%%%%%%%%%%%%%%%%%%%%%%%%%%%%%%%%%%%%%%%%%%%%%%%%%%%%%%%%%%%%%%%%%%%%%%%%%%%%%%%%%%%%%%%%%%%%%%%%%%%%%%%%%%%%%%%%%%%%%%%%%%%

\subsection{Cargas inducidas}

La distribución de cargas inducidas sobre la superficie del conductor que define la cavidad.

\textbf{Solución:}

%%%%%%%%%%%%%%%%%%%%%%%%%%%%%%%%%%%%%%%%%%%%%%%%%%%%%%%%%%%%%%%%%%%%%%%%%%%%%%%%%%%%%%%%%%%%%%%%%%%%%%%%%%%%%%%%%%%%%%%%%%%%%%%%%%%%%%%%%%%%%%%%%%%%%%%%%%%%%%%%%%%%%%%%%%%%%%%%%%%%%%%%%%%%%%%%%%%%%%%

Según la sección 1.6 de la página 31 del libro \textit{Classical Electrodynamics} de John David Jackson, 
la densidad superficial de carga $\sigma(\vec{r})$ está determinada por los campos eléctricos $\vec{E}_1$ y $\vec{E}_2$ 
que atraviesan dicha superficie por debajo y por arriba, respectivamente, mediante la relación:
    \begin{equation}\label{2.3_jackson}
        \left( \vec{E}_2-\vec{E}_1\right) \cdot \hat{n}=\frac{\sigma}{\epsilon_0},
    \end{equation}
donde $\hat{n}$ es el vector unitario normal a la superficie que apunta desde la región 1 hacia la región 2. 
En este caso, la región 1 es el interior de la cavidad esférica y la región 2 es el material conductor. 
Por lo tanto, $\hat{n}$ apunta desde la cavidad hacia el conductor, lo que en coordenadas esféricas con origen en el centro de la cavidad corresponde a 
$\hat{n} = \hat{r}$, el vector unitario radial hacia afuera.


Como el conductor es ideal, el campo eléctrico se anula en su interior; en forma rigurosa adoptamos los valores límite laterales:
\begin{equation*}
    \vec{E}_2(R^+,\theta,\varphi)=\lim_{r\to R^+}\vec E(r,\theta,\varphi)=\vec 0,
    \qquad
    \vec{E}_1(R^-,\theta,\varphi)=\lim_{r\to R^-}\vec E(r,\theta,\varphi).
\end{equation*}
Con $\hat n=\hat r$, la ecuación \eqref{2.3_jackson} queda:
\begin{equation*}
-\vec{E}_1(R^-,\theta,\varphi)\cdot\hat r
    =\frac{\sigma}{\epsilon_0}.
\end{equation*}
Despejando la densidad de carga superficial $\sigma$:
\begin{equation}\label{2.3_sigma}
    \sigma=-\,\epsilon_0\,\vec{E}_1(R^-,\theta,\varphi)\cdot\hat r.
\end{equation}

El campo interior en la superficie (tomado por el lado de la cavidad) se obtiene a partir de \eqref{2.2_campo_electrico} evaluada en $r=R$:
\begin{equation}\label{2.3_E_1}
    \vec{E}_1=\frac{q}{4\pi\epsilon_0}\left(
            \frac{R\,\hat{r}-d\,\hat{z}}{\|R\,\hat{r}-d\,\hat{z}\|^3}
            - \frac{R}{d}\,\frac{R\,\hat{r}-\tfrac{R^2}{d}\,\hat{z}}{\left\|R\,\hat{r}-\tfrac{R^2}{d}\,\hat{z}\right\|^3}\right).
\end{equation}
Sustituyendo \eqref{2.3_E_1} en \eqref{2.3_sigma} se obtiene
\begin{equation*}
    \sigma(\theta)=-\epsilon_0\,\left[\frac{q}{4\pi\varepsilon_0}\left(
            \frac{R\,\hat{r}-d\,\hat{z}}{\|R\,\hat{r}-d\,\hat{z}\|^3}
            - \frac{R}{d}\,\frac{R\,\hat{r}-\tfrac{R^2}{d}\,\hat{z}}{\left\|R\,\hat{r}-\tfrac{R^2}{d}\,\hat{z}\right\|^3}\right)\right]\cdot\hat{r} ,
\end{equation*}
a partir de la cual procedemos con el desarrollo explícito.

    \begin{align*}
        \sigma
        &=-\epsilon_0\,\left[\frac{q}{4\pi\epsilon_0}\left(
            \frac{R\,\hat{r}-d\,\hat{z}}{\|R\,\hat{r}-d\,\hat{z}\|^3}
            - \frac{R}{d}\,\frac{R\,\hat{r}-\tfrac{R^2}{d}\,\hat{z}}{\left\|R\,\hat{r}-\tfrac{R^2}{d}\,\hat{z}\right\|^3}\right)\right]\cdot\hat{r} \\[6pt]
        &=-\frac{q}{4\pi}\,\left(
            \frac{R\,\hat{r}-d\,\hat{z}}{\|R\,\hat{r}-d\,\hat{z}\|^3}\cdot\hat{r}   
            - \frac{R}{d}\,\frac{R\,\hat{r}-\tfrac{R^2}{d}\,\hat{z}}{\left\|R\,\hat{r}-\tfrac{R^2}{d}\,\hat{z}\right\|^3}\cdot\hat{r}   \right) \\[6pt]
        &=-\frac{q}{4\pi}\,\left(
            \frac{R\,\hat{r}\cdot\hat{r}-d\,\hat{z}\cdot\hat{r}}{\|R\,\hat{r}-d\,\hat{z}\|^3}
            - \frac{R}{d}\,\frac{R\,\hat{r}\cdot\hat{r}-\tfrac{R^2}{d}\,\hat{z}\cdot\hat{r}}{\left\|R\,\hat{r}-\tfrac{R^2}{d}\,\hat{z}\right\|^3}   \right) \\[6pt]
        &=-\frac{q}{4\pi}\,\left(
            \frac{R-d\cos\theta}{\|R\,\hat{r}-d\,\hat{z}\|^3}   
            - \frac{R}{d}\,\frac{R-\tfrac{R^2}{d}\cos\theta}{\left\|R\,\hat{r}-\tfrac{R^2}{d}\,\hat{z}\right\|^3}  \right).
    \end{align*}
Donde el ángulo entre $\hat r$ y $\hat z$ coincide con la colatitud $\theta$ del vector radial unitario.

Sustituyendo \eqref{2.1_magnitud_d} y \eqref{2.1_magnitud_d'}:
    \begin{align*}
        \sigma
        &=-\frac{q}{4\pi}\,\left(
            \frac{R-d\cos\theta}{\left(R^2+d^2-2Rd\cos\theta\right)^\frac{3}{2}}   
            - \frac{R}{d}\,\frac{R-\tfrac{R^2}{d}\cos\theta}{\left(R^2+\left(\tfrac{R^2}{d}\right)^2-2R\tfrac{R^2}{d}\cos\theta\right)^\frac{3}{2}}  \right) \\[6pt]
        &=-\frac{q}{4\pi}\,\left(
            \frac{R-d\cos\theta}{\left(R^2+d^2-2Rd\cos\theta\right)^\frac{3}{2}}   
            - \frac{R}{d}\,\frac{R-\tfrac{R^2}{d}\cos\theta}{\left(\tfrac{R^2d^2}{d^2}+\tfrac{R^4}{d^2}-2R\tfrac{R^2d}{d^2}\cos\theta\right)^\frac{3}{2}}  \right) \\[6pt]
        &=-\frac{q}{4\pi}\,\left(
            \frac{R-d\cos\theta}{\left(R^2+d^2-2Rd\cos\theta\right)^\frac{3}{2}}   
            - \frac{R}{d}\,\frac{R-\tfrac{R^2}{d}\cos\theta}{\left(\tfrac{R^2d^2}{d^2}+\tfrac{R^4}{d^2}-2R\tfrac{R^2d}{d^2}\cos\theta\right)^\frac{3}{2}}  \right) \\[6pt]
        &=-\frac{q}{4\pi}\,\left(
            \frac{R-d\cos\theta}{\left(R^2+d^2-2Rd\cos\theta\right)^\frac{3}{2}}   
            - \frac{R}{d}\,\frac{R-\tfrac{R^2}{d}\cos\theta}{\left[ \tfrac{R^2}{d^2}\left( d^2+R^2-2Rd\cos\theta \right) \right]^\frac{3}{2}}  \right) \\[6pt]
        &=-\frac{q}{4\pi}\,\left(
            \frac{R-d\cos\theta}{\left(R^2+d^2-2Rd\cos\theta\right)^\frac{3}{2}}   
            - \frac{R}{d\frac{R^3}{d^3}}\,\frac{R-\tfrac{R^2}{d}\cos\theta}{ \left( d^2+R^2-2Rd\cos\theta\right)^\frac{3}{2}}  \right) \\[6pt]
        &=-\frac{q}{4\pi}\,\left(
            \frac{R-d\cos\theta}{\left(R^2+d^2-2Rd\cos\theta\right)^\frac{3}{2}}   
            - \frac{ \frac{d^2}{R^2} \left( R-\tfrac{R^2}{d}\cos\theta \right)}{ \left( d^2+R^2-2Rd\cos\theta\right)^\frac{3}{2}}  \right) \\[6pt]
        &=-\frac{q}{4\pi}\frac{1}{\left(R^2+d^2-2Rd\cos\theta\right)^\frac{3}{2}} \left(R-d\cos\theta -\frac{d^2}{R}+d\cos\theta \right) \\[6pt]
        &=-\frac{q}{4\pi}\frac{1}{\left(R^2+d^2-2Rd\cos\theta\right)^\frac{3}{2}}  \left(\frac{R^2-d^2}{R}\right)\\[6pt]
        &=-\frac{q}{4\pi}\frac{R^2-d^2}{R\left(R^2+d^2-2Rd\cos\theta\right)^\frac{3}{2}}.
    \end{align*}
Sustituyendo \eqref{2.1_magnitud_d}, se obtiene:
\begin{equation*}
    \sigma=-\frac{q}{4\pi}\frac{R^2-d^2}{\|R\,\hat{r}-d\,\hat{z} \|^3}.
\end{equation*}
Esta es la distribución de carga inducida sobre la superficie de la cavidad. Como se esperaba, la carga inducida es de signo opuesto a $q$ cuando $R^{2} > d^{2}$, lo cual es coherente con la interpretación física del fenómeno de carga inducida.
    
\clearpage
%%%%%%%%%%%%%%%%%%%%%%%%%%%%%%%%%%%%%%%%%%%%%%%%%%%%%%%%%%%%%%%%%%%%%%%%%%%%%%%%%%%%%%%%%%%%%%%%%%%%%%%%%%%%%%%%%%%%%%%%%%%%%%%%%%%%%%%%%%%%%%%%%%%%%%%%%%%%%%%%%%%%%%%%%%%%%%%%%%%%%%%%%%%%%%%%%%%%%%%

\subsection{Gráfico de campo y equipotenciales}

Grafique el campo eléctrico en el interior de la cavidad y algunas superficies equipotenciales.

\textbf{Solución:}

%%%%%%%%%%%%%%%%%%%%%%%%%%%%%%%%%%%%%%%%%%%%%%%%%%%%%%%%%%%%%%%%%%%%%%%%%%%%%%%%%%%%%%%%%%%%%%%%%%%%%%%%%%%%%%%%%%%%%%%%%%%%%%%%%%%%%%%%%%%%%%%%%%%%%%%%%%%%%%%%%%%%%%%%%%%%%%%%%%%%%%%%%%%%%%%%%%%%%%%

Para generar la figura, se fijan primero los parámetros físicos: el radio $R$ de la cavidad esférica, la distancia $d$ entre el centro y la carga real $q$ (ubicada sobre el eje $z$), y la magnitud de dicha carga. Adicionalmente, se incorporan la carga imagen $q'$ y su posición $d'$, con las dependencias demostradas en \eqref{2.1_d'} y \eqref{2.1_carga_imagen}.

\begin{center}
\begin{minted}{python}

# --- Parámetros físicos de la cavidad y cargas (modificables) ---
R = 1.0       # radio de la cavidad (m)
d = 0.5       # posición de la carga real sobre el eje z (m), 0<d<R
q = 1e-9      # carga real (C)

# --- Carga imagen y posición (conductor puesto a tierra) ---
q_prime = -(R/d) * q
d_prime = (R**2)/d
\end{minted}
\end{center}

A continuación se define la rutina \texttt{campo\_E}, que evalúa el campo eléctrico total $\vec{E}=(E_x,E_y,E_z)$ por superposición de la contribución de la carga real situada en $(0,0,d)$ y de su imagen en $(0,0,d')$, de acuerdo con \eqref{2.2_campo_electrico}. Se introduce una regularización numérica mínima para evitar divisiones por cero en $r\to 0$.

\begin{center}
\begin{minted}{python}
def campo_E(x, y, z):
    """
    Campo eléctrico total $(E_x,E_y,E_z)$ por superposición de la carga real $q$ en $(0,0,d)$
    y su imagen $q'$ en $(0,0,d')$ dentro de una cavidad esférica conductora a potencial cero.

    Args:
        x, y, z (array_like): Coordenadas de evaluación (m).

    Returns:
        tuple[np.ndarray, np.ndarray, np.ndarray]: $(E_x, E_y, E_z)$ en V/m.

    Notas:
        - Se añade $10^{-18}$ en las normas para evitar división por cero numérica ($r\\to 0$).
    """
    # --- Vector hacia la carga real (r - r1) y norma ---
    Rx1, Ry1, Rz1 = x, y, z - d
    r1 = np.sqrt(Rx1**2 + Ry1**2 + Rz1**2) + 1e-18

    # --- Vector hacia la imagen y norma  ---
    Rx2, Ry2, Rz2 = x, y, z - d_prime
    r2 = np.sqrt(Rx2**2 + Ry2**2 + Rz2**2) + 1e-18

    # --- Superposición de contribuciones de Coulomb ---
    Ex = constante_de_coulomb * ( q*Rx1/(r1**3) + q_prime*Rx2/(r2**3) )
    Ey = constante_de_coulomb * ( q*Ry1/(r1**3) + q_prime*Ry2/(r2**3) )
    Ez = constante_de_coulomb * ( q*Rz1/(r1**3) + q_prime*Rz2/(r2**3) )
    return Ex, Ey, Ez
\end{minted}
\end{center}

La función \texttt{campo\_electrico\_y\_equipotenciales\_2d} produce la visualización en el plano $xz$ para $y=0$, empleando un \emph{quiver} con flechas normalizadas (dirección de $\vec{E}$) y color asociado a $|\vec{E}|$, junto con curvas de nivel del potencial $\Phi$. Para evitar valores extremos asociados a singularidades, los niveles de contorno se fijan por percentiles.

\begin{center}
\begin{minted}{python}
def potencial_phi(x, y, z):
    """
    Potencial eléctrico $\\Phi$ dentro de la cavidad (referido a cero en el conductor)
    por la superposición de $q$ en $(0,0,d)$ y $q'$ en $(0,0,d')$.

    Args:
        x, y, z (array_like): Coordenadas de evaluación (m).

    Returns:
        np.ndarray: Potencial $\\Phi$ (V).
    """
    # --- Distancias a la carga real y a la imagen (regularizadas) ---
    r1 = np.sqrt(x**2 + y**2 + (z - d)**2) + 1e-18
    r2 = np.sqrt(x**2 + y**2 + (z - d_prime)**2) + 1e-18
    
    # --- Superposición escalar del potencial ---
    return constante_de_coulomb*(q/r1 + q_prime/r2)
\end{minted}
\end{center}

La función \texttt{campo\_electrico\_y\_equipotenciales\_2d} produce la visualización en el plano $xz$ para $y=0$, empleando un \emph{quiver} con flechas normalizadas (dirección de $\vec{E}$) y color asociado a $|\vec{E}|$, junto con curvas de nivel del potencial $\Phi$. Para evitar valores extremos asociados a singularidades, los niveles de contorno se fijan por percentiles.

\begin{center}
\begin{minted}{python}
def campo_electrico_y_equipotenciales_2d(
    xlim=None, zlim=None, n_points=60,
    niveles_phi=12, escala_quiver=80
):
    """
    Visualización 2D en el plano xz (y=0) del campo eléctrico (quiver) y
    superficies equipotenciales de una cavidad esférica conductora con una carga $q$.

    Args:
        xlim, zlim (tuple[float,float] | None): Límites en x y z (m). Por defecto $\\pm1.05R$.
        n_points (int): Resolución de la malla (puntos por eje).
        niveles_phi (int): Número de curvas equipotenciales a mostrar.
        escala_quiver (float): Escala de quiver (mayor $\\Rightarrow$ flechas más cortas).

    Returns:
        - Guarda 'campo_electrico_y_equipotenciales_2d-normalizado-colores.svg' en el directorio actual.
        - Para visualizar en pantalla, descomentar 'plt.show()'.
    """
    
    # --- Límites por defecto (ligeramente mayores que el radio R) ---
    if xlim is None: xlim = (-1.05*R, 1.05*R)
    if zlim is None: zlim = (-1.05*R, 1.05*R)

    # --- Malla 2D en el plano xz (y=0) ---
    x = np.linspace(*xlim, n_points) 
    z = np.linspace(*zlim, n_points) 
    X, Z = np.meshgrid(x, z, indexing="xy") 
    Y = np.zeros_like(X)  

    # --- Máscara de cavidad ($r<R$) para ocultar exterior ---
    r = np.sqrt(X**2 + Z**2)
    dentro = (r < R)  

    # --- Campo y magnitud (solo se dibuja dentro de la cavidad) ---
    Ex, Ey, Ez = campo_E(X, Y, Z)  
    Emag = np.sqrt(Ex**2 + Ez**2) + 1e-18
    U = np.where(dentro, Ex/Emag, np.nan)  # Componente x normalizada (solo dentro de la cavidad)
    W = np.where(dentro, Ez/Emag, np.nan)  # Componente z normalizada (solo dentro de la cavidad)

    # --- Potencial y recorte a la cavidad ---
    Phi = potencial_phi(X, Y, Z)  # Potencial en todos los puntos
    Phi_plot = np.where(dentro, Phi, np.nan)  # Potencial solo dentro de la cavidad

    # --- Figura ---
    plt.figure(figsize=(7, 7))

    # --- Niveles de equipotencial (percentiles para evitar extremos numéricos) ---
    try:
        phi_valid = Phi_plot[np.isfinite(Phi_plot)]
        low = np.nanpercentile(phi_valid, 5)   
        high = np.nanpercentile(phi_valid, 85) 
        levels = np.linspace(low, high, niveles_phi) 
    except ValueError:
        levels = niveles_phi
        
    # --- Contornos equipotenciales ($\\Phi = \\text{cte}$) ---
    cs = plt.contour(X, Z, Phi_plot, levels=levels, linewidths=0.7, colors='k', alpha=0.7)
    plt.clabel(cs, inline=True, fontsize=7, fmt="%.0e")  #

    # --- Quiver del campo (dirección normalizada, color por $|\\vec{E}|$) ---
    plt.quiver(
        X, Z, U, W, Emag,  # Posiciones, componentes normalizadas y magnitud para color
        angles="xy", scale_units="xy", scale=escala_quiver, pivot="middle",
        cmap="viridis"
    )
    plt.colorbar(label="Magnitud del campo eléctrico ($V/m$)", shrink=0.85)
    
    # --- Geometría: borde de la cavidad + cargas (real e imagen) ---
    ang = np.linspace(0, 2*np.pi, 512)
    plt.plot(R*np.cos(ang), R*np.sin(ang), 'k-', lw=1.5, label="Superficie de la cavidad (R)")
    plt.plot(0, d, 'o', ms=6, color='r', label="Carga real (q)")
    plt.plot(0, d_prime, 'o', ms=5, color='b', alpha=0.7, label="Carga imagen (q')")

    # --- Estética de ejes y salida ---
    plt.gca().set_aspect("equal", adjustable="box")  
    plt.xlim(*xlim)
    plt.ylim(1*zlim[0], 2.5*zlim[1]) 
    plt.xlabel("Coordenada x (m)")  
    plt.ylabel("Coordenada z (m)")  
    plt.title("Campo eléctrico y superficies equipotenciales\nen una cavidad esférica conductora")
    plt.legend(loc="upper right", framealpha=0.9)  # Leyenda explicativa
        
    # --- Exportar a archivo vectorial (SVG) ---
    plt.tight_layout()  
    plt.savefig('campo_electrico_y_equipotenciales_2d-normalizado-colores.svg', format='svg')
    #plt.show()  
    plt.close()
    return 'campo_electrico_y_equipotenciales_2d-normalizado-colores.svg'
\end{minted}
\end{center}

En la llamada de ejemplo, el parámetro \texttt{n\_points} controla la resolución de la malla en el plano $xz$; \texttt{niveles\_phi} especifica la cantidad de curvas equipotenciales; y \texttt{escala\_quiver} ajusta la longitud aparente de las flechas (sin afectar su dirección ni la codificación por color). Opcionalmente, \texttt{xlim} y \texttt{zlim} permiten acotar la ventana de visualización.

\begin{center}
\begin{minted}{python}
quiver_2d_con_color_y_equipotenciales(n_points=60, niveles_phi=6, escala_quiver=25)
\end{minted}
\end{center}

La figura resultante se muestra en la Fig.~\ref{fig:cavidad_quiver_equipot}. Las flechas normalizadas indican la dirección de $\vec{E}$ y la escala de color representa su magnitud $|\vec{E}|$. Se destaca el contorno $r=R$ de la cavidad, la posición de la carga real y la carga imagen.

\begin{figure}
    \centering
    \includegraphics[width=0.8\linewidth]{campo_electrico_superficies_equipotenciales.png}
    \caption{Campo eléctrico y superficies equipotenciales en el plano $xz$ ($y=0$) de una cavidad esférica conductora conectada a tierra. Las \emph{flechas normalizadas} muestran la dirección de $\vec{E}$ y el \emph{color} codifica la magnitud $|\vec{E}|$. Se indica el borde de la cavidad ($r=R$), la posición de la carga real $q$ (rojo) y  la posición de la carga imaginaria $q'$ (azul). Los parámetros por defecto utilizados son $R=1.0\,\mathrm{m}$, $d=0.5\,\mathrm{m}$ y $q=10^{-9}\,\mathrm{C}$.}
    \label{fig:cavidad_quiver_equipot}
\end{figure}

\clearpage
\appendix
%%%%%%%%%%%%%%%%%%%%%%%%%%%%%%%%%%%%%%%%%%%%%%%%%%%%%%%%%%%%%%%%%%%%%%%%%%%%%%%%%%%%%%%%%%%%%%%%%%%%%%%%%%%%%%%%%%%%%%%%%%%%%%%%%%%%%%%%%%%%%%%%%%%%%%%%%%%%%%%%%%%%%%%%%%%%%%%%%%%%%%%%%%%%%%%%%%%%%%%

%%%%%%%%%%%%%%%%%%%%%%%%%%%%%%%%%%%%%%%%%%%%%%%%%%%%
\section{Demostraciones de antiderivadas}\label{app:antiderivadas}
%%%%%%%%%%%%%%%%%%%%%%%%%%%%%%%%%%%%%%%%%%%%%%%%%%%%

En esta sección se presentan las demostraciones detalladas de las antiderivadas empleadas en el cálculo
del campo eléctrico de la placa rectangular.

\subsection*{Demostración 1}

Consideremos el cambio de variable secundario $u=\rho\tan\theta \quad \Rightarrow \quad du=\rho\sec^2\theta\,d\theta$. Entonces,

\begin{align*}
    -\int \frac{u}{\big(u^2+\rho^2\big)^{3/2}}\,du
    &= -\int\frac{(\rho\tan\theta)(\rho\sec^2\theta)}{\big(\rho^2+\rho^2\tan^2\theta\big)^{3/2}}\,d\theta \\
    &= -\int\frac{\rho^2\tan\theta\sec^2\theta}{\big[\rho^2\big(1+\tan^2\theta\big)\big]^{3/2}}\,d\theta\\
    &= -\int\frac{\rho^2\tan\theta\sec^2\theta}{\rho^3\big(\sec^2\theta\big)^{3/2}}\,d\theta\\
    &= -\frac{1}{\rho}\int\frac{\tan\theta\sec^2\theta}{\sec^3\theta}\,d\theta\\
    &= -\frac{1}{\rho}\int \frac{\tan\theta}{\sec\theta}\, d\theta\\
    &= -\frac{1}{\rho}\int\sin\theta\, d\theta\\
    &= -\frac{1}{\rho}\big[-\cos\theta \big].
\end{align*}

Por consiguiente,
\begin{equation*}
    -\int \frac{u}{\big(u^2+\rho^2\big)^{3/2}}\,du = -\frac{1}{\rho}\big[-\cos\theta \big].
\end{equation*}

A partir de la sustitución $u=\rho\tan\theta$ y las identidades trigonométricas:

\begin{align*}
    1+\tan^2\theta=\sec^2\theta \quad &\Rightarrow \quad \cos^2\theta=\frac{1}{1+\tan^2\theta}, \\[4pt]
    &\Rightarrow \quad \cos\theta = \sqrt{\frac{1}{1+\tan^2\theta}}.
\end{align*}

De este modo,
\begin{align*}
    \cos\theta &= \sqrt{\frac{1}{1+\left(\frac{u}{\rho}\right)^2}} \\[4pt]
    &= \sqrt{\frac{\rho^2}{\rho^2+u^2}}\\[4pt]
    &= \frac{\rho}{\sqrt{\rho^2+u^2}}.
\end{align*}

Finalmente, se obtiene:
\begin{equation}\label{eq:app:int-u-over}
    -\int \frac{u}{\big(u^2+\rho^2\big)^{3/2}}\,du
    = \frac{1}{\sqrt{\rho^2+u^2}}+C,
\end{equation}
donde $C$ es una constante de integración real.

\newpage

\subsection*{Demostración 2}

De modo análogo, consideremos el cambio de variable secundario $u=\rho\tan\theta$ (con $du=\rho\sec^2\theta\,d\theta$). Entonces,

\begin{align*}
    \int \frac{1}{\big(u^2+\rho^2\big)^{3/2}}\,du
    &= \int\frac{\rho\sec^2\theta}{\big(\rho^2+\rho^2\tan^2\theta\big)^{3/2}}\,d\theta \\[4pt]
    &= \int\frac{\rho\sec^2\theta}{\big[\rho^2\big(1+\tan^2\theta\big)\big]^{3/2}}\,d\theta\\[4pt]
    &= \int\frac{\rho\sec^2\theta}{\rho^3\big(\sec^2\theta\big)^{3/2}}\,d\theta\\[4pt]
    &= \frac{1}{\rho^2}\int\frac{\sec^2\theta}{\sec^3\theta}\,d\theta\\[4pt]
    &= \frac{1}{\rho^2}\int \frac{1}{\sec\theta}\, d\theta\\[4pt]
    &= \frac{1}{\rho^2}\int\cos\theta\, d\theta\\[4pt]
    &= \frac{1}{\rho^2}\big[\sin\theta \big].
\end{align*}

En consecuencia,
\begin{equation*}
    \int \frac{1}{\big(u^2+\rho^2\big)^{3/2}}\,du = \frac{1}{\rho^2}\big[\sin\theta \big].
\end{equation*}

De la sustitución $u=\rho\tan\theta$ se obtiene, mediante identidades trigonométricas:

\begin{align*}
    1+\tan^2\theta=\sec^2\theta \quad &\Rightarrow \quad 1-\sin^2\theta =\frac{1}{1+\tan^2\theta}, \\[4pt]
    &\Rightarrow \quad \sin\theta = \sqrt{1- \frac{1}{1+\tan^2\theta}}.
\end{align*}

Por tanto,
\begin{align*}
    \sin\theta &= \sqrt{1- \frac{1}{1+\left(\frac{u}{\rho}\right)^2}}\\[4pt]
    &= \sqrt{1- \frac{\rho^2}{\rho^2+u^2}}\\[4pt]
    &= \sqrt{\frac{u^2}{\rho^2+u^2}}\\[4pt]
    &= \frac{u}{\sqrt{\rho^2+u^2}}.
\end{align*}

Finalmente,
\begin{equation}\label{eq:app:int-one-over}
    \int \frac{1}{\big(u^2+\rho^2\big)^{3/2}}\,du
    = \frac{1}{\rho^2}\,\frac{u}{\sqrt{\rho^2+u^2}}+C,
\end{equation}
donde $C$ es una constante de integración real.

\newpage

\subsection*{Demostración 3}

Consideremos la integral
    \begin{equation*}
        \int \frac{dt}{\sqrt{t^2+\alpha^2}},
    \end{equation*}
con $\alpha>0$. Empleamos la sustitución hiperbólica $t=\alpha\sinh\theta \quad \Rightarrow \quad  dt=\alpha\cosh\theta\,d\theta$. Donde
    \begin{align*}
        \sqrt{t^2+\alpha^2}&=\sqrt{(\alpha\sinh\theta)^2+\alpha^2}\\[4pt]
        &=\sqrt{\alpha^2\sinh^2\theta+\alpha^2}\\[4pt]
        &=\sqrt{\alpha^2(\sinh^2\theta+1)}\\[4pt]
        &=\sqrt{\alpha^2\cosh^2\theta}\\[4pt]
        &=\alpha\cosh\theta.
    \end{align*}
Por lo tanto:
    \begin{equation*}
        \sqrt{t^2+\alpha^2}=\alpha\cosh\theta.
    \end{equation*}
Entonces,
    \begin{align*}
        \int \frac{dt}{\sqrt{t^2+\alpha^2}}
        &= \int \frac{\alpha\cosh\theta\,d\theta}{\alpha\cosh\theta}\\[4pt]
        &= \int d\theta\\[4pt]
        &= \theta + C.
    \end{align*}
Invirtiendo la sustitución, de $t=\alpha\sinh\theta$ se obtiene
    \begin{equation*}
        \theta=\operatorname{arsinh}\!\left(\frac{t}{\alpha}\right).
    \end{equation*}
Por consiguiente,
    \begin{equation}\label{eq:app:int-asinh}
        \int \frac{dt}{\sqrt{t^2+\alpha^2}}
        = \operatorname{arsinh}\!\left(\frac{t}{\alpha}\right)+C.
    \end{equation}
donde $C$ es una constante de integración real.

\textit{Observación.} Usando $\operatorname{arsinh} x=\ln\!\big(x+\sqrt{x^2+1}\big)$, también puede escribirse
    \begin{equation*}
        \int \frac{dt}{\sqrt{t^2+\alpha^2}}
        = \ln\!\left(\frac{t}{\alpha}+\sqrt{\frac{t^2}{\alpha^2}+1}\right)+C
        = \ln\!\big(t+\sqrt{t^2+\alpha^2}\big)-\ln\alpha + C,
    \end{equation*}
lo cual es equivalente tras absorber $-\ln\alpha$ en la constante de integración.

\newpage

\subsection*{Demostración 4}

Consideremos la integral

\begin{equation*}
    \int \frac{A}{(s^2+z^2)\sqrt{A^2+s^2+z^2}}\,ds,
    \qquad z\neq0.
\end{equation*}

Procedemos por sustituciones trigonométricas, al estilo de las demostraciones previas. Consideramos la sustitución $s=z\tan\theta \quad \Rightarrow \quad ds=z\sec^2\theta\,d\theta$, donde 
    \begin{align*}
        s^2+z^2&=(z\tan\theta)^2+z^2 \\[4pt]
        &=z^2\tan^2\theta+z^2 \\[4pt]
        &=z^2(\tan^2\theta + 1) \\[4pt]
        &=z^2 \sec^2\theta.
    \end{align*}
Por lo tanto:
    \begin{equation*}
        s^2+z^2=z^2 \sec^2\theta.
    \end{equation*}
Luego, considerando el $1$ conveniente $\cos^2\theta\sec^2\theta=1$, se obtiene:
    \begin{align*}
        \sqrt{A^2+s^2+z^2}
        &= \sqrt{A^2+z^2\sec^2\theta}\\[4pt]
        &= \sqrt{A^2\cos^2\theta\sec^2\theta+z^2\sec^2\theta}\\[4pt]
        &= \sqrt{\sec^2\theta\big(A^2\cos^2\theta+z^2\big)}\\[4pt]
        &= \sec\theta\,\sqrt{A^2\cos^2\theta+z^2}.
    \end{align*}
Sustituyendo en la integral,
    \begin{align*}
        \int \frac{A}{(s^2+z^2)\sqrt{A^2+s^2+z^2}}\,ds
        &= \int \frac{A}{z^2\sec^2\theta\;\sec\theta\sqrt{A^2\cos^2\theta+z^2}}\;z\sec^2\theta\,d\theta\\[4pt]
        &= \int \frac{A}{z\,\sec\theta\sqrt{A^2\cos^2\theta+z^2}}\,d\theta\\[4pt]
        &= \frac{A}{z}\int \frac{\cos\theta}{\sqrt{A^2\,(1-\sin^2\theta)+z^2}}\,d\theta.
    \end{align*}
Ahora introducimos $u=\sin\theta \quad \Rightarrow \quad du=\cos\theta\,d\theta$:
    \begin{align*}
        \frac{A}{z}\int \frac{\cos\theta}{\sqrt{A^2\,(1-\sin^2\theta)+z^2}}\,d\theta
        &= \frac{A}{z}\int \frac{1}{\sqrt{A^2(1-u^2)+z^2}}\,du\\[4pt]
        &= \frac{A}{z}\int \frac{1}{\sqrt{(A^2+z^2)-A^2u^2}}\,du.
    \end{align*}
Usamos la primitiva estándar\footnote{Una derivación puede encontrarse en \textit{OpenStax, Calculus Vol.~1}, §5.7 \emph{Integrals Resulting in Inverse Trigonometric Functions}, disponible en \url{https://openstax.org/books/calculus-volume-1/pages/5-7-integrals-resulting-in-inverse-trigonometric-functions}. Véase también la versión en LibreTexts: \url{https://math.libretexts.org/Bookshelves/Calculus/Calculus_(OpenStax)/05:_Integration/5.07:_Integrals_Resulting_in_Inverse_Trigonometric_Functions}.}
\begin{equation*}
    \int \frac{1}{\sqrt{\beta^2-\gamma^2u^2}}\,du
    = \frac{1}{\gamma}\,\arcsin\!\left(\frac{\gamma u}{\beta}\right)+C,
    \qquad (\beta>\gamma>0).
\end{equation*}
\begin{equation*}
    \int \frac{1}{\sqrt{\beta^2-\gamma^2u^2}}\,du
    = \frac{1}{\gamma}\,\arcsin\!\left(\frac{\gamma u}{\beta}\right)+C,
    \qquad (\beta>\gamma>0).
\end{equation*}

    \begin{equation*}
        \int \frac{1}{\sqrt{\beta^2-\gamma^2u^2}}\,du
        = \frac{1}{\gamma}\,\arcsin\!\left(\frac{\gamma u}{\beta}\right)+C
        \qquad (\beta>\gamma>0),
    \end{equation*}
con $\beta=\sqrt{A^2+z^2}$ y $\gamma=|A|$. Suponiendo $A\neq 0$ (el caso $A=0$ es trivial), resulta
    \begin{align*}
        \frac{A}{z}\int \frac{1}{\sqrt{(A^2+z^2)-A^2u^2}}\,du
        &= \frac{A}{z}\cdot\frac{1}{A}\,\arcsin\!\left(\frac{A\,u}{\sqrt{A^2+z^2}}\right)+C\\[4pt]
        &= \frac{1}{z}\,\arcsin\!\left(\frac{A\,\sin\theta}{\sqrt{A^2+z^2}}\right)+C.
    \end{align*}
Finalmente, expresamos $\sin\theta$ en términos de $s$. Como $s=z\tan\theta$,
    \begin{align*}
        \sec^2\theta&=1+\tan^2\theta \\[4pt]
        &=1+\frac{s^2}{z^2} \\[4pt]
        &=\frac{s^2+z^2}{z^2}
    \end{align*}
Entonces, 
    \begin{align*}
        \sin\theta&=\frac{\tan\theta}{\sec\theta} \\[4pt]
        &= \frac{ \dfrac{s}{z}}{\dfrac{\sqrt{(s^2+z^2)}}{z}} \\[4pt]
        &= \frac{s}{\sqrt{s^2+z^2}}.
    \end{align*}
Con esto,
    \begin{equation*}
        \int \frac{A}{(s^2+z^2)\sqrt{A^2+s^2+z^2}}\,ds
        = \frac{1}{z}\,\arcsin\!\left(\frac{A\,s}{\sqrt{A^2+z^2}\,\sqrt{s^2+z^2}}\right)+C.
    \end{equation*}
Para presentar el resultado en la forma dada originalmente, usamos
    \begin{equation*}
    \arcsin y=\arctan\!\left(\frac{y}{\sqrt{1-y^2}}\right),
    \end{equation*}
con
    \begin{equation*}
    y=\frac{A\,s}{\sqrt{A^2+z^2}\,\sqrt{s^2+z^2}}
    \quad\Rightarrow\quad
    \sqrt{1-y^2}
    =\frac{z\,\sqrt{A^2+s^2+z^2}}{\sqrt{A^2+z^2}\,\sqrt{s^2+z^2}}.
    \end{equation*}
Por tanto,
    \begin{align*}
        \frac{1}{z}\,\arcsin(y)
        &= \frac{1}{z}\,\arctan\!\left(
            \frac{y}{\sqrt{1-y^2}}
        \right)+C\\[4pt]
        &= \frac{1}{z}\,\arctan\!\left(
            \frac{A\,s}{z\,\sqrt{A^2+s^2+z^2}}
        \right)+C,
    \end{align*}
Finalmente, se obtiene:
    \begin{equation}\label{eq:app:int-A-over}
        \int \frac{A}{(s^2+z^2)\sqrt{A^2+s^2+z^2}}\,ds
        = \frac{1}{z}\,\arctan\!\left(\frac{A\,s}{z\,\sqrt{A^2+s^2+z^2}}\right)+C,
        \qquad z\neq 0,
    \end{equation}
donde $C$ es una constante de integración real.

\end{document}
