\documentclass{article}

% ==================================================
% Utilidades generales
% ==================================================
\usepackage{fvextra}   % Mejora entornos tipo verbatim (base de minted)
\usepackage{csquotes}  % Comillas tipográficas y soporte para citas

% ==================================================
% Paquetes matemáticos y tipográficos
% ==================================================
\usepackage{cancel}    % Permite tachar términos en ecuaciones
\usepackage{mathrsfs}  % Fuentes caligráficas (\mathscr)
\usepackage{amssymb}   % Símbolos matemáticos extra
\usepackage{amsmath}   % Entornos y mejoras de matemáticas
\usepackage{amsfonts}  % Fuentes matemáticas adicionales
\usepackage{mathtools} % Extensión de amsmath (ajustes finos)
\usepackage{bigints}   % Integrales grandes y personalizables
\usepackage{appendix}  % Soporte para entorno appendices

% Límite directo con notación personalizada
\newcommand{\dirlim}[3]{\mathop{#1}\limits_{#2}^{\nearrow\, #3}}

\usepackage{subcaption} % Subfiguras dentro de una misma figura

% ==================================================
% Gráficos y TikZ
% ==================================================
\usepackage{tikz}      % Dibujos vectoriales en LaTeX
\usetikzlibrary{
    decorations.pathmorphing, % Líneas decoradas (muelles, etc.)
    arrows.meta,              % Flechas modernas y configurables
    calc,                     % Cálculo de coordenadas (($(A)!t!(B)$, etc.)
    babel                     % Adaptación de TikZ a babel (spanish, shorthands)
}

% ==================================================
% Código fuente (minted)
% ==================================================
\usepackage{minted}  % Resaltado de sintaxis con Pygments (requiere -shell-escape)
\usepackage{xcolor}  % Colores (usado por minted y para enlaces)

% Configuración estética de minted
\definecolor{codebg}{rgb}{0.1, 0.1, 0.1} % Fondo oscuro para bloques de código
\setminted{
    style=monokai,   % Estilo de colores tipo Monokai
    bgcolor=codebg,  % Fondo definido arriba
    linenos,         % Números de línea
    breaklines,      % Partir líneas largas automáticamente
    fontsize=\small  % Tamaño de letra del código
}

% ==================================================
% Entornos tipo terminal / listas de código
% ==================================================
\usepackage{tcolorbox}           % Cajas coloreadas (terminal, notas, etc.)
\tcbuselibrary{listingsutf8}    % Soporte UTF-8 para listings integrados en tcolorbox
\usepackage{listings}           % Entorno alternativo para código (si lo llegas a usar)
\usepackage{bera}               % Fuente monoespaciada estética (para código)

% ==================================================
% Referencias internas por nombre
% ==================================================
\usepackage{nameref}   % Permite referirse a secciones por nombre (\nameref)

% Numeración de ecuaciones por sección
\numberwithin{equation}{section}

% ==================================================
% Colores y enlaces
% ==================================================
\definecolor{linkblue}{RGB}{50, 30, 200} % Azul sobrio para enlaces

\usepackage[colorlinks=true,
            linkcolor=linkblue,  % Color para enlaces internos (secciones, figuras)
            urlcolor=linkblue,   % Color para URLs
            citecolor=linkblue,  % Color para citas bibliográficas
            filecolor=linkblue]{hyperref} % Enlaces a archivos

% ==================================================
% Idioma
% ==================================================
\usepackage[spanish,shorthands=off]{babel}
% spanish: reglas tipográficas y traducción de elementos (Contenido, Figura, etc.)
% shorthands=off: evita conflictos de babel con símbolos como "<" y ">" usados en TikZ

% ==================================================
% Imágenes externas
% ==================================================
\usepackage{graphicx} 
\graphicspath{{images/}} % Carpeta base para las figuras del documento

% ==================================================
% Configuración de página y márgenes
% ==================================================
\usepackage[a4paper,
            left=1.5cm,
            right=1.5cm,
            top=20mm,
            bottom=20mm]{geometry} % Márgenes ajustados para texto denso

% Tipografía mejorada (Latin Modern)
\usepackage{lmodern}

% ==================================================
% Estilo de títulos de secciones
% ==================================================
\usepackage{titlesec}
% Añade punto tras el número de sección: "1. Título" en lugar de "1 Título"
\titleformat{\section}{\huge\bfseries}{\thesection.}{1em}{}  

% ==================================================
% Encabezados y pies de página
% ==================================================
\usepackage{fancyhdr}
\pagestyle{fancy}
\fancyhf{}
\fancyhead[L]{\textit{Tarea 3, 2025-2}}  % Encabezado izquierdo
\fancyhead[R]{Electrodinámica I}        % Encabezado derecho
% (Pie de página vacío; números de página irán en la posición por defecto si se desea)

% ==================================================
% Estilo de párrafos
% ==================================================
\setlength{\parindent}{0pt}  % Sin sangría en el inicio de párrafo
\setlength{\parskip}{5pt}    % Espacio entre párrafos
\sloppy                      % Evita líneas sobresalidas (divide palabras más libremente)

% ==================================================
% Índice (Tabla de contenidos)
% ==================================================
\usepackage{tocloft}
\setcounter{tocdepth}{2} % Profundidad del índice (hasta subsecciones)

% ==================================================
% Notas al pie
% ==================================================
\usepackage[bottom]{footmisc}              % Fija las notas al pie cerca de la parte inferior
\setlength{\skip\footins}{6pt}             % Espacio entre texto y bloque de notas
\setlength{\footnotesep}{0.5\baselineskip} % Separación entre notas individuales
\raggedbottom                               % Evita estiramientos verticales forzados

% Regla horizontal más compacta para separar las notas del texto
\renewcommand\footnoterule{\kern -2pt \hrule width .4\columnwidth \kern 4pt}

% ==================================================
% Figuras y tablas
% ==================================================
\usepackage{float}               % Control explícito de la colocación de flotantes
\floatplacement{figure}{ht}      % Figuras preferentemente "here" o "top"
\floatplacement{table}{ht}       % Tablas preferentemente "here" o "top"


\begin{document}

% Portada
\begin{titlepage}
    \centering
    \vspace*{3cm} % Ajuste en la posición vertical
    % Logo centrado
    \includegraphics[width=0.6\textwidth]{UdeC_azul_centrado.png} 
    
    \vspace{1cm}
    \thispagestyle{empty} % Sin número en la portada

    % Título de la tarea
    {\Huge \textbf{Tarea 03} \par}
    
    \vspace{0.5cm}
    {\Huge \textbf{Electrodinámica I} \par}
    \vspace{1.5cm}

    % Nombre del autor
    {\Large José Ignacio Rosas Sepúlveda \par}
    \vspace{1cm}
    
    % Fechas de la tarea
    {\Large Noviembre 2025 \par}
    \vfill
\end{titlepage}

% Índice
\tableofcontents
\newpage

%%%%%%%%%%%%%%%%%%%%%%%%%%%%%%%%%%%%%%%%%%%%%%%%%%%%%%%%%%%%%%%%%%%%%%%%%%%%%%%%%%%%%%%%%%%%%%%%%%%%%%%%%%%%%%%%%%%%%%%%%%%%%%%%%%%%%%%%%%%%%%%%%%%%%%%%%%%%%%%%%%%%%%%%%%%%%%%%%%%%%%%%%%%%%%%%%%%%%%%

\section{Problema 1}

Una esfera conductora de radio \(R\) flota semisumergida en un líquido de constante dieléctrica \(\kappa_1\).
La región sobre el líquido contiene un gas de constante dieléctrica \(\kappa_2\). Si la carga externa total sobre la
esfera es \(Q\), determine el campo eléctrico que ella produce y las densidades de carga de polarización
en cada punto de su superficie.

\textbf{Solución:}

%%%%%%%%%%%%%%%%%%%%%%%%%%%%%%%%%%%%%%%%%%%%%%%%%%%%%%%%%%%%%%%%%%%%%%%%%%%%%%%%%%%%%%%%%%%%%%%%%%%%%%%%%%%%%%%%%%%%%%%%%%%%%%%%%%%%%%%%%%%%%%%%%%%%%%%%%%%%%%%%%%%%%%%%%%%%%%%%%%%%%%%%%%%%%%%%%%%%%%%

%%%% Presentacion del sistema de referencia

Consideremos un sistema de coordenadas esféricas $(r,\theta,\varphi)$ con \textbf{origen en el centro de la esfera}, cuyo observador se ubica en dicho punto. Alineamos los ejes de modo que el plano $xy$ coincida con la interfase entre los dieléctricos. De esta forma, un corte transversal del sistema en el plano $yz$ intersecando en $x=0$ (el origen del sistema de coordenadas y centro de la esfera) es tal como el que se observa en la figura~\ref{fig:esfera-semisumergida}. Se identifica simetría azimutal en el sistema.

\begin{figure}[ht]
\centering
\begin{tikzpicture}[
    scale=1.0,
    line cap=round,
    line join=round,
    >=Stealth,
    interface/.style={very thick},
    sph/.style={thick},
    dashedm/.style={densely dashed},
    field/.style={-Stealth, thick},
    normalv/.style={-Stealth, thick, gray!70},
    chargeplus/.style={font=\large, blue!70!black}
]
  %------------- Parámetros geométricos -------------
  \def\R{2.4}      % radio de la esfera
  \def\yI{0.0}     % altura de la interfaz respecto al centro (y = \yI)
  \coordinate (C) at (0,0);  % centro de la esfera

  %------------- Medios dieléctricos -------------
  % Líquido (abajo)
  \fill[blue!8] (-4,-3.2) rectangle (4,\yI);
  % Gas (arriba)
  \fill[green!5] (-4,\yI) rectangle (4,3.2);

 %------------- Interfaz dieléctrica -------------
  \draw[interface] (-4,\yI) -- (4,\yI);
  \node[anchor=south east, black!70] at (-3.9,\yI) {Gas, $\kappa_2$};
  \node[anchor=north east, black!70] at (-3.9,\yI) {Líquido, $\kappa_1$};

  %------------- Esfera conductora -------------
  \draw[sph, fill=gray!15] (C) circle (\R);
  \fill (C) circle (1.2pt) node[below right, xshift=0mm] {O};

  % Radio R
  \draw[dashedm] (C) -- ++(\R,0)
      node[midway, below] {$R$};

    % Carga en la superficie
    \foreach \ang in {0,5,10,...,360}{
        \node[red, fill=red, circle, inner sep=0.75pt] at ({\R*cos(\ang)},{\R*sin(\ang)}) {};
    }

  % Etiqueta de la esfera y la carga total
  \node[anchor=south west, black!80] at (-1,\R*0.015) {Esfera conductora};
  %\node[anchor=west, black!80] at (\R*0.15,-\R*0.05) {Carga total $Q$};



\end{tikzpicture}
\caption{Esfera conductora de radio $R$ semisumergida en la interfaz entre un líquido (constante dieléctrica $\kappa_1$) y un gas (constante dieléctrica $\kappa_2$). La esfera posee una carga total $Q$ distribuida en su superficie.}
\label{fig:esfera-semisumergida}
\end{figure}

%%%%% DEFINIR LA PERMITIVIDAD EN EL ESPACIO

En función de las coordenadas radial y colatitudinal $(r,\theta)$ se describe la \textbf{permitividad dieléctrica} del sistema, para todo punto en el espacio, distinguiendo el interior del conductor y las regiones ocupadas por el gas y el líquido, de modo:
\begin{equation}\label{1.varepsilon}
    \epsilon(r,\theta)=
    \left\{
        \begin{aligned}
        \epsilon_0 \,, &\qquad 0<r<R \;;\\[6pt]
        \epsilon_0\,\kappa_2\,, &\qquad  r>R \;\wedge \; 0\leq \theta \leq \frac{\pi}{2} \;;\\[6pt]
        \epsilon_0\,\kappa_1\,, &\qquad r>R \;\wedge \;\frac{\pi}{2} < \theta \leq \pi \;.
        \end{aligned}
    \right.
\end{equation}

Idealizamos tanto el líquido como el gas como medios \textbf{dieléctricos lineales e isótropos}, caracterizados por las \textbf{constantes dieléctricas $\kappa_1$ y $\kappa_2$}, respectivamente. En este tipo de medios \textbf{la polarización $\vec{P}$ depende local y linealmente de EL campo eléctrico macroscópico} del sistema, $\vec{E}$, es decir 
\begin{equation}\label{1.POLARIZACION}
    \vec{P}_i(\vec{r})=\epsilon_0\chi_i\,\vec{E}(\vec{r}) \,,
\end{equation}
donde: $\vec{r}$ es un punto de observación sobre el espacio; $\chi_i$ es la \textbf{susceptibilidad eléctrica del medio}, con $i=1,2$ un indice que etiqueta al medio dieléctrico en cuestión. 

Cabe mencionar que la susceptibilidad eléctrica se relaciona con la \textbf{constante dieléctrica} $\kappa_i$ (que es el dato que nos entrega el problema para ambos medios dieléctricos) de la siguiente forma 
\begin{equation}\label{1.susceptibilidad}
    \kappa_i=1+\chi_i \qquad \Longleftrightarrow \qquad \chi_i=\kappa_i-1 \;.
\end{equation}
Es por ello que podemos definir una segunda función por tramos con la cual \textbf{se describe la susceptibilidad dieléctrica del sistema en función de las coordenadas $(r,\theta)$}, es decir 
\begin{equation}\label{1.chi}
    \chi_e(r,\theta)=
    \left\{
        \begin{aligned}
        0 \,, &\qquad   0<r<R \;;\\[6pt]
        \kappa_2-1\,, &\qquad  r>R \;\wedge \; 0\leq \theta \leq \frac{\pi}{2} \;;\\[6pt]
        \kappa_1-1\,, &\qquad r>R \;\wedge \;\frac{\pi}{2} < \theta \leq \pi \;.
        \end{aligned}
    \right.
\end{equation}


%%%%%%%%%%%%%
Por otra parte, el \textbf{desplazamiento eléctrico} del sistema, en las regiones de los medios dieléctricos, esta dado por
\begin{equation*}
    \vec{D}_i(\vec{r})=\epsilon_0\vec{E}(\vec{r})+\vec{P}_i(\vec{r})\,.
\end{equation*}
Sustituyendo \eqref{1.POLARIZACION} y \eqref{1.susceptibilidad} en esta ecuación, se sigue
\begin{align*}
    \vec{D}_i(\vec{r})&=\epsilon_0\vec{E}(\vec{r})+\epsilon_0\chi_i\,\vec{E}(\vec{r})\\[4pt]
    &=\epsilon_0(1+\chi_i)\vec{E}(\vec{r}) \\[4pt]
    &=\epsilon_0\kappa_i\vec{E}(\vec{r})\,.
\end{align*}
En general, se describe el desplazamiento eléctrico del sistema en todo el espacio como el producto entre el campo eléctrico macroscópico, $\vec{E}(\vec{r})$, y la función que hemos definido anteriormente en \eqref{1.varepsilon}, la cual describe la permitividad dieléctrica del sistema, es decir
\begin{equation}\label{1.desplazamiento_electrico}
    \vec{D}(\vec{r})=\epsilon(r,\theta)\vec{E}(\vec{r}) \,.
\end{equation}

Ahora por la ley de Gauss para el desplazamiento eléctrico, es decir, considerando el flujo de $\vec{D}$ que atraviesa una superficie gaussiana esférica de radio $r$, circuncentrica a la esfera conductora, denotada por $S_r$ que encierra solo las cargas libres $Q_{free}$ presentes en el sistema,
\begin{equation*}
    \oint_{ S_r}  \vec{D}(\vec{r})\cdot  d\vec{A}=Q_{\text{free}}\,.
\end{equation*}

Sustituyendo \eqref{1.desplazamiento_electrico} en esta ultima ecuación, se obtiene
\begin{equation}\label{1.gauss_final}
    \oint_{S_r} \epsilon(r,\theta)\vec{E}(\vec{r}) \cdot d\vec{A}=Q_{\text{free}}\,.
\end{equation}
Ahora bien, si consideramos que el radio de $S_r$ es menor al de la esfera conductora, $r<R$, entonces la superficie gaussiana descrita en \eqref{1.gauss_final} no encierra cargas libres, es decir $Q_{\text{free}}=0$ (Véase la Figura~\ref{fig:gauss_interior}). 

Físicamente esto es dado porque el sistema se encuentra en \textbf{estado estacionario}, las cargas libres del conductor están distribuidas en su totalidad sobre la superficie del mismo dada la presencia de un campo externo distinto a EL campo eléctrico macroscópico que se nos pide calcular (nótese que el campo externo contribuye a EL campo eléctrico macroscópico por principio de superposición lineal), estas cargas libres distribuidas sobre la superficie del conductor generan su propio campo eléctrico de modo que se equilibra con el campo externo anulando toda fuerza que ponga en dinámica las cargas al interior del conductor, es así como tanto física como matemáticamente podemos describir que al interior del conductor no hay cargas libres y que incluso El campo eléctrico macroscópico es nulo en dicha región. 

\begin{figure}
\centering
\begin{tikzpicture}[
    scale=2,
    interface/.style={very thick},
    sph/.style={thick},
    dashedm/.style={densely dashed},
    field/.style={-Stealth, thick},
    normalv/.style={-Stealth, thick, gray!70},
    chargeplus/.style={font=\large, blue!70!black}
]

    % Parámetros geométricos
    \def\R{1.4}      % Radio esfera conductora
    \def\r{0.8}      % Radio superficie gaussiana
    \def\yI{0.0}     % Altura de la interfaz (y = \yI)

    % Medios dieléctricos
    \fill[blue!8] (-2.5,-1.5) rectangle (2.5,\yI);
    \fill[green!5] (-2.5,\yI) rectangle (2.5,1.5);

    % Interfaz dieléctrica
    \draw[interface] (-2.5,\yI) -- (2.5,\yI);
    \node[anchor=south east, black!70] at (-2.4,\yI) {Gas, $\kappa_2$};
    \node[anchor=north east, black!70] at (-2.4,\yI) {Líquido, $\kappa_1$};

    % Esfera conductora
     % Esfera conductora
    \draw[sph, fill=gray!15] (0,0) circle (\R);
    \draw[dashedm] (C) -- ++(\R,0)
          node[midway, below] {$R$};
    \node[anchor=south west, black!80] at (-0.7,\R*0.015) {Esfera conductora};
     \fill (C) circle (1.2pt) node[below right, xshift=0mm] {O};

    % Carga en la superficie
    \foreach \ang in {0,5,10,...,360}{
        \node[red, fill=red, circle, inner sep=0.75pt] at ({\R*cos(\ang)},{\R*sin(\ang)}) {};
    }

    % Superficie gaussiana
    \draw[dashed, blue, thick] (0,0) circle (\r);
    \node[blue] at (0,-\r-0.15) {$r<R$};

    % Texto central indicando que no hay carga encerrada
    \node at (0,0.65) {$S_r$};
    \node[red] at (0,-0.3) {$Q_{\text{free}} = 0$};

\end{tikzpicture}
\caption{Superficie gaussiana esférica $S_r$ (con $r<R$) concéntrica con la esfera conductora semisumergida.  
Toda la carga reside sobre la superficie del conductor, por lo que $S_r$ no encierra carga alguna.}
\label{fig:gauss_interior}
\end{figure}

En resumen, para $r<R$ se tiene  
\begin{equation*}
    \epsilon_0\oint_{S_r} \vec{E}(\vec{r}) \cdot d\vec{A}=0\,, 
\end{equation*}
con $\epsilon(r,\theta)$ una función constante al interior de la esfera conductora, la ecuación anterior se satisface en el equilibrio electrostático sabiendo que el campo eléctrico se anula en la región interior al conductor, es decir 
\begin{equation*}
    \vec{E}(\vec{r})=\vec{0} \,,\qquad r<R\,,
\end{equation*}
como ya mencionamos, físicamente esto se interpreta como si al interior de la esfera conductora hubiera campo eléctrico externo, las cargas libres se moverían hasta anularlo para así lograr el estado estacionario.

%%%%%% CALCULO DEL CAMPO ELECTRICO AL EXTERIOR

Para $S_r$ con $r>R$, se tendrán todas las cargas libres del sistema agrupadas sobre la superficie del conductor, es decir $Q_{\text{free}}=Q$ (Veasé la Figura~\ref{fig:gauss_fuera}). Entonces, el campo eléctrico al exterior de la esfera lo podemos calcular con \eqref{1.gauss_final} de modo que la ecuación quedara dada por 
\begin{equation}\label{1.gauss_desplazamiento_electrico}
    \oint_{S_r} \epsilon(r,\theta)\vec{E}(\vec{r})\cdot d\vec{A}=\,Q \,.
\end{equation}

\begin{figure}
\centering
\begin{tikzpicture}[
    scale=2,
    interface/.style={very thick},
    sph/.style={thick},
    dashedm/.style={densely dashed},
    field/.style={-Stealth, thick},
    normalv/.style={-Stealth, thick, gray!70},
    chargeplus/.style={font=\large, blue!70!black}
]

    % Parámetros geométricos
    \def\R{1.0}      % Radio de la esfera conductora
    \def\r{1.6}      % Radio de la superficie gaussiana (r > R)
    \def\yI{0.0}     % Altura de la interfaz (y = \yI)

    % Medios dieléctricos
    \fill[blue!8] (-2.5,-2) rectangle (2.5,\yI);
    \fill[green!5] (-2.5,\yI) rectangle (2.5,2);

    % Interfaz dieléctrica
    \draw[interface] (-2.5,\yI) -- (2.5,\yI);
    \node[anchor=south east, black!70] at (-2.4,\yI) {Gas, $\kappa_2$};
    \node[anchor=north east, black!70] at (-2.4,\yI) {Líquido, $\kappa_1$};

    % Esfera conductora
    \draw[sph, fill=gray!15] (0,0) circle (\R);
    \draw[dashedm] (C) -- ++(\R,0)
          node[midway, below] {$R$};
    \node[anchor=south west, black!80] at (-0.7,\R*0.015) {Esfera conductora};
    \fill (C) circle (1.2pt) node[below right, xshift=0mm] {O};
    \node[red] at (\R*0.9, \R*0.9) {$Q_{\text{free}}=Q$};

    % Carga en la superficie
    \foreach \ang in {0,5,10,...,360}{
        \node[red, fill=red, circle, inner sep=0.75pt] at ({\R*cos(\ang)},{\R*sin(\ang)}) {};
    }

    % Superficie gaussiana S_r (r>R)
    \draw[blue, dashed, thick] (0,0) circle (\r);
    \node[blue] at (0,-\r-0.18) {$r>R$};
    \node[blue] at (0,\r+0.15) {$S_r$};

    % Flechas radiales saliendo del conductor y cruzando S_r
    \foreach \ang in {30,150,270}{
        \draw[->, thick]
            ({1.05*\R*cos(\ang)}, {1.05*\R*sin(\ang)})
            -- ({1.05*\r*cos(\ang)}, {1.05*\r*sin(\ang)});
    }

    % Etiqueta del campo sobre la superficie gaussiana
    \node at (1.8,1) {$\epsilon(r,\theta)\,\vec{E}(\vec{r})$};

\end{tikzpicture}
\caption{Esquema de la ley de Gauss para el desplazamiento eléctrico
en una esfera conductora de radio $R$ semisumergida en la interfaz entre un líquido (constante dieléctrica $\kappa_1$) y un gas (constante dieléctrica $\kappa_2$).}
\label{fig:gauss_fuera}
\end{figure}

%%%%%%% INTRODUCCION AL POTENCIAL 
Ahora bien, no conocemos el campo eléctrico macroscópico neto $\vec{E}(\vec{r})$ en la región exterior a la esfera, particularmente si este esta descrito por una función con dependencia radial, de modo que no cuente con dependencia angular alguna (que no este en función de las coordenadas $(\theta,\varphi)$). Por la simetría de revolución alrededor del eje $z$ del sistema, es de al menos suponer que $\vec{E}$ sea independiente de la coordenada azimutal $\varphi$. Por otra parte, dadas las regiones $\Omega_i$, con $i=1,2$,
\begin{align*}
    \Omega_1&=\left\{ (r,\theta,\varphi)\in\mathbb{R}^3\big|\;r>R \;,\; \pi /2\leq \theta \leq \pi \right\} \qquad \text{(hemisferio donde gobierna el liquido)} \,,\\[6pt]
    \Omega_2&=\left\{ (r,\theta,\varphi)\in\mathbb{R}^3\big|\;r>R\;,\; 0\leq \theta \leq \pi /2 \right\} \qquad \text{(hemisferio donde gobierna el gas)} \,,
\end{align*}
a priori no hay razón para descartar que el campo eléctrico macroscópico presente comportamientos distintos en cada una de las regiones dieléctricas $\Omega_1$ y $\Omega_2$.

Para despejar estas incógnitas, primero determinaremos el potencial electrostático 
$\Phi$ y luego obtendremos el campo eléctrico macroscópico mediante 
$\vec{E} = -\vec{\nabla}\Phi$. En ausencia de cargas libres, el potencial en un medio 
dieléctrico satisface
\begin{equation*}
    \vec{\nabla}\cdot(\varepsilon\,\vec{\nabla}\Phi)=0.
\end{equation*}

Como en el interior del conductor y en cada uno de los dos medios (gas y líquido) 
la permitividad $\varepsilon$ es constante, en cada región esta ecuación se reduce 
a la ecuación de Laplace
\begin{equation*}
    \nabla^{2}\Phi = 0.
\end{equation*}

Buscaremos entonces la solución general de $\nabla^{2}\Phi=0$ en el interior del 
conductor y en la región exterior $r>R$, e impondremos después las condiciones de 
borde físicas del sistema.

Dada la simetría azimutal del sistema, la solución general a la ecuación de Laplace quedara dada en función de los polinomios de Legendre (estandar). Salvaguardando \textbf{la hipótesis que el campo eléctrico se comportara de forma distinta dependiendo del medio en que se encuentra}, esto nos conduce a considerar el potencial electrostático como una función compuesta, la cual describe \textbf{potenciales electrostáticos en medios distintos} en función de la coordenada colatitudinal $\theta$ que es justamente la que discrimina el medio dieléctrico del sistema para $r>R$ según la ecuación \eqref{1.varepsilon}. De este modo, el potencial queda dado por
\begin{equation}\label{1.potencial_general}
    \Phi(\vec{r})=\left\{
    \begin{aligned}
        \sum_{n=0}^{\infty}\left(A_n^{(2)}r^n+B_n^{(2)}r^{-(n+1)}\right)P_n(\cos\theta)  \,, \quad \text{si}  \quad 0\leq \theta \leq \frac{\pi}{2}  \\[4pt]
        \sum_{n=0}^{\infty}\left(A_n^{(1)}r^n+B_n^{(1)}r^{-(n+1)}\right)P_n(\cos\theta)  \,, \quad \text{si} \quad \frac{\pi}{2} < \theta \leq \pi 
    \end{aligned}
    \right. \quad ,
\end{equation}
donde $A_n^{(1)}, \,B_n^{(1)}$ son las constantes a determinar por las condiciones de borde del potencial electrostático en el medio dieléctrico asociado a $\kappa_1$, forma análoga con $A_n^{(2)}, \,B_n^{(2)}$ respecto a $\kappa_2$. 

Por la continuidad del potencial electrostático en todo el espacio, de inmediato nos damos cuenta que \eqref{1.potencial_general} debe satisfacer la condición 
\begin{equation*}
    \lim_{\theta \to \frac{\pi}{2}^{+}} \Phi(\vec{r})=\lim_{\theta \to \frac{\pi}{2}^{-}} \Phi(\vec{r}) \,,
\end{equation*}
la cual debe cumplirse para todo $r>R$. Si se escribe explícitamente esta igualdad usando \eqref{1.potencial_general} en $\theta=\pi/2$, se obtiene una relación entre los coeficientes $A_n^{(1)},B_n^{(1)}$ y $A_n^{(2)},B_n^{(2)}$ (en particular para aquellos órdenes $n$ tales que $P_n(0)\neq 0$). Además, la expansión en polinomios de Legendre es única y el potencial electrostático es, en rigor, un solo campo escalar definido en todo el dominio, de modo que no se pierde generalidad al reagrupar la solución imponiendo que un mismo conjunto de coeficientes describa el potencial a ambos lados de la interfaz. En este sentido, es natural tomar
\begin{equation*}
    A_n^{(1)}=A_n^{(2)}\equiv A_n \qquad \text{y} \qquad B_n^{(1)}=B_n^{(2)}\equiv B_n\,.
\end{equation*}
Esto indica que el potencial electrostático es el mismo sobre ambos medios; cualquier diferencia entre ellos quedará codificada posteriormente en las relaciones constitutivas que vinculan $\Phi$ con los campos macroscópicos. Lo anterior tampoco demuestra aún que el campo eléctrico macroscópico $\vec{E}$ no tiene dependencia colatitudinal $\theta$, puesto que no hemos terminado de caracterizar la solución del potencial eléctrico con todas las condiciones de borde del sistema. 

De este modo el potencial electrostático en todo el espacio de momento queda dado por
\begin{equation}\label{1.potencial}
    \Phi(\vec{r})=\sum_{n=0}^{\infty}\left(A_nr^n+B_nr^{-(n+1)}\right)P_n(\cos\theta)  \,, 
\end{equation}

Dado que el potencial electrostático decrece cuando la norma de $\vec{r}$, denotada desde ahora como $\|\vec{r}\|=r$,  tiende a infinito y debe converger cuando la norma de $\vec{r}$ tiende a cero, entonces \eqref{1.potencial} satisfacer las condiciones
\begin{align}
    \lim_{r \to \infty} \Phi(\vec{r})=0 \,, \qquad r>R\;; \label{1.infinito} \\[6pt]
    \lim_{r\to 0} \Phi(\vec{r})<\infty \,, \qquad r<R\;. \label{1.cero}
\end{align}

Imponiendo \eqref{1.infinito} y \eqref{1.cero} sobre \eqref{1.potencial}, se obtiene 

\begin{equation}\label{1.potencial_2}
    \Phi(\vec{r})=\left\{
    \begin{aligned}
        &\sum_{n=0}^{\infty}A_nr^nP_n(\cos\theta)  \,,\qquad r<R\;; \\[6pt]
        &\sum_{n=0}^{\infty}B_nr^{-(n+1)}P_n(\cos\theta)  \,, \qquad r>R\;.
    \end{aligned}
    \right.
\end{equation}

Como ya se menciono, el potencial electrostático es descrito por una función continua, por lo mismo este debe satisfacer condiciones de continuidad en la superficie de la esfera conductora, lo cual impone la condición
\begin{equation}\label{1.Radio}
    \lim_{r \to R^{+}} \Phi(\vec{r})=\lim_{r \to R^{-}} \Phi(\vec{r}) = \Phi(R,\theta) \,.
\end{equation}

Imponiendo \eqref{1.Radio} sobre \eqref{1.potencial_2}, se sigue que
\begin{align*}
\lim_{r \to R^{-}}\sum_{n=0}^{\infty}A_nr^nP_n(\cos\theta)&=\lim_{r \to R^{+}}\sum_{n=0}^{\infty}B_nr^{-(n+1)}P_n(\cos\theta)= \Phi(R,\theta)\,,\\[6pt]
\Longrightarrow \qquad \sum_{n=0}^{\infty}A_nR^nP_n(\cos\theta)&=\sum_{n=0}^{\infty}B_nR^{-(n+1)}P_n(\cos\theta)\,,\\[6pt]
\Longrightarrow \qquad B_n&=A_nR^{2n+1} \,.
\end{align*}
Por lo que el potencial electrostacio se escribe como
\begin{equation}\label{1.potencial_3}
    \Phi(\vec{r})=\left\{
    \begin{aligned}
        &\sum_{n=0}^{\infty}B_nR^{-(2n+1)}r^nP_n(\cos\theta)  \,,\qquad r<R\;; \\[6pt]
        &\sum_{n=0}^{\infty}B_nr^{-(n+1)}P_n(\cos\theta)  \,, \qquad r>R\;.
    \end{aligned}
    \right.
\end{equation}
Como en la superficie del conductor el potencial electrostático es constante, entonces 
\begin{equation*}
    \Phi(R,\theta)=\Phi_0\,,
\end{equation*}
Entonces, para terminar de imponer la condición \eqref{1.Radio}, notemos que 
\begin{align*}
    \lim_{r \to R^{+}} \Phi(\vec{r})=\Phi(R,\theta) &\qquad \Longrightarrow \qquad \sum_{n=0}^{\infty}B_nR^{-(n+1)}P_n(\cos\theta)= \Phi_0 \\[4pt]
    &\qquad \Longrightarrow \qquad B_n = \left\{
    \begin{aligned}
        \Phi_0 R\,,\qquad n=0\;, \\[4pt]
        0\,,\qquad n\neq0\;.
    \end{aligned}
    \right.
\end{align*}
Por lo que el potencial electrostático adopta la forma:
\begin{equation}\label{1.potencial_full}
    \Phi(\vec{r})=\left\{
    \begin{aligned}
        &\Phi_0  \,,\qquad r<R\;; \\[6pt]
        &\frac{\Phi_0 R}{r} \,, \qquad r>R\;.
    \end{aligned}
    \right.
\end{equation}
Con ello se ve que el potencial electrostático resultante no depende de la coordenada angular $\theta$; la presencia de los dos dieléctricos sólo se manifiesta en el potencial electrostático sobre la superficie del conductor $\Phi_0$. Dado que el sistema es estacionario, podemos deducir la forma del campo eléctrico en todo el espacio por medio de la relación $\vec{E}=-\vec{\nabla}\Phi$, obteniéndose
\begin{equation}\label{1.campo_electrico}
    \vec{E}(r)=\left\{
    \begin{aligned}
        \vec0  \,&,\qquad r<R\;; \\[6pt]
        \frac{\Phi_0R}{r^2}\hat{r} \,&, \qquad r>R\;.
    \end{aligned}
    \right.
\end{equation}
Observe que \textbf{hemos demostrado la dependencia radial del campo eléctrico macroscópico en todo el espacio}. Ahora solo basta determinar el potencial electrostático en la superficie del conductor esférico. 

Para asegurarnos que la función compuesta dada en \eqref{1.campo_electrico} describa fielmente el campo eléctrico macroscópico del sistema, matemáticamente debemos determinar $\Phi_0$ de modo que este quede en función de los parámetros físicos que caracterizan al sistema. Para ello, consideremos la ecuación \eqref{1.gauss_desplazamiento_electrico} y sustituyamos en ella las ecuaciones \eqref{1.varepsilon} y \eqref{1.campo_electrico}, para $r>R$ y con $\hat{n}=\hat{r}$, es decir 
\begin{align*}
    Q &= \oint_{S_r} \epsilon(r,\theta)\vec{E}(\vec{r})\cdot d\vec{A} \\[4pt]
    &=  \int_{0}^{2\pi}\int_{0}^{\pi}\epsilon(r,\theta)\left(\frac{\Phi_0R}{r^2}\hat{r}\right)\cdot\left(r^2\sin\theta\,d\theta\,d\varphi \,\hat{n}\right) \\[4pt]
    &= R\Phi_0\int_{0}^{2\pi}\int_{0}^{\pi}\epsilon(r,\theta)\sin\theta\,d\theta\,d\varphi \,\hat{r}\cdot\hat{r} \\[4pt]
    &=  2\pi R\Phi_0\left(\int_{0}^{\frac{\pi}{2}}\epsilon_0\kappa_2\sin\theta\,d\theta\, + \int_{\frac{\pi}{2}}^{\pi}\epsilon_0\kappa_1\sin\theta\,d\theta \right) \\[4pt]
    &=  2\pi\epsilon_0 R\Phi_0\left(\kappa_2 + \kappa_1\right) \,,
\end{align*}
ahora bien podemos despejar $\Phi_0$, determinando así el potencial electrostático sobre la superficie del conductor esférico, de modo:
\begin{equation}\label{1.constante_C_0}
    \Phi_0= \frac{Q}{2\pi \epsilon_0 R(\kappa_2+\kappa_1)}\,.
\end{equation}

Sustituyendo \eqref{1.constante_C_0} en \eqref{1.campo_electrico}, \textbf{obtenemos el campo eléctrico macroscópico del sistema en toda región del espacio}:
\begin{equation}\label{1.campo_electrico_full}
    \vec{E}(r)=\left\{
    \begin{aligned}
        \vec0  \,&,\qquad r<R\;; \\[6pt]
        \frac{Q}{2\pi \epsilon_0 (\kappa_2+\kappa_1)r^2}\hat{r} \,&, \qquad r>R\;.
    \end{aligned}
    \right.
\end{equation}

Recordando la discusión sobre la polarización en los medios que nos llevo a presentar la ecuación \eqref{1.POLARIZACION}, podemos describir la \textbf{polarización del sistema en toda región del espacio} como una función compuesta definida de la forma
\begin{equation*}
    \vec{P}(r,\theta)=\epsilon_0\,\chi_e(r,\theta)\,\vec{E}(r)\,.
\end{equation*}

A partir del campo eléctrico macroscópico ya obtenido en \eqref{1.campo_electrico_full} y la función que describe la susceptibilidad dieléctrica del sistema dada en \eqref{1.chi}, se sigue entonces para la polarización del sistema:
\begin{equation}\label{1.polarizacion_espacio_calculado}
    \vec{P}(r,\theta)=
    \left\{
        \begin{aligned}
        \vec{0} \,, &\qquad   0<r<R \;;\\[6pt]
        \dfrac{Q\,(\kappa_2-1)}{2\pi (\kappa_2+\kappa_1)\,r^2}\,\hat{r}\,, &\qquad  r>R \;\wedge \; 0\leq \theta \leq \dfrac{\pi}{2} \;;\\[6pt]
        \dfrac{Q\,(\kappa_1-1)}{2\pi (\kappa_2+\kappa_1)\,r^2}\,\hat{r}\,, &\qquad r>R \;\wedge \;\dfrac{\pi}{2} < \theta \leq \pi \;.
        \end{aligned}
    \right.
\end{equation}


Con ello, calculamos las densidades de carga ligada, superficiales como volumétricas, por medio de las funciones
\begin{align*}
    \sigma_b(r,\theta)&=\vec{P}(r,\theta)\cdot\hat{n}_{\text{e}}\,, \\[6pt]
    \rho_b(r,\theta)&=-\vec\nabla\cdot\vec{P}(r,\theta) \,.
\end{align*}
Aquí $\hat{n}_{\text e}$ denota el vector normal unitario saliendo de cada medio dieléctrico. En la interfaz esférica conductor–dieléctrico (superficie $r=R$) el volumen dieléctrico ocupa la región $r>R$, de modo que la normal saliente del medio apunta hacia el conductor y se tiene
\[
\hat{n}_{\text e} = -\hat r \qquad \text{sobre } r=R\,.
\]
En la interfaz plana entre los dos medios (plano $\theta = \pi/2$) la normal es perpendicular a dicho plano, pero como $\vec P$ es puramente radial, $\vec P\cdot\hat n_{\text e}=0$ allí y no aparece carga ligada superficial en esa frontera.

Evaluando entonces la densidad de carga ligada superficial sobre la superficie de contacto entre el conductor y cada dieléctrico, esto es, en $r=R^{+}$, se obtiene
\begin{equation*}
    \sigma_b(\theta)=\vec P(R^+,\theta)\cdot\hat n_{\text e}
    =
    \left\{
        \begin{aligned}
        -\,\dfrac{Q\,(\kappa_2-1)}{2\pi (\kappa_2+\kappa_1)\,R^2}\,, &\qquad  r=R^{+} \;\wedge \; 0\leq \theta \leq \dfrac{\pi}{2} \;;\\[6pt]
        -\,\dfrac{Q\,(\kappa_1-1)}{2\pi (\kappa_2+\kappa_1)\,R^2}\,, &\qquad r=R^{+} \;\wedge \;\dfrac{\pi}{2} < \theta \leq \pi \;.
        \end{aligned}
    \right.
\end{equation*}
El signo menos proviene precisamente de la elección $\hat n_{\text e}=-\hat r$, esto es, de tomar la normal saliendo del medio dieléctrico hacia el interior del conductor.

Por otra parte, para la densidad volumétrica de cargas ligadas basta notar que, en la región donde $r>R$, se tiene
\[
\vec\nabla\cdot\vec P(r,\theta)
=
\dfrac{1}{r^2}\dfrac{\partial}{\partial r}\Big(r^2 \vec{P}(r,\theta)\Big)=0
\]
para $r\neq 0$, y además $\vec P=\vec 0$ en el interior del conductor. Por lo tanto, la densidad volumétrica de cargas ligadas será nula en todo el espacio:
\begin{equation*}
    \rho_b(r,\theta)=0 \,.
\end{equation*}

En conclusión, el campo eléctrico en todo el espacio está dado en \eqref{1.campo_electrico_full} y las densidades de carga ligadas son únicamente superficiales mas no volumétricas, concentrándose sobre la superficie de contacto $r=R$ entre el conductor y cada medio dieléctrico, como se observa en las expresiones anteriores.


\clearpage
%%%%%%%%%%%%%%%%%%%%%%%%%%%%%%%%%%%%%%%%%%%%%%%%%%%%%%%%%%%%%%%%%%%%%%%%%%%%%%%%%%%%%%%%%%%%%%%%%%%%%%%%%%%%%%%%%%%%%%%%%%%%%%%%%%%%%%%%%%%%%%%%%%%%%%%%%%%%%%%%%%%%%%%%%%%%%%%%%%%%%%%%%%%%%%%%%%%%%%%

\section{Problema 2}
Una esfera de radio \(R\), que tiene una densidad de carga superficial \(\sigma\) constante rígidamente ligada
a ella, gira con velocidad angular \(\omega\) alrededor de un eje que pasa por su centro.
\begin{enumerate}\itemsep0.2em
  \item Integrando directamente la expresión general para el potencial vectorial de una distribución de
  corriente, encuentre una expresión para el campo magnético \(\vec B\) producido en su exterior.

  \textbf{Sugerencia.} La siguiente identidad podría ser de utilidad:
  \begin{equation*}
    \oint_{\partial V} \frac{dS'_i}{\lvert \mathbf{x}-\mathbf{x}'\rvert}
    \;=\;
    \int_V \partial'_i\!\left(\frac{1}{\lvert \mathbf{x}-\mathbf{x}'\rvert}\right)\, dV'
    \;=\;
    -\int_V \partial_i\!\left(\frac{1}{\lvert \mathbf{x}-\mathbf{x}'\rvert}\right)\, dV'
    \;=\;
    -\,\partial_i\!\left(\int_V \frac{dV'}{\lvert \mathbf{x}-\mathbf{x}'\rvert}\right).
  \end{equation*}

  \item Grafique el campo magnético fuera de la esfera.
\end{enumerate}

\textbf{Solución:}

%%%%%%%%%%%%%%%%%%%%%%%%%%%%%%%%%%%%%%%%%%%%%%%%%%%%%%%%%%%%%%%%%%%%%%%%%%%%%%%%%%%%%%%%%%%%%%%%%%%%%%%%%%%%%%%%%%%%%%%%%%%%%%%%%%%%%%%%%%%%%%%%%%%%%%%%%%%%%%%%%%%%%%%%%%%%%%%%%%%%%%%%%%%%%%%%%%%%%%%

\subsection{Campo magnético en el exterior de la esfera a través del potencial vectorial por integración directa}

Elegimos un sistema de coordenadas esféricas $(r,\theta,\varphi)$ con origen en el centro de la
esfera, alineando la dirección positiva del eje $z$ con el eje de rotación de la esfera (Véase la Figura~\ref{fig:sistema}). De este modo, la velocidad angular de la esfera queda descrita de la forma
\begin{equation*}
    \vec{\omega} = \omega\,\hat z.
\end{equation*}

\begin{figure}
    \centering
    \includegraphics[width=1\linewidth]{fig1.png}
    \caption{Sistema de coordenadas esféricas $(r,\theta,\varphi)$ utilizado para describir la esfera conductora de radio $R$ que rota con velocidad angular $\vec{\omega}$ alrededor del eje $z$. Se indica un punto genérico $\vec r\,'$ sobre la superficie, empleado en la integración del potencial vectorial.}
    \label{fig:sistema}
\end{figure}

Dado que la densidad de carga superficial $\sigma$ de la esfera, es constante y rígidamente ligada a ella, se tendrá para la \textbf{densidad de corriente superficial} que esta tendrá dependencia de la velocidad de un punto arbitrario sobre la superficie del cuerpo esférico, esto de la forma
\begin{equation}\label{2.densidad_de_corriente}
  \vec K(\vec{r}\,')
  = \sigma\,\vec v(\vec{r}\,'),
\end{equation}
donde $\vec{r}\,'$ es un punto arbitrario de la superficie esférica de radio $R$, descrito en coordenadas esféricas $(R,\theta',\phi')$ con el vector de posición
\begin{equation*}
  \vec r\,' = R\,\hat r\,'\,.
\end{equation*}
La velocidad de un punto de la superficie en un cuerpo rígido en rotación, como lo es la esfera, estara dada por el producto cruz entre la velocidad angular del cuerpo y un vector posición sobre la superficie del mismo, de manera
\begin{equation}\label{2.velocidad}
\vec v(\vec r\,') = \vec{\omega}\times\vec r\,'.
\end{equation}

Sustituyendo \eqref{2.velocidad} en \eqref{2.densidad_de_corriente}, se obtiene la densidad de corriente superficial en función de un punto arbitrario sobre la superficie de la esfera, es decir
\begin{equation}
  \vec K(\vec{r}\,')
  = \sigma\,(\vec{\omega}\times\vec{r}\,').
  \label{2.K_def}
\end{equation}

En la magnetostática, el \textbf{potencial vectorial magnético} $\vec{A}$ generado por una distribución de corriente
superficial $\vec K$ definido sobre una superficie $S_R$ (en este caso, la superficie de la esfera de radio $R$ es denotada por $S_R$) está dado por 
\begin{equation}
  \vec A(\vec{r})
  = \frac{\mu_0}{4\pi}
    \oint_{S_R}
    \frac{\vec K(\vec{r}\,')}{\|\vec r - \vec r\,'\|}\,dS',
  \label{2.A_general}
\end{equation}
donde $\vec r$ es el punto de observación; $dS'$ es el elemento de área sobre la superficie de la esfera de radio $R$ y por consiguiente la integral recorre toda la superficie de la misma. 

Sustituyendo \eqref{2.K_def} en \eqref{2.A_general} y usando
$\vec r\,' = R\,\hat r\,'$, se obtiene
\begin{align*}
  \vec A(\vec r)
  &= \frac{\mu_0}{4\pi}
     \oint_{S_R}
     \frac{\sigma\bigl(\vec\omega\times\vec r\,'\bigr)}{\|\vec r - \vec r\,'\|}\,dS'
     \\[4pt]
    &= \frac{\mu_0\sigma}{4\pi}
     \oint_{S_R}
     \frac{\vec\omega\times (R\,\hat{r}\,')}{\|\vec r - \vec r\,'\|}\,dS'
     \\[4pt]
  &= \frac{\mu_0\sigma R}{4\pi}
     \oint_{S_R}
     \frac{\vec\omega\times\hat r\,'}{\|\vec r - \vec r\,'\|}\,dS'\,.
\end{align*}
Es conveniente reescribir el elemento de área $dS'$ como un elemento de área vectorial de forma
$d\vec S' = \hat r\,'\,dS'$. Con ello,
\begin{equation*}
  \bigl(\vec\omega\times\hat r\,'\bigr)dS'
  = \vec\omega\times(\hat r\,'\,dS')
  = \vec\omega\times d\vec S'\,,
\end{equation*}
y la expresión para el potencial vectorial magnetico se simplifica a
\begin{equation*}
  \vec A(\vec r)
  \;=\; \frac{\mu_0\sigma R}{4\pi}
  \oint_{S_R}
  \frac{\vec\omega\times d\vec S'}{\|\vec r - \vec r\,'\|}\,.
\end{equation*}
Como $\vec\omega$ es constante, puede salir de la integral:
\begin{equation}
  \vec A(\vec r)
  =\frac{\mu_0\sigma R}{4\pi}\,
  \vec\omega\times
  \oint_{S_R}
  \frac{d\vec S'}{\|\vec r - \vec r\,'\|}\,.
  \label{2.A_desarrollo}
\end{equation}

Consideremos la identidad sugerida, para una superficie cerrada $S$ que encierra un volumen $V$,
\begin{equation*}
    \oint_S\frac{d\vec{S}}{\|\vec{r}-\vec{r}\,'\|}=-\nabla\left(\int_V\frac{dV'}{\|\vec{r}-\vec{r}\,'\|}\right)\,.
\end{equation*}

En nuestro caso $V_R$ denota el volumen de la esfera conductora de radio $R$ cuya superficie, como ya se menciono anteriormente, es denotada por $S_R$. Vemos que la identidad sugerida la podemos implementar en \eqref{2.A_desarrollo} de manera

\begin{equation}
  \vec A(\vec r)
  = \frac{\mu_0\sigma R}{4\pi}\,
  \vec\omega\times
  \left[-\nabla\left(\int_{V_R}\frac{dV'}{\|\vec{r}-\vec{r}\,'\|}\right)\right]\,.
  \label{2.A_desarrollo_mas}
\end{equation}

Ahora bien, para $r>R$, es decir en la región exterior de la esfera, se tendrá (véase el Apéndice~\ref{ap:potencial} para una demostración rigurosa de esta identidad)
\begin{equation}\label{2.integral_volumen}
    \int_V\frac{dV'}{\|\vec{r}-\vec{r}\,'\|}=\frac{4\pi R^3}{3}\frac{1}{r}\,,
\end{equation}
donde $r:=\|\vec{r}\|$.

Calculando el gradiente de \eqref{2.integral_volumen}, con $V:=V_R$, se obtiene
\begin{equation}\label{2.gradiante_integral_volumen}
    \nabla\left(\int_{V_R}\frac{dV'}{\|\vec{r}-\vec{r}\,'\|}\right)=-\frac{4\pi R^3}{3}\frac{1}{r^2}\hat{r} \,.
\end{equation}

Sustituyendo \eqref{2.gradiante_integral_volumen} en \eqref{2.A_desarrollo}, se sigue el desarrollo del potencial magnetostatico en la región exterior de la esfera conductora, $r>R$, de modo 
\begin{align*}
    \vec{A}(\vec{r})&=\frac{\mu_0\sigma R}{4\pi} \vec{\omega}\times
    \left(\frac{4\pi R^3}{3}\frac{1}{r^2}\hat{r}\right) \\[6pt]
    &=\frac{\mu_0\sigma R^4}{3r^2} \vec{\omega}\times
    \hat{r} \\[6pt]
    &=\frac{\mu_0\sigma R^4\omega}{3r^2}\sin\theta\,\hat\varphi \,,
\end{align*}
donde, dado que $\vec{\omega}=\omega\hat{z}$ y $\vec{\omega}\times\hat{r}=\omega\sin\theta\,\hat\varphi$. Finalmente se obtiene
\begin{equation}\label{2.A_vectorial_exterior}
    \vec{A}(\vec{r})=\frac{\mu_0\sigma R^4\omega}{3r^2}\sin\theta\,\hat\varphi \,, \qquad r>R\,.
\end{equation}

Este resultado demuestra que $\vec A$ es puramente azimutal y decae como
$1/r^2$ en el exterior. Con el potencial vectorial ya determinado en la región exterior de la esfera,~\eqref{2.A_vectorial_exterior}, estamos en condiciones de calcular el campo magnético.

Recordemos que, en magnetostática, el campo magnético es igual al rotor del potencial vectorial magnético, es decir
\begin{equation*}
    \vec B(\vec r) = \vec\nabla\times\vec A(\vec r).
\end{equation*}
Dicho rotor en coordenadas esféricas se calcula como:
\begin{equation*}
    \vec{B}(\vec{r}) = \frac{1}{r^2 \sin\theta}
        \begin{vmatrix}
          \hat{r} & r\hat{\theta} & r\sin\theta\,\hat{\varphi} \\[0.5em]
          \dfrac{\partial}{\partial r} & \dfrac{\partial}{\partial \theta} & \dfrac{\partial}{\partial \varphi} \\[1em]
          A_r & r A_\theta & r\sin\theta\, A_\varphi
        \end{vmatrix}
\end{equation*}
Ahora bien, para un campo puramente azimutal, como es el caso del potencial vectorial magnetico que describe este sistema, $\vec A = A_\varphi(r,\theta)\,\hat\varphi$, el determinante anterior toma una forma particularmente sencilla:
\begin{equation*}
    \vec{B}(\vec{r}) = \frac{1}{r^2 \sin\theta}
        \begin{vmatrix}
          \hat{r} & r\hat{\theta} & r\sin\theta\,\hat{\varphi} \\[0.5em]
          \dfrac{\partial}{\partial r} & \dfrac{\partial}{\partial \theta} & \dfrac{\partial}{\partial \varphi} \\[1em]
          0 & 0 & r\sin\theta\, A_\varphi
        \end{vmatrix} \,,
\end{equation*}
del cual se deduce que el campo magnético del sistema estará descrito por
\begin{equation}\label{2.BBBB}
    \vec{B}(\vec{r})=B_r(\vec{r})\,\hat{r}+B_{\theta}(\vec{r})\,\hat{\theta}\,,
\end{equation}
donde las componentes de dicho campo vectorial están determinadas por
\begin{align*}
    B_r &= \frac{1}{r\sin\theta}\,
          \frac{\partial}{\partial\theta}
          \big(\sin\theta\,A_\varphi\big)\,,\\[6pt]
    B_\theta &= -\frac{1}{r}\,
               \frac{\partial}{\partial r}
               \big(r\,A_\varphi\big)\,,\\[6pt]
    B_\varphi &= 0\,.
\end{align*}
Identificando desde \eqref{2.A_vectorial_exterior} a $A_\varphi$, como
\begin{equation*}
    A_\varphi(r,\theta)
    = \frac{\mu_0 R^4\sigma\omega}{3}\,
      \frac{\sin\theta}{r^2},
\end{equation*}
es claro que toda la información de $\vec B$ está codificada en la dependencia radial y angular de $A_\varphi$. 

Derivando explícitamente en las componentes no nulas del campo magnético del sistema, se obtienen las siguientes igualdades:

\begin{itemize}
    \item \textbf{Para la componente radial de $\vec{B}(r,\theta)$:}
        \begin{align}
          B_r(r,\theta)
          &= \frac{1}{r\sin\theta}\,
                  \frac{\partial}{\partial\theta}
                  \big(\sin\theta\,A_\varphi\big) \notag\\[6pt]
          &= \frac{1}{r\sin\theta}
             \frac{\partial}{\partial\theta}
             \left(
               \frac{\mu_0 R^4\sigma\omega}{3}\,
               \frac{\sin^2\theta}{r^2}
             \right) \notag\\[6pt]
           &= \frac{2\mu_0 R^4\sigma\omega}{3}\,
             \frac{\cos\theta}{r^3} \notag \,,
          \\[18pt]
            \therefore \qquad B_r(r,\theta)&= \frac{2\mu_0 R^4\sigma\omega}{3}\,\frac{\cos\theta}{r^3} \label{2.B_r}\,.
    \end{align}
    \item \textbf{Para la componente colatitudinal de $\vec{B}(r,\theta)$:}
    \begin{align}
        B_\theta(r,\theta)
        &= -\frac{1}{r}\,
                       \frac{\partial}{\partial r}
                       \big(r\,A_\varphi\big) \notag\\[6pt]
        &= -\frac{1}{r}
             \frac{\partial}{\partial r}
             \left(
               r\,
               \frac{\mu_0 R^4\sigma\omega}{3}\,
               \frac{\sin\theta}{r^2}
             \right) \notag \\[6pt]
        &= \frac{\mu_0 R^4\sigma\omega}{3}\,
             \frac{\sin\theta}{r^3}\,, \notag \\[18pt]
             \therefore \qquad B_\theta(r,\theta)&= \frac{\mu_0 R^4\sigma\omega}{3}\,
             \frac{\sin\theta}{r^3}\,. \label{2.B_theta}
    \end{align}
\end{itemize}

Sustituyendo \eqref{2.B_r} y \eqref{2.B_theta} en \eqref{2.BBBB}, especificando ahora la dependencia radial y colatitudinal $(r,\theta)$, se obtiene el campo magnético del sistema en la región exterior de la esfera
\begin{equation}
  \vec B(r,\theta)
  = \frac{\mu_0 R^4\sigma\omega}{3r^3}
    \big(
      2\cos\theta\,\hat r
      + \sin\theta\,\hat\theta
    \big),
  \qquad r>R.
  \label{2.B_exterior}
\end{equation}

\subsection{Gráfico del campo magnético al exterior de la esfera}

Para responder a la segunda parte del problema, basta con representar el campo
\eqref{2.B_exterior} en el plano que contiene al eje de rotación (por ejemplo, el plano $zy$).
Las líneas de campo tendrán la forma característica de un dipolo magnético: salen de la
región polar “norte” de la esfera, se arquean hacia afuera y vuelven a entrar por la región
polar “sur”, cerrándose en bucles alrededor del eje. Cuanto más alejado del origen, más
parecido es el campo al de un dipolo puntual y más rápidamente decae su intensidad como
$1/r^3$ (Véase la Figura~\ref{fig:campo_magnetico}).

Para generar la Figura~\ref{fig:campo_magnetico}, se utilizó la expresión
del campo magnético en la región exterior, ecuación~\eqref{2.B_exterior},
reescrita en coordenadas cartesianas en el plano $zy$ (con $x=0$). En este
plano se tiene $r = \sqrt{y^2 + z^2}$ y $\cos\theta = z/r$, de modo que el
campo dipolar de momento $\vec m \parallel \hat z$ puede escribirse como
\begin{align*}
    B_y(y,z) &= \frac{3C\,yz}{r^5}\,,\\[6pt]
    B_z(y,z) &= \frac{C\,(2z^2 - y^2)}{r^5}\,,\\[6pt]
    C &= \frac{\mu_0\sigma\omega R^4}{3}\,,
\end{align*}
que es la forma implementada en el 
\href{https://udeconce-my.sharepoint.com/:u:/g/personal/jrosas2022_udec_cl/IQBozq0VCHkbSb6BNYfmKNA_ATMKwhhBu3T-bwQgbCYgBBU?e=SZUAbL}{\texttt{script numérico}}
utilizado para trazar las líneas de campo.


\begin{figure}
    \centering
    \includegraphics[width=1\linewidth]{fig2.png}
    \caption{Líneas de campo magnético generadas por la esfera cargada en rotación,
    calculadas a partir de \eqref{2.B_exterior} y representadas en el plano $zy$.
    La región sombreada corresponde a la proyección de la esfera conductora, se indica
    el eje de rotación $\boldsymbol{\omega}$ y la posición de los polos magnéticos
    norte y sur. El patrón de líneas ilustra el comportamiento característico de un
    dipolo magnético alineado con el eje $z$.}
    \label{fig:campo_magnetico}
\end{figure}

\clearpage
%%%%%%%%%%%%%%%%%%%%%%%%%%%%%%%%%%%%%%%%%%%%%%%%%%%%%%%%%%%%%%%%%%%%%%%%%%%%%%%%%%%%%%%%%%%%%%%%%%%%%%%%%%%%%%%%%%%%%%%%%%%%%%%%%%%%%%%%%%%%%%%%%%%%%%%%%%%%%%%%%%%%%%%%%%%%%%%%%%%%%%%%%%%%%%%%%%%%%%%

\begin{appendices}
\section{Demostración del Potencial Newtoniano para una Esfera Uniforme}
\label{ap:potencial}
La integral del lado izquierdo de \eqref{2.integral_volumen} puede
interpretarse como el potencial newtoniano generado por una esfera
homogénea de densidad unitaria ocupando el volumen $V$:
\begin{equation*}
    I(r) := \int_V \frac{dV'}{\lVert\vec r - \vec r'\rVert}.
\end{equation*}
Para $r>R$ el punto de observación se encuentra fuera de la esfera, de
modo que toda la fuente está contenida en el interior. En esa región se
tiene
\begin{equation*}
    \nabla^2 I(r) = \int_V \nabla^2
    \left(\frac{1}{\lVert\vec r - \vec r'\rVert}\right) dV' = 0 \,,
\end{equation*}
ya que $\nabla^2 (1/\lVert\vec r - \vec r'\rVert)
= -4\pi \delta^{(3)}(\vec r-\vec r')$ y el soporte de la delta queda
fuera del dominio considerado. Por simetría esférica $I(r)$ depende
sólo del módulo $r=\lVert\vec r\rVert$, y la solución general de la
ecuación de Laplace con simetría radial en la región $r>R$ es de la
forma
\begin{equation*}
    I(r) = A + \frac{B}{r} \,.
\end{equation*}
La condición de decaimiento $I(r)\to 0$ cuando $r\to\infty$ obliga a
tomar $A=0$. Para determinar $B$ basta analizar el comportamiento
asintótico de la integral para $r\gg R$. En ese límite puede
aproximarse
\begin{equation*}
    \frac{1}{\lVert\vec r - \vec r'\rVert}
    = \frac{1}{r} + \mathcal O\!\left(\frac{R^2}{r^3}\right),
\end{equation*}
de modo que
\begin{equation*}
    I(r) = \int_V \frac{dV'}{\lVert\vec r - \vec r'\rVert}
    = \frac{1}{r} \int_V dV' + \mathcal O\!\left(\frac{R^2}{r^3}\right)
    = \frac{\mathrm{Vol}(V)}{r}
      + \mathcal O\!\left(\frac{R^2}{r^3}\right).
\end{equation*}
Como $\mathrm{Vol}(V) = \tfrac{4\pi R^3}{3}$, el término dominante a
grandes distancias es
\begin{equation*}
    I(r) \sim \frac{4\pi R^3}{3}\,\frac{1}{r} \,.
\end{equation*}
Por unicidad de la solución de la ecuación de Laplace con estas
condiciones de contorno, este comportamiento asintótico fija
unívocamente el coeficiente $B$, y se concluye que
\begin{equation*}
    I(r) = \frac{4\pi R^3}{3}\,\frac{1}{r} \,, \qquad r>R,
\end{equation*}
lo que justifica la igualdad empleada en \eqref{2.integral_volumen}.

\end{appendices}

\end{document}
