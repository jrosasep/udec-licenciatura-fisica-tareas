\documentclass{article}
\usepackage{csquotes}

% Paquetes matemáticos y tipográficos
\usepackage{cancel}
\usepackage{mathrsfs}
\usepackage{amssymb}
\usepackage{amsmath}
\usepackage{amsfonts}
\usepackage{mathtools}

% Permite referencias personalizadas
\usepackage{nameref}

% Numeración de ecuaciones por sección
\numberwithin{equation}{section}

% Hipervinculos
\usepackage[colorlinks=true,
            linkcolor=black,
            urlcolor=black,
            citecolor=black,
            filecolor=black,
            pdfborder={0 0 0}]{hyperref}

% Idioma en español
\usepackage[spanish]{babel}

% Manejo de imágenes
\usepackage{graphicx} 
\graphicspath{ {images/} }

% Configuración de márgenes
\usepackage[a4paper, left=1.5cm, right=1.5cm, top=20mm, bottom=20mm]{geometry}

% Tipografía mejorada
\usepackage{lmodern}

% Estilo de títulos con punto después del número
\usepackage{titlesec}
\titleformat{\section}{\huge\bfseries}{\thesection.}{1em}{}  % Título más grande

% Encabezados sin pie de página
\usepackage{fancyhdr}
\pagestyle{fancy}
\fancyhf{}
\fancyhead[L]{\textit{Ecuaciones Diferenciales y Funciones de Legendre}}
\fancyhead[R]{Física Matemática 2}

% Mejor separación de párrafos
\setlength{\parindent}{0pt}
\setlength{\parskip}{5pt}

% Evita hifenaciones excesivas
\sloppy

% Configuración del índice
\usepackage{tocloft}
\setcounter{tocdepth}{2}

\begin{document}

% Portada
\begin{titlepage}
    \centering
    \vspace*{3cm} % Ajuste en la posición vertical
    % Logo centrado
    \includegraphics[width=0.6\textwidth]{UdeC_azul_centrado.png} 
    
    \vspace{1cm}
    \thispagestyle{empty} % Sin número en la portada

    % Título de la tarea
    {\Huge \textbf{Tarea 02 - Ecuaciones Diferenciales y Funciones de Legendre} \par}
    
    \vspace{0.5cm}
    {\Huge \textbf{Física Matemática 2} \par}
    \vspace{1.5cm}

    % Nombre del autor
    {\Large José Ignacio Rosas Sepúlveda \par}
    \vspace{1cm}
    
    % Fechas de la tarea
    {\Large Abril - Mayo 2025 \par}
    \vfill
\end{titlepage}

% Índice
\tableofcontents
\newpage

%%%%%%%%%%%%%%%%%%%%%%%%%%%%%%%%%%%%%%%%%%%%%%%%%%%%%%%%%%%%%%%%%%%%%%%%%%%%%%%%%%%%%%%%%%%%%%%%%%%%%%%%%%%%%%%%%%%%%%%%%%%%%%%%%%%%%%%%%%%%%%%%%%%%%%%%%%%%%%%%%%%%%%%%%%%%%%%%%%%%%%%%%%%%%%%%%%%%%%%

\section{Ejercicio 1}
Una cuerda de longitud $L$, fijada en sus dos extremos, es pulsada en su punto medio por una cantidad $A$ (con unidades de longitud), para luego ser liberada, de modo que el desplazamiento inicial es dado por
\begin{equation*}
u(x,0) =
\begin{cases}
\displaystyle \frac{2Ax}{L}, & 0 \leq x \leq \frac{L}{2}, \\[8pt]
\displaystyle \frac{2A(L - x)}{L}, & \frac{L}{2} \leq x \leq L.
\end{cases}
\end{equation*}
Suponga además que la cuerda se encuentra quieta en $t = 0$.

%%%%%%%%%%%%%%%%%%%%%%%%%%%%%%%%%%%%%%%%%%%%%%%%%%%%%%%%%%%%%%%%%%%%%%%%%%%%%%%%%%%%%%%%%%%%%%%%%%%%%%%%%%%%%%%%%%%%%%%%%%%%%%%%%%%%%%%%%%%%%%%%%%%%%%%%%%%%%%%%%%%%%%%%%%%%%%%%%%%%%%%%%%%%%%%%%%%%%%%

\subsection{Identificación del modelo y condiciones de borde}

Identifique la ecuación que debe utilizar, junto a las condiciones de borde físicas impuestas al sistema.

\textbf{Solución:} 

%%%%%%%%%%%%%%%%%%%%%%%%%%%%%%%%%%%%%%%%%%%%%%%%%%%%%%%%%%%%%%%%%%%%%%%%%%%%%%%%%%%%%%%%%%%%%%%%%%%%%%%%%%%%%%%%%%%%%%%%%%%%%%%%%%%%%%%%%%%%%%%%%%%%%%%%%%%%%%%%%%%%%%%%%%%%%%%%%%%%%%%%%%%%%%%%%%%%%%%

Consideramos un sistema de referencia bidimensional con eje horizontal $x$ (posición a lo largo de la cuerda) y eje vertical $u$ (desplazamiento transversal). En su configuración de equilibrio, la cuerda se encuentra alineada con el eje $x$. 

Supondremos las siguientes hipótesis físicas fundamentales para el modelado:
\begin{itemize}
    \item La cuerda es ideal, es decir esta es perfectamente flexible, sin rigidez ni fricción interna. 
    \item Las partículas que componen la cuerda se desplazan únicamente en dirección al eje $u$.
    \item Las oscilaciones son pequeñas, lo que implica que la pendiente $\frac{\partial u}{\partial x}$ permanece acotada y $\ll 1$ en todo instante.
    \item La tensión de la cuerda es siempre en la dirección de la tangente, y la pendiente en cualquier punto de la cuerda desplazada es pequeña ($\ll 1$).
    \item Se asume que la tensión $T$ se mantiene constante en toda la longitud de la cuerda en todo instante.
\end{itemize}

Analicemos un segmento de cuerda entre los puntos $x$ y $x + \Delta x$. Su masa es aproximadamente $\rho\,\Delta x$, donde $\rho$ es la \textbf{densidad lineal de masa} y $\Delta x$ representa una longitud infinitesimal arbitraria de cuerda. La tensión $T$, que actúa siempre \textbf{tangencialmente}, tiene \textbf{componentes verticales} en los extremos dadas por $T\sin\alpha$ y $T\sin\beta$, izquierdo y derecho respectivamente, tal como se observa en la Figura \ref{fig:cuerda}.
\begin{figure}[ht]
    \centering
    \includegraphics[width=0.5\linewidth]{1.1_fig1.jpg}
    \caption{Segmento de cuerda desplazado transversalmente entre los puntos $x$ y $x + \Delta x$.}
    \label{fig:cuerda}
\end{figure}

Para analizar las fuerzas sobre el segmento considerado, notamos que, por hipótesis, no hay movimiento en la dirección horizontal. Por tanto, la suma de fuerzas en el eje $x$ se anula. Bajo estas consideraciones, la fuerza neta ejercida sobre el segmento, en dirección vertical, corresponde a la diferencia entre las componentes verticales de la tensión en sus extremos:

\begin{equation}\label{1.1_fuerza_neta_seno}
    F(x,t)=T\sin\beta - T\sin\alpha \,.
\end{equation}

Dado que por hipótesis trabajamos bajo el régimen de pequeñas oscilaciones, los ángulos $\alpha$ y $\beta$ son pequeños. Esto permite aplicar la aproximación:
\begin{align*}
    \alpha&\approx \sin\alpha\approx\tan\alpha \,,\\
    \beta&\approx\sin\beta\approx \tan\beta \,.
\end{align*}
En consecuencia, la expresión para la fuerza neta vertical se reescribe, bajo la aproximación angular, como:
\begin{equation}\label{1.1_fuerza_neta}
    F(x,t)=T\tan\beta - T\tan\alpha \,.
\end{equation}
Por otra parte, en pequeñas oscilaciones como es el caso que estamos considerando, por hipótesis la pendiente $\frac{\partial u}{\partial x}$ es pequeña. Así, podemos considerar las siguientes aproximaciones que por conveniencia tomaremos como igualdades:
\begin{align}
    \tan\alpha&=\frac{\partial u}{\partial x}(x,t)\,, \label{1.1_tan_alpha}\\
    \tan\beta&=\frac{\partial u}{\partial x}(x+\Delta x,t)\,.\label{1.1_tan_beta}
\end{align}

Sustituyendo \eqref{1.1_tan_alpha} y \eqref{1.1_tan_beta} en \eqref{1.1_fuerza_neta}, obtenemos:

\begin{equation}\label{1.1_fuerza_neta_2}
    F(x,t)=T\left( \frac{\partial u}{\partial x}(x+\Delta x,t) - \frac{\partial u}{\partial x}(x,t )\right) \,.
\end{equation}
Definimos $u(x,t)$ como el desplazamiento transversal (vertical) de la cuerda respecto a su posición de equilibrio, en la posición horizontal $x$ y tiempo $t$. Aplicamos la segunda ley de Newton al segmento considerado. Como su masa es $m = \rho\,\Delta x$, tenemos:
\begin{equation}\label{1.1_segunda_ley_de_newton_1}
    \rho\, \Delta x\,\frac{\partial^2 u}{\partial t^2}(x,t) = T\left( \frac{\partial u}{\partial x}(x+\Delta x,t) - \frac{\partial u}{\partial x}(x,t)\right)\,.
\end{equation}

Dividiendo ambos lados de \eqref{1.1_segunda_ley_de_newton_1} entre $\rho\,\Delta x$:
\begin{equation}\label{1.1_segunda_ley_de_newton_2}
    \frac{\partial^2 u}{\partial t^2}(x,t) = \frac{T}{\rho \,\Delta x}\left( \frac{\partial u}{\partial x}(x+\Delta x,t) - \frac{\partial u}{\partial x}(x,t)\right)\,.
\end{equation}

Tomando el límite $\Delta x \to 0$ en \eqref{1.1_segunda_ley_de_newton_2}, obtenemos:
\begin{equation}\label{1.1_segunda_ley_de_newton_3}
    \frac{\partial^2 u}{\partial t^2}(x,t) = \frac{T}{\rho}\lim_{\Delta x \to 0}\frac{1}{\Delta x}\left( \frac{\partial u}{\partial x}(x+\Delta x,t) - \frac{\partial u}{\partial x}(x,t)\right) \,.
\end{equation}

Notamos que el segundo miembro del lado derecho corresponde, por definición, a la segunda derivada de $u(x,t)$ respecto a $x$:
\begin{equation}\label{1.1_segunda_derivada}
    \frac{\partial^2u}{\partial x^2}(x,t)=\lim_{\Delta x \to 0}\frac{1}{\Delta x}\left( \frac{\partial u}{\partial x}(x+\Delta x,t) - \frac{\partial u}{\partial x}(x,t)\right)
\end{equation}

Sustituyendo \eqref{1.1_segunda_derivada} en \eqref{1.1_segunda_ley_de_newton_3}, obtenemos entonces una ecuación de onda del movimiento en el eje $u$ de la cuerda en el tiempo:
\begin{equation}\label{1.1_ecuación_de_onda}
\frac{\partial^2 u}{\partial t^2}(x,t) = c^2\,\frac{\partial^2u}{\partial x^2}(x,t) \,,
\end{equation}

donde $c$ representa la \textbf{velocidad de propagación de ondas} transversales en la cuerda:
\begin{equation*}
    c=\sqrt{\frac{T}{\rho}}
\end{equation*}

En resumen, bajo las hipótesis físicas adoptadas, el desplazamiento transversal $u(x,t)$ de la cuerda está gobernado por la ecuación de onda unidimensional \eqref{1.1_ecuación_de_onda}.

Finalmente, en cuanto a las condiciones de borde impuestas por el problema, el enunciado establece que la cuerda está fija en ambos extremos. Esto implica que \textbf{el desplazamiento vertical debe ser nulo en $x = 0$ y $x = L$ para todo tiempo}. Matemáticamente, estas condiciones se expresan como:

\begin{equation}\label{1.1_condiciones_de_borde}
u(0,t) = 0, \qquad u(L,t) = 0 \quad \text{para todo } t \geq 0.
\end{equation}

Estas son de tipo \textbf{Dirichlet}, ya que se impone directamente el valor de la función $u(x,t)$ en los extremos del intervalo\footnote{Parte del razonamiento presentado se apoya en el enfoque desarrollado en: \emph{Matemática Aplicada, Universidad Nacional de Trujillo} (2018), disponible en \url{https://mateapliunt.edu.pe/revista/Contenido/2018/1/Html/ART11/node2.html}.}.


\clearpage
%%%%%%%%%%%%%%%%%%%%%%%%%%%%%%%%%%%%%%%%%%%%%%%%%%%%%%%%%%%%%%%%%%%%%%%%%%%%%%%%%%%%%%%%%%%%%%%%%%%%%%%%%%%%%%%%%%%%%%%%%%%%%%%%%%%%%%%%%%%%%%%%%%%%%%%%%%%%%%%%%%%%%%%%%%%%%%%%%%%%%%%%%%%%%%%%%%%%%%%

\subsection{Solución por el método de separación de variables}

Utilizando el Método de Separación de Variables detalladamente, encuentre el desplazamiento en cualquier punto de la cuerda para $t > 0$.

\textbf{Solución:}

%%%%%%%%%%%%%%%%%%%%%%%%%%%%%%%%%%%%%%%%%%%%%%%%%%%%%%%%%%%%%%%%%%%%%%%%%%%%%%%%%%%%%%%%%%%%%%%%%%%%%%%%%%%%%%%%%%%%%%%%%%%%%%%%%%%%%%%%%%%%%%%%%%%%%%%%%%%%%%%%%%%%%%%%%%%%%%%%%%%%%%%%%%%%%%%%%%%%%%%

Consideramos la ecuación de onda unidimensional \eqref{1.1_ecuación_de_onda}, que constituye una \textbf{ecuación diferencial parcial (EDP)} lineal y de segundo orden, dependiente de dos variables independientes:
\begin{itemize}
\item $x$: posición horizontal de un punto en la cuerda.
\item $t$: tiempo.
\end{itemize}
A continuación, aplicaremos el \textbf{método de separación de variables} para obtener soluciones que satisfagan esta ecuación bajo las condiciones de borde e iniciales impuestas por el problema.

Supondremos una solución \textbf{completamente separable} de la forma:
\begin{equation}\label{1.2_solución_propuesta}
    u(x,t) = X(x)\,T(t)\,,
\end{equation}
donde $X(x)$ y $T(t)$ son funciones desconocidas, cada una dependiente de una sola variable.

Sustituyendo la solución propuesta \eqref{1.2_solución_propuesta} en la ecuación \eqref{1.1_ecuación_de_onda}, se obtiene:

\begin{equation*}
    \frac{\partial^2}{\partial t^2} \left(X(x)\,T(t)\right) = c^2\,\frac{\partial^2}{\partial x^2}\left(X(x)\,T(t)\right) \,.
\end{equation*}
Como $X(x)$ no depende del tiempo y $T(t)$ no depende de la posición, ambas funciones pueden tratarse como constantes al aplicar las derivadas parciales con respecto a la variable opuesta. En consecuencia, se obtiene:
\begin{equation}\label{1.2_ecuación_de_onda_y_solución_propuesta}
X(x)\,\frac{d^2 T}{dt^2}(t) = c^2\,T(t)\,\frac{d^2 X}{dx^2}(x)\,,
\end{equation}
donde las derivadas parciales se han reemplazado por derivadas ordinarias, ya que cada función depende exclusivamente de una sola variable.

Dividimos ambos miembros de la ecuación \eqref{1.2_ecuación_de_onda_y_solución_propuesta} por la función $u(x,t)$, dada por la expresión \eqref{1.2_solución_propuesta}. Se obtiene:

\begin{equation*}
    \frac{1}{X(x)\,T(t)}\left( X(x)\,\frac{d^2 T}{dt^2}(t) \right) = \frac{1}{X(x)\,T(t)}\left( c^2\,T(t)\,\frac{d^2 X}{dx^2}(x) \right).
\end{equation*}

Al simplificar, llegamos a una expresión en la que cada miembro depende exclusivamente de una de las variables:

\begin{equation}\label{1.2_ecuación_de_onda_y_solución_propuesta_arreglo_2}
    \frac{1}{T(t)}\,\frac{d^2 T}{dt^2}(t) = \frac{c^2}{X(x)}\,\frac{d^2 X}{dx^2}(x).
\end{equation}

Dado que el lado izquierdo de la ecuación \eqref{1.2_ecuación_de_onda_y_solución_propuesta_arreglo_2} depende únicamente de $t$, y el derecho únicamente de $x$, y la igualdad debe cumplirse para todo par $(x,t)$ en el dominio del problema, ambos miembros deben ser iguales a una \textbf{constante de separación}, que denotaremos por $\lambda \in \mathbb{R}$.

De este modo, la ecuación diferencial parcial original queda reducida al siguiente sistema de ecuaciones diferenciales ordinarias:

\begin{align*}
    \frac{1}{T(t)}\,\frac{d^2 T}{dt^2}(t) &= \lambda, \\
    \frac{c^2}{X(x)}\,\frac{d^2 X}{dx^2}(x) &= \lambda.
\end{align*}

Despejando las derivadas de segundo orden, se obtiene el sistema equivalente:
\begin{align}
    \frac{d^2 T}{dt^2}(t) - \lambda\,T(t) &= 0, \label{1.2_EDO_en_T} \\[4pt]
    \frac{d^2 X}{dx^2}(x) - \frac{\lambda}{c^2}\,X(x) &= 0. \label{1.2_EDO_en_X}
\end{align}

El tipo de soluciones de este sistema de EDOs dependerá del signo de la constante $\lambda$. Nuestro objetivo será identificar los valores de $\lambda$ que conduzcan a soluciones no triviales y físicamente significativas, es decir, que satisfagan las condiciones de borde del problema.

Expresamos dichas condiciones de borde, dadas originalmente en \eqref{1.1_condiciones_de_borde}, en términos de la solución propuesta \eqref{1.2_solución_propuesta}:

\begin{equation*}
\left\{
\begin{aligned}
    u(0,t) &= X(0)\,T(t)\, ,\\
    u(L,t) &= X(L)\,T(t) \,.
\end{aligned}
\right. \quad 
\Rightarrow \quad 
\left\{ 
\begin{aligned}
    X(0)\,T(t)&=0\, ,\\
    X(L)\,T(t) &=0\,.
\end{aligned}
\right.
\end{equation*}
Como estas condiciones deben cumplirse para todo $t > 0$, y buscamos soluciones no triviales en las que $T(t) \neq 0$, se concluye que las condiciones de borde recaen exclusivamente sobre la función $X(x)$:
\begin{equation}\label{1.2_condiciones_de_borde_X(x)}
\left\{
    \begin{aligned}
    X(0)&=0\, ,\\
    X(L) &=0\,.
\end{aligned}
\right.
\end{equation}

Estas condiciones imponen restricciones específicas sobre el comportamiento de $X(x)$ en los extremos del intervalo espacial $x \in [0, L]$.

Procedemos ahora a analizar las soluciones de la ecuación diferencial \eqref{1.2_EDO_en_X}, caso por caso, según el signo de la constante de separación $\lambda$.

\noindent\textbf{Caso 1: $\lambda = 0$}

En primer lugar, consideramos el caso $\lambda = 0$. La ecuación diferencial ordinaria \eqref{1.2_EDO_en_X} asociada a la variable espacial $x$ se reduce a:
\begin{equation*}
\frac{d^2 X}{dx^2}(x) = 0\,,
\end{equation*}
cuya solución general es una función lineal:
\begin{equation*}
X(x) = a_1 + a_2 x, \qquad a_1\,, a_2 \in \mathbb{R}\,.
\end{equation*}
Verificamos ahora si esta solución es compatible con las condiciones de borde \eqref{1.2_condiciones_de_borde_X(x)}:
\begin{equation*}
\left\{
\begin{aligned}
X(0) &= a_1, \\
X(L) &= a_1 + a_2 L.
\end{aligned}
\right.
\qquad \Rightarrow \qquad
\left\{
\begin{aligned}
a_1 &= 0, \\
a_2 &= 0.
\end{aligned}
\right.
\end{equation*}
Así, la única solución posible en este caso es $X(x) = 0$, lo que implica que la solución completa es $u(x,t) = X(x)T(t) = 0$ para todo $(x,t)$, es decir, la \textbf{solución trivial}.

Como esta no representa una oscilación física de la cuerda, la descartamos. Por tanto, el caso $\lambda = 0$ no contribuye a la solución general del problema.

%%%%%%%%%%%%%%
\noindent\textbf{Caso 2: $\lambda > 0$}

Supongamos ahora que $\lambda > 0$. En tal caso, es conveniente definir un parámetro auxiliar $\omega \in \mathbb{R}^+$ tal que $\lambda = \omega^2$. De este modo, la ecuación diferencial \eqref{1.2_EDO_en_X} se reescribe como:

\begin{equation}\label{1.2_edo_caso_lambda_positivo}
    \frac{d^2 X}{dx^2}(x) - \frac{\omega^2}{c^2}\,X(x) = 0.
\end{equation}

La solución general de \eqref{1.2_edo_caso_lambda_positivo} es una combinación lineal de exponenciales reales:
\begin{equation*}
    X(x) = a_1 e^{\frac{\omega}{c}x} + a_2 e^{-\frac{\omega}{c}x}, \quad a_1,\,a_2 \in \mathbb{R}.
\end{equation*}

Aplicamos ahora las condiciones de borde \eqref{1.2_condiciones_de_borde_X(x)}:
\begin{equation*}
\left\{
\begin{aligned}
X(0) &= a_1 + a_2\,, \\
X(L) &= a_1 e^{\frac{\omega}{c}L} + a_2 e^{-\frac{\omega}{c}L}\,.
\end{aligned}
\right.
\end{equation*}
De estas igualdades se deduce el sistema:
\begin{equation*}
\left\{
\begin{aligned}
    a_1 + a_2 &= 0 \,, \\
    a_1 e^{\frac{\omega}{c}L} + a_2 e^{-\frac{\omega}{c}L} &= 0 \,.
\end{aligned}
\right. 
\end{equation*}

De la primera ecuación se deduce que $a_2 = -a_1$. Sustituyendo en la segunda:
\begin{equation*}
a_1 \left( e^{\frac{\omega}{c}L} - e^{-\frac{\omega}{c}L} \right) = 0.
\end{equation*}

Como $e^{\frac{\omega}{c}L} - e^{-\frac{\omega}{c}L} \neq 0$ para $\omega > 0$, se concluye que $a_1 = 0$, y por tanto $a_2 = 0$.

En consecuencia, $X(x) = 0$ es la única solución posible, es decir, la \textbf{solución trivial}. Por lo tanto, descartamos el caso $\lambda > 0$.

\noindent\textbf{Caso 3: $\lambda < 0$}

Finalmente, consideramos el caso $\lambda < 0$. En este contexto, es útil introducir un parámetro auxiliar $\omega \in \mathbb{R}^+$ tal que $\lambda = -\omega^2$. De este modo, la ecuación diferencial \eqref{1.2_EDO_en_X} toma la forma:

\begin{equation}\label{1.2_edo_caso_lambda_negativo}
    \frac{d^2 X}{dx^2}(x) + \frac{\omega^2}{c^2}\,X(x) = 0.
\end{equation}

La solución general de \eqref{1.2_edo_caso_lambda_negativo} es una combinación lineal de funciones trigonométricas:
\begin{equation}\label{1.2_solución_generica_edo_caso_lambda_negativo}
    X(x) = a_1 \cos\left(\frac{\omega}{c}x\right) + a_2 \sin\left(\frac{\omega}{c}x\right), \quad a_1,\,a_2 \in \mathbb{R}.
\end{equation}

Aplicamos las condiciones de borde homogéneas \eqref{1.2_condiciones_de_borde_X(x)}:
\begin{equation*}
\left\{
\begin{aligned}
    X(0) &= a_1\,, \\
    X(L) &= a_1 \cos\left(\frac{\omega}{c}L\right) + a_2 \sin\left(\frac{\omega}{c}L\right).
\end{aligned}
\right.
\end{equation*}

La primera ecuación implica inmediatamente que $a_1 = 0$. Sustituyendo en la segunda:

\begin{equation*}
X(L) = a_2 \sin\left(\frac{\omega}{c}L\right) = 0.
\end{equation*}

Para obtener una solución no trivial ($a_2 \neq 0$), debe cumplirse:
\begin{equation*}
\sin\left(\frac{\omega}{c}L\right) = 0,
\end{equation*}
lo que ocurre si y sólo si
\begin{equation}\label{1.2_omega_seno_igual_a_cero}
\frac{\omega}{c}L = n\pi \quad \Leftrightarrow \quad \omega_n = \frac{n\pi c}{L}, \qquad n \in \mathbb{N}.
\end{equation}

Sustituyendo el valor de $\omega_n$ en la relación $\lambda = -\omega^2$, se obtiene el conjunto de valores permitidos para la constante de separación:
\begin{equation}\label{1.2_lambda_n}
    \lambda_n = -\left(\frac{n\pi c}{L}\right)^2, \qquad n \in \mathbb{N}.
\end{equation}

Dado que los únicos valores admisibles de $\lambda$ son los $\lambda_n$ definidos en \eqref{1.2_lambda_n}, para cada $n \in \mathbb{N}$ se obtiene una solución correspondiente a la variable espacial:
\begin{equation}\label{1.2_solución_función_auxiliar_en_x}
    X_n(x) = a_2 \sin\left(\frac{n\pi}{L}x\right)\,.
\end{equation}

Analizamos ahora la EDO \eqref{1.2_EDO_en_T} asociada a la variable temporal $t$. Sustituyendo en ella el valor de $\lambda_n$ para un $n \in \mathbb{N}$ arbitrario, se obtiene:
\begin{equation*}
\frac{d^2 T }{d t^2} \left(t\right) +\left(\frac{n\pi c}{L}\right)^2\; T(t) = 0\,. 
\end{equation*}

La solución general de la EDO anterior es una combinación lineal de funciones trigonométricas:
\begin{equation}\label{1.2_solución_función_auxiliar_en_t}
    T_n(t) = b_1 \cos\left(\frac{\pi n c} {L}t\right) + b_2 \sin\left(\frac{\pi n c}{L}t\right)\,.
\end{equation}

donde $b_1, \, b_2 \in \mathbb{R}$ son constantes arbitrarias.

Sustituyendo las expresiones de $X_n(x)$ y $T_n(t)$ dadas en \eqref{1.2_solución_función_auxiliar_en_x} y \eqref{1.2_solución_función_auxiliar_en_t} en la forma completamente separable propuesta \eqref{1.2_solución_propuesta}, se obtiene:
\begin{equation*}
\begin{aligned}
    u_n(x,t) &= a_2 \sin\left(\frac{\pi n}{L}x\right)\left[b_1 \cos\left(\frac{\pi n c} {L}t\right) + b_2 \sin\left(\frac{\pi n c}{L}t\right)\right] \\[5pt]
    &= \sin\left(\frac{\pi n}{L}x\right)\left[a_2 b_1 \cos\left(\frac{\pi n c} {L}t\right) + a_2 b_2 \sin\left(\frac{\pi n c}{L}t\right)\right]\,.
\end{aligned}
\end{equation*}

\clearpage

Agrupando constantes, definimos $A_n := a_2 b_1$ y $B_n := a_2 b_2$, lo que nos lleva a una solución particular para un valor fijo de $n$:

\begin{equation}\label{1.2_solución_particular}
    u_n(x,t) = \left[A_n \cos\left(\frac{\pi n c} {L}t\right) + B_n \sin\left(\frac{\pi n c}{L}t\right)\right] \sin\left(\frac{\pi n}{L}x\right).
\end{equation}

La solución general de la ecuación de onda se obtiene como superposición (suma infinita) de las soluciones particulares $u_n(x,t)$ para todos los valores $n \in \mathbb{N}$:
\begin{equation}\label{1.2_solución_general_implicita}
    u(x,t) = \sum_{n=1}^{\infty} u_n(x,t)\,.
\end{equation}
Sustituyendo la expresión obtenida en \eqref{1.2_solución_particular} en \eqref{1.2_solución_general_implicita}, se obtiene una forma explícita para la solución general:

\begin{equation}\label{1.2_solución_general_explcita}
    u(x,t) = \sum_{n=1}^{\infty} \left[A_n \cos\left(\frac{\pi n c}{L}t\right) + B_n \sin\left(\frac{\pi n c}{L}t\right)\right] \sin\left(\frac{\pi n}{L}x\right)\,.
\end{equation}

Las condiciones iniciales impuestas al sistema son:
\begin{align}
    u(x,0) &=
    \begin{cases}
        \displaystyle \frac{2Ax}{L}\,, & 0 \leq x \leq \frac{L}{2}\,,\\[8pt]
        \displaystyle \frac{2A(L - x)}{L}\,, & \frac{L}{2} \leq x \leq L\,,
    \end{cases}\label{1.2_condiciones_iniciales_u}  \\[10pt]
    \frac{\partial u}{\partial t}(x,0) &= 0\,.\label{1.2_condiciones_iniciales_u'} 
\end{align}

Evaluamos la solución general \eqref{1.2_solución_general_explcita} en $t = 0$:
    \begin{equation}\label{1.2_solución_con_t=0}
        u(x,0) = \sum_{n=1}^{\infty} A_n \sin\left(\frac{n\pi}{L}x\right)\,.
    \end{equation}

La expresión obtenida en \eqref{1.2_solución_con_t=0} corresponde a una serie de Fourier en senos, la cual representa funciones impares periódicas de periodo $2L$.

Para poder representar $u(x,0)$ mediante una serie de senos en el intervalo $[-L, L]$, extendemos la función definida originalmente en $[0,L]$ a todo el intervalo $[-L,L]$ como una función impar. Denotamos esta extensión por $O_u(x)$, definida como:
\begin{equation}\label{1.2_extension_impar_por_definición}
     O_u(x) = \left\{ \begin{array}{lcc} -u(-x,0) & si & -L \leq x < 0 \\ u(x,0) & si & 0 \leq x \leq L \end{array} \right.
\end{equation}

Sustituyendo en la definición anterior la condición inicial \eqref{1.2_condiciones_iniciales_u}, se obtiene la expresión explícita de $O_u(x)$ por tramos:
\begin{equation}\label{1.2_extensión_final}
O_u(x) = \begin{cases}
\displaystyle -\frac{2A}{L}(L + x)\,, & -L \leq x \leq -\frac{L}{2}\,,\\[8pt]
\displaystyle \frac{2A}{L}x\,, & -\frac{L}{2} < x \leq \frac{L}{2}\,, \\[8pt]
\displaystyle \frac{2A}{L}(L - x)\,, & \frac{L}{2} < x \leq L\,.
\end{cases}
\end{equation}

De este modo, la función $O_u(x)$ puede representarse como una serie de Fourier en senos:

\begin{equation}\label{1.2_definición_función_f}
    O_u(x) = \sum_{n=1}^{\infty} A_n \sin\left(\frac{n\pi}{L}x\right)\,.
\end{equation}

Dado que $O_u(x)$ es impar y de periodo $2L$, sus coeficientes de Fourier correspondientes a la base ortonormal $\{\sin(\frac{n\pi x}{L})\}_{n=1}^\infty$ en $[-L,L]$ están dados por:
    \begin{equation*}
        A_n = \frac{1}{L} \int_{-L}^{L} O_u(x) \sin\left( \frac{n\pi}{L} x \right) \, dx\,.
    \end{equation*}

Para su cálculo explícito, descomponemos la integral según los tramos en que está definida la función $O_u(x)$:
\begin{equation*}
\begin{aligned}
    A_n &= \frac{1}{L} \int_{-L}^{L} O_u(x) \sin\left( \frac{n\pi}{L} x \right) \, dx \\
        &= \frac{1}{L} \left[
            \int_{-L}^{-\frac{L}{2}} -\frac{2A(L +x)}{L}  \sin\left( \frac{n\pi}{L} x \right) \, dx
            + \int_{-\frac{L}{2}}^{\frac{L}{2}} \frac{2A}{L}x \sin\left( \frac{n\pi}{L} x \right) \, dx
            + \int_{\frac{L}{2}}^{L} \frac{2A}{L}(L - x) \sin\left( \frac{n\pi}{L} x \right) \, dx
        \right] \\
        &= \frac{1}{L} \left[I_1+I_2+I_3
        \right] \\
\end{aligned}
\end{equation*}
Definimos entonces:
\begin{align}
    I_1&:=\int_{-L}^{-\frac{L}{2}} -\frac{2A(L +x)}{L}  \sin\left( \frac{n\pi}{L} x \right) \, dx \,, \label{1.2_I_1}\\
    I_2&:= \int_{-\frac{L}{2}}^{\frac{L}{2}} \frac{2A}{L}x \sin\left( \frac{n\pi}{L} x \right) \, dx\,, \label{1.2_I_2}\\
    I_3&:=\int_{\frac{L}{2}}^{L} \frac{2A}{L}(L - x) \sin\left( \frac{n\pi}{L} x \right) \, dx \,.\label{1.2_I_3}
\end{align}
Con esto, el coeficiente $A_n$ se reescribe como:
\begin{equation}\label{1.2_A_n_i}
A_n = \frac{1}{L} \left[I_1 + I_2 + I_3\right]\,.
\end{equation}
Aplicamos en la integral \eqref{1.2_I_1} el cambio de variable $x = -x$, con lo cual $dx = -dx$:
\begin{equation*}
\begin{aligned}
     I_1&=\int_{L}^{\frac{L}{2}} -\frac{2A(L -x)}{L}  \sin\left( \frac{n\pi}{L} (-x) \right) \, (-dx) \\
     &=\int_{L}^{\frac{L}{2}} \frac{2A(L -x)}{L}  \sin\left( -\frac{n\pi}{L}x \right) \, dx \\
     &=\int_{L}^{\frac{L}{2}} \frac{2A(L -x)}{L} \left[- \sin\left(\frac{n\pi}{L}x \right)\right] \, dx \\
     &=\int_{\frac{L}{2}}^{L} \frac{2A(L -x)}{L}  \sin\left(\frac{n\pi}{L}x \right) \, dx \\
     &=I_3
\end{aligned}    
\end{equation*}

Por lo tanto, $I_1 = I_3$. Sustituyendo en \eqref{1.2_A_n_i}, se concluye que $I_1 + I_3 = 2I_3$, y así:
\begin{equation}\label{1.2_A_n_2I_3}
A_n = \frac{1}{L} \left[ I_2 + 2I_3 \right]\,.
\end{equation}

Observamos que la función integrando en $I_2$ es par, ya que:
\begin{equation*}
    \begin{aligned}
        \frac{2A}{L}(-x) \sin\left( \frac{n\pi}{L} (-x) \right)&=-\frac{2A}{L}x \sin\left( -\frac{n\pi}{L}x \right) \\
        &=-\frac{2A}{L}x \left[-\sin\left(\frac{n\pi}{L}x \right)\right] \\
        &=\frac{2A}{L}x \sin\left(\frac{n\pi}{L}x \right)
    \end{aligned}
\end{equation*}
Entonces, por la paridad del integrando y el hecho de que el dominio de integración es simétrico respecto del origen, podemos reescribir $I_2$ como:
\begin{equation}\label{1.2_I_2_par}
   I_2= 2\int_{0}^{\frac{L}{2}} \frac{2A}{L}x \sin\left( \frac{n\pi}{L} x \right) \, dx\,.
\end{equation}

Sustituyendo las expresiones de $I_2$ y $I_3$, dadas en \eqref{1.2_I_2_par} y \eqref{1.2_I_3}, en la fórmula \eqref{1.2_A_n_2I_3}, obtenemos la siguiente expresión para el coeficiente $A_n$:
\begin{equation*}
       A_n=\frac{2}{L} \left[\int_{0}^{\frac{L}{2}} \frac{2A}{L}x \sin\left( \frac{n\pi}{L} x \right) \, dx+\int_{\frac{L}{2}}^{L} \frac{2A}{L}(L - x) \sin\left( \frac{n\pi}{L} x \right) \, dx \right] \,
\end{equation*}

\clearpage

Evaluando las integrales en la expresión anterior, el coeficiente $A_n$ toma la forma explícita:

\begin{equation}\label{1.2_A_n}
    A_n = \frac{16A}{n^2 \pi^2} \, \sin^2\left( \frac{n\pi}{4} \right) \sin\left( \frac{n\pi}{2} \right)\,.
\end{equation}

Observemos que el factor $\sin\left( \frac{n\pi}{2} \right)$ introduce una \textbf{condición de selección} sobre los valores de $n$, ya que su valor es cero para todo $n$ par, lo que implica que $A_n = 0$ cuando $n$ es par. En consecuencia, restringimos el análisis a los valores con $n$ impar, es decir, $n = 2m + 1$ para $m \in \mathbb{N}_0$. Notemos ademas que:

\begin{equation*}
\begin{aligned}
     \sin\left( \frac{n\pi}{2} \right) &=\sin\left( \frac{(1+2m)\pi}{2} \right)\\  
     &=\sin\left( \frac{\pi}{2}+m\pi \right) \\
     &=\left\{
     \begin{aligned}
     1 \quad &\text{si}\;m\;\text{es par} \\
     -1 \quad &\text{si}\;m\;\text{es impar} 
\end{aligned}
\right. \\
&=(-1)^m \, .
\end{aligned}
\end{equation*}

Si consideramos $n=1+2m$, entonces tendremos $m=\frac{n-1}{2}$, es decir:

\begin{equation}\label{1.2_seno_simplificado}
    \sin\left( \frac{n\pi}{2} \right)=(-1)^{\frac{n-1}{2}} \, .
\end{equation}

Por otra parte, para todo $n$ impar, se tiene:
\begin{equation}\label{1.2_seno_impar}
    \sin^2\left( \frac{n\pi}{4} \right)=\frac{1}{2}
\end{equation}
Esto puede verificarse utilizando identidades reconociendo que los ángulos involucrados son múltiplos impares de $\frac{\pi}{4}$, cuyos cuadrados del seno siempre valen $\frac{1}{2}$.

Sustituyendo \eqref{1.2_seno_simplificado} y \eqref{1.2_seno_impar} en \eqref{1.2_A_n}, el coeficiente de Fourier $A_n$ se reescribe como:

\begin{equation}\label{1.2_A_m}
    A_n = \frac{8A\,\left( -1 \right)^{\frac{n-1}{2}}}{n^2 \pi^2}\,.
\end{equation}

Para determinar el coeficiente $B_n$ en \eqref{1.2_solución_general_explcita}, derivamos parcialmente la solución general respecto al tiempo $t$, obteniendo:

\begin{equation*}
\begin{aligned}
    \frac{\partial u}{\partial t}(x,t) &= \frac{\partial }{\partial t} \left\{ \sum_{n=1}^{\infty} \left[ A_n \cos\left( \frac{n\pi c}{L} t \right) + B_n \sin\left( \frac{n\pi c}{L} t \right) \right] \sin\left( \frac{n\pi}{L} x \right) \right\} \\
    &= \sum_{n=1}^{\infty} \sin\left( \frac{n\pi}{L} x \right) \frac{d}{dt} \left[ A_n \cos\left( \frac{n\pi c}{L} t \right) + B_n \sin\left( \frac{n\pi c}{L} t \right) \right] \\
    &= \sum_{n=1}^{\infty} \sin\left( \frac{n\pi}{L} x \right) \left[ \frac{B_n\,n\pi c}{L} \cos\left( \frac{n\pi c}{L} t \right) - \frac{A_n\,n\pi c}{L} \sin\left( \frac{n\pi c}{L} t \right) \right]\,.
\end{aligned}
\end{equation*}

De este modo, obtenemos la expresión general para la derivada temporal:

\begin{equation}\label{1.2_derivada_de_la_solución}
    \frac{\partial u}{\partial t}(x,t) = \sum_{n=1}^{\infty} \sin\left( \frac{n\pi}{L} x \right) \left[ \frac{B_n\,n\pi c}{L} \cos\left( \frac{n\pi c}{L} t \right) - \frac{A_n\,n\pi c}{L} \sin\left( \frac{n\pi c}{L} t \right) \right]\,.
\end{equation}

Evaluamos esta expresión en $t = 0$:

\begin{equation*}
\begin{aligned}
    \frac{\partial u}{\partial t}(x,0) &= \sum_{n=1}^{\infty} \sin\left( \frac{n\pi}{L} x \right) \left[ \frac{B_n\,n\pi c}{L} \cos(0) - \frac{A_n\,n\pi c}{L} \sin(0) \right] \\
    &= \sum_{n=1}^{\infty} \frac{B_n\,n\pi c}{L} \sin\left( \frac{n\pi}{L} x \right)\,.
\end{aligned}
\end{equation*}

Finalmente, al imponer la condición inicial \eqref{1.2_condiciones_iniciales_u'} , se obtiene:

\begin{equation}\label{1.2_suma_B_n}
    0 = \sum_{n=1}^{\infty} \frac{B_n\,n\pi c}{L} \sin\left( \frac{n\pi}{L} x \right)\,.
\end{equation}

Como $\left\{ \sin\left( \frac{n\pi}{L} x \right) \right\}_{n=1}^\infty$ forma una base ortonormal completa en el subespacio de funciones impares de $L^2([-L, L])$, y dado que hemos extendido $u(x,0)$ como una función impar en ese dominio, la única forma en que la igualdad \eqref{1.2_suma_B_n} se verifique para todo $x \in [-L, L]$ es que todos los coeficientes de la base se anulen, es decir, $B_n = 0$ para todo $n$.
\begin{equation}\label{1.2_B_m}
    B_n = 0.
\end{equation}

Sustituyendoe n \eqref{1.2_solución_general_explcita} los valores de los coeficientes $A_n$ y $B_n$, dados por \eqref{1.2_A_m} y \eqref{1.2_B_m} respectivamente, obtenemos finalmente la solución explícita de la ecuación de onda bajo las condiciones iniciales:

\begin{equation}\label{1.2_condiciones_inciales_solución_general_explcita}
    u(x,t) = \sum_{\substack{n=1 \\ n \,\text{es impar}}}^{\infty} \frac{8A\,(-1)^{\frac{n-1}{2}}}{n^2 \pi^2} \cos\left( \frac{n\pi c}{L} t \right) \sin\left( \frac{n\pi}{L} x \right)\,.
\end{equation}

\clearpage
%%%%%%%%%%%%%%%%%%%%%%%%%%%%%%%%%%%%%%%%%%%%%%%%%%%%%%%%%%%%%%%%%%%%%%%%%%%%%%%%%%%%%%%%%%%%%%%%%%%%%%%%%%%%%%%%%%%%%%%%%%%%%%%%%%%%%%%%%%%%%%%%%%%%%%%%%%%%%%%%%%%%%%%%%%%%%%%%%%%%%%%%%%%%%%%%%%%%%%%
\subsection{Energía total de las oscilaciones}

La energía cinética del $(2n + 1)$-ésimo modo normal puede obtenerse como

\begin{equation}\label{1.3_energia_cinetica}
E_{2n+1} = \int_0^L \frac{1}{2} \rho\,\left( \dot{u}_{2n+1} \right)^2 dx\,,
\end{equation}

donde $\dot{u}_{2n+1}$ corresponde a la derivada temporal del $(2n+1)$-ésimo término de la solución hallada en el apartado anterior, \textbf{cuando la función temporal alcanza su valor máximo}, y $\rho$ corresponde a la densidad de masa de la cuerda. Con esto en mente, encuentre la energía total de las oscilaciones de la cuerda.

\textbf{Solución:} 

%%%%%%%%%%%%%%%%%%%%%%%%%%%%%%%%%%%%%%%%%%%%%%%%%%%%%%%%%%%%%%%%%%%%%%%%%%%%%%%%%%%%%%%%%%%%%%%%%%%%%%%%%%%%%%%%%%%%%%%%%%%%%%%%%%%%%%%%%%%%%%%%%%%%%%%%%%%%%%%%%%%%%%%%%%%%%%%%%%%%%%%%%%%%%%%%%%%%%%%

Notemos que el término $(2n+1)$-ésimo de la solución obtenida en \eqref{1.2_condiciones_inciales_solución_general_explcita} está dado por:

\begin{equation}\label{1.3_(2n+1)-esimo_termino}
    u_{2n+1}(x,t)=\frac{8A\,(-1)^n}{(2n+1)^2 \pi^2} \cos\left( \frac{(2n+1)\pi c}{L} t \right) \sin\left( \frac{(2n+1)\pi}{L} x \right) \, .
\end{equation}

La derivada temporal de \eqref{1.3_(2n+1)-esimo_termino} se expresa como:

\begin{equation}\label{1.3_derivada_1}
    \dot{u}_{2n+1}(x,t) = \frac{\partial}{\partial t}  \left[ \frac{8A\,(-1)^{n}}{(2n+1)^2 \pi^2} \,  \cos\left( \frac{(2n+1)\pi c}{L} t \right) \sin\left( \frac{(2n+1)\pi}{L} x \right)\right] \,,
\end{equation}

Observamos que los siguientes factores no dependen del tiempo:

\begin{equation*}
    \frac{8A\,(-1)^n}{(2n+1)^2 \pi^2} \quad \text{y} \quad \sin\left( \frac{(2n+1)\pi}{L} x \right) \,.
\end{equation*}

Por tanto, al calcular la derivada en \eqref{1.3_derivada_1}, estos pueden considerarse constantes, y se obtiene:

\begin{equation}\label{1.3_derivada_2}
    \dot{u}_{2n+1}(x,t)  = \frac{8A\,(-1)^n}{(2n+1)^2 \pi^2} \ \frac{\partial}{\partial t} \left[\cos\left( \frac{(2n+1)\pi c}{L} t \right) \right]\sin\left( \frac{(2n+1)\pi}{L} x \right) \, ,
\end{equation}

donde la derivada de la función temporal respecto al tiempo es:

\begin{equation}\label{1.3_derivada_función_temporal}
    \frac{\partial}{\partial t} \left[\cos\left( \frac{(2n+1)\pi c}{L} t \right) \right]=-\frac{(2n+1)\pi c}{L} \,\sin\left( \frac{(2n+1)\pi c}{L} t \right) \,.
\end{equation}

Si suponemos que \textbf{la función temporal alcanza su valor máximo}, esto es,

\begin{equation*}
    \sin\left( \frac{(2n+1)\pi c}{L} t \right)= 1 \,,
\end{equation*}

entonces la derivada \eqref{1.3_derivada_función_temporal} se simplifica a:

\begin{equation}\label{1.3_derivada_función_temporal_maximo}
    \frac{\partial}{\partial t} \left[\cos\left( \frac{(2n+1)\pi c}{L} t \right) \right]=-\frac{(2n+1)\pi c}{L} \,.
\end{equation}

Sustituyendo \eqref{1.3_derivada_función_temporal_maximo} en \eqref{1.3_derivada_2} y simplificando, se obtiene:

\begin{equation}\label{1.3_derivada_temporal_(2n+1)-esimo_termino}
    \dot{u}_{2n+1}(x,t) =-\frac{8A\,c\,(-1)^n}{(2n+1) \pi \, L} \sin\left( \frac{(2n+1)\pi}{L} x \right) \, .    
\end{equation}

Procedamos ahora a calcular la \textbf{energía cinética asociada al modo normal $(2n+1)$}. Al sustituir la expresión obtenida en \eqref{1.3_derivada_temporal_(2n+1)-esimo_termino} dentro de la definición de energía cinética dada por \eqref{1.3_energia_cinetica}, se obtiene:

\begin{equation}\label{1.3_energia_cinetica_(2n+1)}
   E_{2n+1}= \int_0^L \frac{1}{2} \rho\,\left[ -\frac{8A\,c\,(-1)^n}{(2n+1) \pi \, L}  \, \sin\left( \frac{(2n+1)\pi}{L} x \right) \right]^2 dx \,.
\end{equation}

Desarrollando la integral en \eqref{1.3_energia_cinetica_(2n+1)}, se tiene:

\begin{equation*}
\begin{aligned}
    E_{2n+1} 
    &=\frac{1}{2} \rho \int_0^L \left[ -\frac{8A\,c\,(-1)^n}{(2n+1) \pi \, L}  \, \sin\left( \frac{(2n+1)\pi}{L} x \right) \right]^2 dx \\
    &=\frac{1}{2} \rho \int_0^L \frac{64A^2\,c^2\,(-1)^{2n}}{(2n+1)^2 \pi^2 \, L^2}  \, \sin^2\left( \frac{(2n+1)\pi}{L} x \right) dx \\
    &=\frac{32A^2\,c^2\,\rho}{(2n+1)^2 \pi^2 \, L^2} \int_0^L \sin^2\left( \frac{(2n+1)\pi}{L} x \right) dx \,.
\end{aligned}
\end{equation*}

Por tanto, la energía del modo impar queda expresada como:

\begin{equation}\label{1.3_energia_cinetica_(2n+1)_integral}
    E_{2n+1} =\frac{32A^2\,c^2\,\rho}{(2n+1)^2 \pi^2 \, L^2} \int_0^L \sin^2\left( \frac{(2n+1)\pi}{L} x \right) dx \,.
\end{equation}

Donde la integral en \eqref{1.3_energia_cinetica_(2n+1)_integral} es dada por:

\begin{equation}\label{1.3_integral_energia_resuelta}
    \int_0^L \sin^2\left( \frac{(2n+1)\pi}{L} x \right) dx = \frac{1}{2} L \,.
\end{equation}

Sustituyendo \eqref{1.3_integral_energia_resuelta} en \eqref{1.3_energia_cinetica_(2n+1)_integral} y simplificando, la energía cinética asociada al modo $(2n+1)$ queda dada por:

\begin{equation}\label{1.3_energia_(2n+1)}
    E_{2n+1} = \frac{16A^2\,c^2\,\rho}{(2n+1)^2 \pi^2 \, L} \,.
\end{equation}

A continuación, determinamos la energía total $E$ del sistema considerando la superposición de todos los modos normales impares. Esta se expresa como:

\begin{equation}\label{1.3_energia_total_indexado}
     E = \sum_{n=0}^\infty E_{2n+1} \,.
\end{equation}

Sustituyendo \eqref{1.3_energia_(2n+1)} en la suma en \eqref{1.3_energia_total_indexado}, obtenemos:
\begin{equation*}
    E = \sum_{n=0}^\infty \frac{16A^2\,c^2\,\rho}{(2n+1)^2 \pi^2 \, L} \,.
\end{equation*}

Extrayendo el factor constante:
\begin{equation}\label{1.3_energia_serie}
    E = \frac{16A^2\,c^2\,\rho}{\pi^2 \, L} \sum_{n=0}^\infty \frac{1}{(2n+1)^2} \,.
\end{equation}

Sabemos que la serie presente en \eqref{1.3_energia_serie} converge a:
\begin{equation}\label{1.3_serie}
    \sum_{n=0}^\infty \frac{1}{(2n+1)^2} = \frac{\pi^2}{8} \,.
\end{equation}

Finalmente, al reemplazar \eqref{1.3_serie} en \eqref{1.3_energia_serie}, se obtiene la energía total del sistema:
\begin{equation}\label{1.3_energia_total}
     E = \frac{2A^2\,c^2\,\rho}{\, L} \,.
\end{equation}

La energía total es proporcional a $A^2$ e inversamente proporcional a $L$, lo que refleja que una mayor deformación inicial implica más energía almacenada, y que cuerdas más largas distribuyen esa energía en una mayor extensión.

 
\clearpage
%%%%%%%%%%%%%%%%%%%%%%%%%%%%%%%%%%%%%%%%%%%%%%%%%%%%%%%%%%%%%%%%%%%%%%%%%%%%%%%%%%%%%%%%%%%%%%%%%%%%%%%%%%%%%%%%%%%%%%%%%%%%%%%%%%%%%%%%%%%%%%%%%%%%%%%%%%%%%%%%%%%%%%%%%%%%%%%%%%%%%%%%%%%%%%%%%%%%%%%
\section{Ejercicio 2}

Considere un cascarón esférico, de radio $R$ y ancho despreciable, que conduce calor en su superficie. Para esto, consideramos la ecuación de difusión del calor sobre la esfera, es decir, para $r = R$ constante,
\begin{equation}\label{2.1_ecuación_de_calor_ejercicio} 
\frac{\partial \Psi}{\partial t} = a^2 \nabla^2 \Psi(R, \theta, \phi, t)\,,
\end{equation}
donde $a$ es una constante positiva cuyo valor depende del material con el que esté construido el cascarón.

Considere el caso particular en que todo el problema (es decir, el dominio, las condiciones iniciales y la solución) tiene simetría axial, de modo que no existe dependencia respecto a la variable $\phi$. En particular, suponga que $\Psi = \Psi(\theta, t)$. Considere además la condición inicial
\begin{equation*}
\Psi(\theta, 0) = f(\theta)\,,
\end{equation*}
donde $f(\theta)$ es una función conocida, finita en todo punto de la esfera.

%%%%%%%%%%%%%%%%%%%%%%%%%%%%%%%%%%%%%%%%%%%%%%%%%%%%%%%%%%%%%%%%%%%%%%%%%%%%%%%%%%%%%%%%%%%%%%%%%%%%%%%%%%%%%%%%%%%%%%%%%%%%%%%%%%%%%%%%%%%%%%%%%%%%%%%%%%%%%%%%%%%%%%%%%%%%%%%%%%%%%%%%%%%%%%%%%%%%%%%
\subsection{Solución general mediante el método de separación de variables}\label{sec:2.1}

Usando el Método de Separación de Variables, encuentre una expresión para la solución $\Psi(\theta,t)$, escrita en términos de $a$, $r$, $n$ y $f$.
 
\textbf{Solución:} 

%%%%%%%%%%%%%%%%%%%%%%%%%%%%%%%%%%%%%%%%%%%%%%%%%%%%%%%%%%%%%%%%%%%%%%%%%%%%%%%%%%%%%%%%%%%%%%%%%%%%%%%%%%%%%%%%%%%%%%%%%%%%%%%%%%%%%%%%%%%%%%%%%%%%%%%%%%%%%%%%%%%%%%%%%%%%%%%%%%%%%%%%%%%%%%%%%%%%%%%

Notemos que el sistema posee simetría esférica, con coordenadas $(R, \theta, \phi)$. Se nos indica que el problema presenta simetría axial, lo cual implica que no hay dependencia respecto de la coordenada $\phi$. Además, como se trabaja sobre un cascarón esférico, la coordenada radial permanece constante, es decir, $r = R$. 

En consecuencia, el \textbf{operador Laplaciano} en coordenadas esféricas, cuya forma general es:

\begin{equation*}
    \nabla^2 = \frac{1}{r^2}  \frac{\partial}{\partial r} \left( r^2 \frac{\partial}{\partial r} \right) 
    + \frac{1}{r^2 \sin\theta}\frac{\partial}{\partial \theta} \left( \sin\theta \frac{\partial}{\partial \theta} \right) 
    + \frac{1}{r^2\sin^2\theta} \frac{\partial^2}{\partial \phi^2} \,,
\end{equation*}

se reduce a la expresión:

\begin{equation}\label{2.1_laplace}
    \nabla^2 = \frac{1}{r^2 \sin\theta}  \frac{\partial}{\partial \theta} \left( \sin\theta \frac{\partial}{\partial \theta} \right) \,,
\end{equation}

ya que no hay dependencia en $r$ ni en $\phi$. 

Así, la ecuación de conducción del calor sobre la superficie, dada por \eqref{2.1_ecuación_de_calor_ejercicio}, queda reducida a una dependencia en $\theta$ y en el tiempo. Al sustituir \eqref{2.1_laplace} en \eqref{2.1_ecuación_de_calor_ejercicio}, y considerando que $\Psi = \Psi(\theta, t)$, obtenemos la siguiente forma de la ecuación diferencial parcial:

\begin{equation}\label{2.1_ecuación_de_calor_consideración} 
    \frac{\partial \Psi}{\partial t} = \frac{a^2}{r^2 \sin\theta}  \frac{\partial}{\partial \theta} \left( \sin\theta \frac{\partial \Psi}{\partial \theta} \right) \,.
\end{equation}

A continuación, deducimos la solución general de \eqref{2.1_ecuación_de_calor_consideración} mediante el método de separación de variables. Observamos que la ecuación diferencial parcial es lineal y depende de dos variables independientes:
    \begin{itemize} 
        \item $\theta$: coordenada colatitudinal.
        \item $t$: tiempo.
    \end{itemize}
    
    Proponemos una solución \textbf{completamente separable} de la forma:
    \begin{equation}\label{2.1_solución_propuesta}
        \Psi(\theta,t) = \Theta(\theta) \, T(t),
    \end{equation}
    donde $\Theta$ y $T$ son funciones auxiliares que dependen exclusivamente de $\theta$ y $t$, respectivamente.

    Sustituimos la solución propuesta \eqref{2.1_solución_propuesta} en la ecuación \eqref{2.1_ecuación_de_calor_consideración}:
    \begin{equation*}
    \begin{aligned}
        \frac{\partial}{\partial t} \left( \Theta T \right) &= \frac{a^2}{r^2 \sin\theta} \frac{\partial}{\partial \theta} \left( \sin\theta \frac{\partial}{\partial \theta} \left( \Theta T \right) \right) \\[5pt]
        \Rightarrow \quad \Theta\frac{d T}{d t} &= \frac{a^2 T}{r^2 \sin\theta} \frac{d}{d \theta} \left( \sin\theta \frac{d \Theta}{d \theta}\right) \\[5pt]
        &= \frac{a^2 T}{r^2 \sin\theta} \left( \cos\theta \frac{d \Theta}{d \theta} + \sin\theta \frac{d^2 \Theta}{d \theta^2} \right) \\[5pt]
        &= \frac{a^2 T}{r^2} \left( \frac{1}{\tan\theta} \frac{d \Theta}{d \theta}+ \frac{d^2 \Theta}{d \theta^2} \right).
    \end{aligned}
    \end{equation*}

    Reescribimos la igualdad obtenida como:
    \begin{equation}\label{2.1_ecuación_de_calor_y_solución_propuesta}
        \Theta\frac{d T}{d t}= \frac{a^2 T(t)}{r^2} \left( \frac{1}{\tan\theta} \frac{d \Theta}{d \theta} + \frac{d^2 \Theta}{d \theta^2}\right)\,.
    \end{equation}

    Dividimos ambos miembros de la ecuación \eqref{2.1_ecuación_de_calor_y_solución_propuesta} por la solución separable \eqref{2.1_solución_propuesta}:
    \begin{equation}\label{2.1_ecuación_de_calor_y_solución_propuesta_arreglo_1}
        \frac{1}{T} \frac{d T}{d t} = \frac{1}{\Theta} \frac{a^2}{r^2} \left( \frac{1}{\tan\theta} \frac{d \Theta}{d \theta} + \frac{d^2 \Theta}{d \theta^2} \right)\,.
    \end{equation}

    Como los lados de la igualdad \eqref{2.1_ecuación_de_calor_y_solución_propuesta_arreglo_1} dependen de variables distintas, ambos deben ser iguales a una \textbf{constante de separación} $\lambda \in \mathbb{R}$. De este modo, obtenemos:
    \begin{equation}\label{2.1_constante_de_separación_1}
        \frac{1}{T} \frac{d T}{d t} = \lambda = \frac{1}{\Theta} \frac{a^2}{r^2} \left( \frac{1}{\tan\theta} \frac{d \Theta}{d \theta} + \frac{d^2 \Theta}{d \theta^2} \right)\,.
    \end{equation}

    Restando $\lambda$ a ambos lados de la igualdad, obtenemos el siguiente sistema:
    \begin{equation}\label{2.1_constante_de_separación_2}
        \frac{1}{T} \frac{d T}{d t} - \lambda = \frac{a^2}{r^2 \Theta} \left( \frac{1}{\tan\theta} \frac{d \Theta}{d \theta} + \frac{d^2 \Theta}{d \theta^2} \right) - \lambda = 0\,.
    \end{equation}

Así, la ecuación de separación nos conduce al siguiente sistema de ecuaciones diferenciales:

\begin{equation*}
\left\{
    \begin{aligned}
        \frac{1}{T}\frac{d T}{d t} - \lambda &= 0, \\
        \frac{a^2}{r^2 \, \Theta} \left( \frac{1}{\tan\theta} \frac{d \Theta}{d \theta} + \frac{d^2 \Theta}{d \theta^2} \right) - \lambda &= 0\, .
    \end{aligned}
\right.
\end{equation*}

Despejamos las derivadas de mayor orden presentes en las ecuaciones de este sistema lineal para finalmente obtener:
\begin{equation}\label{2.1_sistema_de_ecuaciones_2}
    \left\{
        \begin{aligned}  
        \frac{d T}{d t} - \lambda \,T&= 0, \\
        \frac{d^2 \Theta}{d \theta^2}+\frac{1}{\tan\theta} \frac{d \Theta}{d \theta} -  \frac{\lambda r^2  }{a^2} \Theta &= 0 \,.
        \end{aligned}
    \right. 
\end{equation}

De este modo, hemos reducido la EDP original \eqref{2.1_ecuación_de_calor_consideración} a un sistema formado por un par de EDOs, cada una en función de una única variable. Encontramos la solución general de cada EDO del sistema \eqref{2.1_sistema_de_ecuaciones_2}.

Para la EDO lineal de primer orden en dependencia de $T(t)$, su solución general sencillamente es

\begin{equation}\label{2.1_solucion_edo_1er_orden}
    T(t)=A_1\exp(\lambda t)\,,
\end{equation}

donde $A_1 \in \mathbb{R}$ una constante arbitra.
    
Para la EDO lineal de segundo orden dependiente de la función auxiliar $\Theta(\theta)$, su solución general es algo más compleja de deducir. Consideremos el siguiente cambio de variable: \label{item:cambio-variable-theta}
\begin{equation}\label{2.1_cambio_de_variable}
    x = \cos\theta\,,
\end{equation}
de modo que la función auxiliar $\Theta$, en función de $\theta$, será igual a una nueva función $\tilde{\Theta}$ en función de $x$, es decir:
\begin{equation}\label{2.1_funciones_auxiliares}
    \Theta(\theta) = \tilde{\Theta}(x)\,.
\end{equation}

Nótese que el cambio de variable definido en \eqref{2.1_cambio_de_variable} impone una restricción sobre el nuevo dominio de la variable $x$, dado que el codominio de la función coseno es el intervalo cerrado $\left[-1,1\right]$. Así, para cualquier $\theta\in\mathbb{R}$, se tiene que $x\in\left[-1,1\right]$.

Calcularemos ahora la primera y segunda derivada de $\Theta$ respecto a $\theta$.

\begin{itemize}
    \item Para la primera derivada de $\Theta$ respecto a $\theta$, aplicando la regla de la cadena sobre la igualdad \eqref{2.1_funciones_auxiliares}, se tiene:
    \begin{equation}\label{2.1_primera_derivada_regla_de_la_cadena}
        \frac{d\Theta}{d\theta} = \frac{d\tilde{\Theta}}{dx} \frac{dx}{d\theta}\,.
    \end{equation}
    Considerando el cambio de variable \eqref{2.1_cambio_de_variable}, obtenemos:
    \begin{equation*}
        x = \cos\theta \quad \Rightarrow \quad \frac{dx}{d\theta} = \frac{d}{d\theta}(\cos\theta)\,,
    \end{equation*}
    es decir,
    \begin{equation}\label{2.1_derivada_cambio_de_variable}
        \frac{dx}{d\theta} = -\sin\theta\,.
    \end{equation}
    Sustituyendo \eqref{2.1_derivada_cambio_de_variable} en \eqref{2.1_primera_derivada_regla_de_la_cadena}, resulta:
    \begin{equation}\label{2.1_primera_derivada}
        \frac{d\Theta}{d\theta} = -\sin\theta \frac{d\tilde{\Theta}}{dx}\,.
    \end{equation}

    \item Para la segunda derivada de $\Theta$ respecto a $\theta$, a partir de los resultados \eqref{2.1_derivada_cambio_de_variable} y \eqref{2.1_primera_derivada}, se tiene:
    \begin{equation*}
    \begin{aligned}
        \frac{d^2\Theta}{d\theta^2} &= \frac{d}{d\theta}\left(-\sin\theta\frac{d\tilde{\Theta}}{dx}\right) \\
        &= \left[\frac{d}{d\theta}(-\sin\theta)\right] \frac{d\tilde{\Theta}}{dx} + (-\sin\theta) \frac{d}{d\theta}\left(\frac{d\tilde{\Theta}}{dx}\right) \\
        &= -\cos\theta \frac{d\tilde{\Theta}}{dx} - \sin\theta \frac{d}{dx}\left(\frac{d\tilde{\Theta}}{dx}\right)\frac{dx}{d\theta} \\
        &= -\cos\theta \frac{d\tilde{\Theta}}{dx} - \sin\theta \frac{d^2\tilde{\Theta}}{dx^2} (-\sin\theta)\,.
    \end{aligned}
    \end{equation*}
    Reordenando los términos:
    \begin{equation}\label{2.1_segunda_derivada}
        \frac{d^2\Theta}{d\theta^2} = \sin^2\theta \frac{d^2\tilde{\Theta}}{dx^2} - \cos\theta \frac{d\tilde{\Theta}}{dx}\,.
    \end{equation}
\end{itemize}
    Considerando ahora la EDO lineal de segundo orden del sistema \eqref{2.1_sistema_de_ecuaciones_2}:
    \begin{equation}\label{2.1_EDO_lineal_de_segundo_orden}
        \frac{d^2\Theta}{d\theta^2} + \frac{1}{\tan\theta} \frac{d\Theta}{d\theta} - \frac{\lambda r^2}{a^2} \Theta = 0\,,
    \end{equation}
    y sustituyendo las expresiones \eqref{2.1_funciones_auxiliares}, \eqref{2.1_primera_derivada} y \eqref{2.1_segunda_derivada}, obtenemos:
    \begin{equation*}
    \begin{aligned}
        &\left(\sin^2\theta \frac{d^2\tilde{\Theta}}{dx^2} - \cos\theta \frac{d\tilde{\Theta}}{dx}\right) + \frac{1}{\tan\theta} \left(-\sin\theta \frac{d\tilde{\Theta}}{dx}\right) - \frac{\lambda r^2}{a^2} \tilde{\Theta} = 0 \\
        &\Rightarrow \quad \sin^2\theta \frac{d^2\tilde{\Theta}}{dx^2} - \cos\theta \frac{d\tilde{\Theta}}{dx} - \cos\theta \frac{d\tilde{\Theta}}{dx} - \frac{\lambda r^2}{a^2} \tilde{\Theta} = 0 \\
        &\Rightarrow \quad \sin^2\theta \frac{d^2\tilde{\Theta}}{dx^2} - 2\cos\theta \frac{d\tilde{\Theta}}{dx} - \frac{\lambda r^2}{a^2} \tilde{\Theta} = 0\,.
    \end{aligned}
    \end{equation*}
    Es decir:
    \begin{equation}\label{2.1_EDO_lineal_de_segundo_orden_sustitucion}
        \sin^2\theta \frac{d^2\tilde{\Theta}}{dx^2} - 2\cos\theta \frac{d\tilde{\Theta}}{dx} - \frac{\lambda r^2}{a^2} \tilde{\Theta} = 0\,.
    \end{equation}

    Ahora, usando la identidad trigonométrica fundamental:
    \begin{equation*}
        \sin^2\theta + \cos^2\theta = 1 \quad \Rightarrow \quad \sin^2\theta = 1 - \cos^2\theta\,,
    \end{equation*}
    y considerando el cambio de variable \eqref{2.1_cambio_de_variable}, tenemos:
    \begin{equation}\label{2.1_seno_cuadrado}
        \sin^2\theta = 1 - x^2.
    \end{equation}

    Sustituyendo \eqref{2.1_seno_cuadrado} en \eqref{2.1_EDO_lineal_de_segundo_orden_sustitucion}, obtendremos una ecuación parecida en principio a la \textbf{Ecuación de Legendre}:
        \begin{equation}\label{2.1_EDO_legendre_similar}
        (1-x^2)\frac{d^2\tilde{\Theta}}{dx^2} - 2x \frac{d\tilde{\Theta}}{dx} - \frac{\lambda r^2}{a^2} \tilde{\Theta} = 0\,.
    \end{equation}
    Si imponemos que la constante que multiplica a la función $\tilde{\Theta}$ sin derivar en la ecuación \eqref{2.1_EDO_legendre_similar} satisfaga la condición
    
    \begin{equation}\label{2.1_ajuste_legendre} -\frac{\lambda r^2}{a^2} = n(n+1)\,, \end{equation}
    
donde $n \in \mathbb{N}_0 $. Observamos que, dado que las constantes $a^2$ y $r^2$ son positivas, y el miembro derecho de la ecuación \eqref{2.1_ajuste_legendre} también es siempre positivo, la constante de separación $\lambda$ debe ser necesariamente negativa, es decir, $\lambda < 0$.

Sustituyendo \eqref{2.1_ajuste_legendre} en \eqref{2.1_EDO_legendre_similar}, se obtiene la siguiente forma de la ecuación diferencial asociada a las funciones angulares:

\begin{equation}\label{2.1_EDO_legendre}
(1-x^2)\frac{d^2\tilde{\Theta}}{dx^2} - 2x \frac{d\tilde{\Theta}}{dx} + n(n+1)\tilde{\Theta} = 0\,,
\end{equation}

la cual corresponde a la ecuación de Legendre. Su solución general está dada por

\begin{equation*}
\tilde{\Theta}(x) = B_1 P_n(x) + B_2 Q_n(x)\,,
\end{equation*}

donde $P_n(x)$ representa los polinomios de Legendre, funciones analíticas en el intervalo cerrado $[-1,1]$, mientras que $Q_n(x)$ son las funciones de Legendre de segunda especie, que presentan divergencias en los extremos $x = \pm 1$.

Por otra parte, el problema impone la condición inicial $\Psi(\theta, 0) = f(\theta)$, donde $f(\theta)$ es una función conocida y \textbf{finita en toda la esfera}. Esto implica que la solución general de la EDP \eqref{2.1_ecuación_de_calor_consideración}, y por ende sus factores separados como $\tilde{\Theta}(x)$, \textbf{deben permanecer finitos} en todo el dominio $x \in [-1,1]$.

Dado que las funciones $Q_n(x)$ \textbf{divergen en los bordes del intervalo}, los términos que las contienen \textbf{no son compatibles con esta condición de regularidad física} (no puede diverger a infinito). En consecuencia, se descartan y se conserva \textbf{únicamente la parte regular de la solución}, expresada mediante los polinomios de Legendre:


\begin{equation}\label{2.1_solución_Legendre}
\tilde{\Theta}(x) = B_1 P_n(x)\,,
\end{equation}

donde $B_1 \in \mathbb{R}$ es una constante real.


Sustituimos \eqref{2.1_funciones_auxiliares} y \eqref{2.1_cambio_de_variable} en \eqref{2.1_solución_Legendre}, de modo que
\begin{equation}\label{2.1_edo_segundo_orden}
    \Theta(\theta) = B_1 P_n(\cos\theta)\,,
\end{equation}

sera la solución general de la EDO lineal de segundo orden \eqref{2.1_EDO_lineal_de_segundo_orden}.
Obtenemos una solución particular de la EDP al sustituir \eqref{2.1_solucion_edo_1er_orden} y \eqref{2.1_edo_segundo_orden} en \eqref{2.1_solución_propuesta}, de modo:
\begin{equation}\label{2.1_solución_particular}
    \Psi_n(\theta,t) = \left(B_1 P_n(\cos\theta)\right)\left(A_1 \exp(\lambda t)\right)\,.
\end{equation}

Recordemos la restricción que impusimos a la constante de separación $\lambda$ en \eqref{2.1_ajuste_legendre}, de la cual se despeja:
\begin{equation}\label{2.1_lambda_n}
    \lambda_n = -\frac{a^2}{r^2} n(n+1), \quad \text{con} \quad n\in\mathbb{N}_0\,.
\end{equation}

Sustituyendo \eqref{2.1_lambda_n} en \eqref{2.1_solución_particular} y reagrupando las constantes $A_1$ y $B_1$ en una nueva constante $A_n := A_1 B_1$, obtenemos:
\begin{equation}\label{2.1_solución_particular_edp}
    \Psi_n(\theta,t) = A_n P_n(\cos\theta) \exp\left(-\frac{a^2}{r^2} n(n+1) t\right)\,,
\end{equation}

donde $A_n$ es una constante real asociada a cada solución particular $\Psi_n$.

\clearpage

La solución general de la EDP \eqref{2.1_ecuación_de_calor_ejercicio} será la superposición de todas las soluciones particulares \eqref{2.1_solución_particular_edp} para todo $n\in\mathbb{N}_0$, es decir:
\begin{equation}\label{2.1_solución_general_edp}
    \Psi(\theta,t) = \sum_{n=0}^\infty A_n P_n(\cos\theta) \exp\left(-\frac{a^2}{r^2} n(n+1) t\right)\,,
\end{equation}

donde los coeficientes $A_n$ se determinan de la condición inicial:
\begin{equation}\label{2.1_condicion_incial}
    \Psi(\theta,0) = f(\theta)\,.
\end{equation}

Evaluamos la expresión general de la solución \eqref{2.1_solución_general_edp} en el instante $t = 0$:

\begin{equation*}
         \Psi(\theta,0)= \sum_{n=0}^\infty A_n P_n(\cos\theta) \exp\left(-\frac{a^2}{r^2} n(n+1) \cdot 0\right)
\end{equation*}

Esto nos lleva a la siguiente expresión para el perfil inicial de temperatura sobre la esfera:
\begin{equation}\label{2.1_t=0}
    \Psi(\theta,0) = \sum_{n=0}^\infty A_n P_n(\cos\theta)\,.
\end{equation}

Imponemos la condición inicial \eqref{2.1_condicion_incial} sobre \eqref{2.1_t=0}, obtenemos la siguiente expansión en serie de Legendre para la función $f(\theta)$:

\begin{equation*}
    f(\theta) = \sum_{n=0}^\infty A_n P_n(\cos\theta)\,,
\end{equation*}
donde $A_n$ estara dado por:
\begin{equation}\label{2.1_coeficiente_an}
    A_n = \frac{2n+1}{2} \int_0^\pi f(\theta)\, P_n(\cos\theta)\, \sin\theta\, d\theta \,.
\end{equation}

Sustituyendo \eqref{2.1_coeficiente_an} en \eqref{2.1_solución_general_edp}, obtenemos:
\begin{equation}
        \Psi(\theta, t) = \sum_{n=0}^\infty \left[\frac{2n+1}{2} \int_0^\pi f(\theta)\, P_n(\cos\theta)\, \sin\theta\, d\theta \right]P_n(\cos\theta)\,\exp\left(-\frac{a^2}{r^2} n(n+1) t\right)
\end{equation}
donde $A_n$ son los coeficientes de la expansión de $f(\theta)$ en la base de polinomios de Legendre $\{P_n(\cos\theta)\}_{n=0}^\infty$. De este modo, se ha encontrado una expresión general para la solución $\Psi(\theta,t)$ de la EDP \eqref{2.1_ecuación_de_calor_ejercicio}, escrita en términos de $a$, $r$, $n$ y $f$.

\clearpage
%%%%%%%%%%%%%%%%%%%%%%%%%%%%%%%%%%%%%%%%%%%%%%%%%%%%%%%%%%%%%%%%%%%%%%%%%%%%%%%%%%%%%%%%%%%%%%%%%%%%%%%%%%%%%%%%%%%%%%%%%%%%%%%%%%%%%%%%%%%%%%%%%%%%%%%%%%%%%%%%%%%%%%%%%%%%%%%%%%%%%%%%%%%%%%%%%%%%%%%
\subsection{Evaluación explícita de la solución para un caso particular}

Evalúe explícitamente la solución en el caso particular en que $f(\theta) = T_0(1 + \beta \cos\theta)^2$.
 
\textbf{Solución:} 

%%%%%%%%%%%%%%%%%%%%%%%%%%%%%%%%%%%%%%%%%%%%%%%%%%%%%%%%%%%%%%%%%%%%%%%%%%%%%%%%%%%%%%%%%%%%%%%%%%%%%%%%%%%%%%%%%%%%%%%%%%%%%%%%%%%%%%%%%%%%%%%%%%%%%%%%%%%%%%%%%%%%%%%%%%%%%%%%%%%%%%%%%%%%%%%%%%%%%%%

Tenemos el caso particular
\begin{equation*}
    f(\theta) = T_0 (1 + \beta \cos\theta)^2\,,
\end{equation*}
donde $T_0$ y $\beta$ son constantes reales. Expandiendo algebraicamente la expresión, se obtiene:
\begin{equation}\label{2.2_f(theta)_expandido}
    f(\theta) = T_0 + 2\beta \,T_0\cos\theta + \beta^2 \,T_0\cos^2\theta \,,
\end{equation}
es decir, una función polinómica de segundo grado en $\cos\theta$.

A continuación, reescribiremos este polinomio como una combinación lineal explícita de los primeros tres polinomios de Legendre $\{P_n(x)\}_{n=0}^2$, donde $x\in[-1,1]$. Recordemos que:
\begin{equation*}
    P_0(x) = 1, \quad P_1(x) = x, \quad P_2(x) = \frac{1}{2}(3x^2 - 1)\,.
\end{equation*}

A partir de $P_2(x)$, despejamos $x^2$:
\begin{equation}\label{2.2_x^2}
    x^2 = \frac{2}{3}P_2(x) + \frac{1}{3} P_0(x) \, .
\end{equation}

De este modo, los monomios $1$, $x$ y $x^2$ pueden representarse como combinaciones lineales de los primeros polinomios de Legendre:
\begin{align*}
    1 &= P_0(x) \,, \\
    x &= P_1(x) \,, \\
    x^2 &= \frac{2}{3}P_2(x) + \frac{1}{3} P_0(x) \,.
\end{align*}

Realizando el cambio de variable $x = \cos\theta$, donde $\theta\in[0,\pi]$, obtenemos:
\begin{equation}\label{2.2_P_0}
    1 = P_0(\cos \theta) \,,
\end{equation}
\begin{equation}\label{2.2_P_1}
    \cos\theta = P_1(\cos \theta) \,,
\end{equation}
\begin{equation}\label{2.2_P_2}
    \cos^2\theta = \frac{2}{3}P_2(\cos \theta) + \frac{1}{3} \, .
\end{equation}

Sustituyendo las identidades \eqref{2.2_P_0}, \eqref{2.2_P_1} y \eqref{2.2_P_2} en la expansión de $f(\theta)$ obtenida en \eqref{2.2_f(theta)_expandido}, resulta:

\begin{equation*}
\begin{aligned}
        f(\theta) &= T_0P_0(\cos \theta) + 2\beta \,T_0P_1(\cos \theta) + \beta^2 \,T_0\left( \frac{2}{3}P_2(\cos\theta) + \frac{1}{3}P_0(\cos \theta)\right)\\
        &= T_0P_0(\cos \theta) + 2\beta \,T_0P_1(\cos \theta) + \frac{2}{3}\beta^2 \,T_0P_2(\cos\theta)  + \frac{1}{3}\beta^2 \,T_0P_0(\cos \theta)\\
        &= T_0\left(1+\frac{1}{3}\beta^2\right)P_0(\cos \theta) + 2\beta \,T_0P_1(\cos \theta) + \frac{2}{3}\beta^2 \,T_0P_2(\cos\theta) \, .
\end{aligned}
\end{equation*}

De este modo, $f(\theta)$ queda expresada como una combinación lineal de los primeros tres polinomios de Legendre en la base generada por $\{P_n(\cos\theta)\}_{n=0}^{2}$:
\begin{equation}\label{2.2_caso_particular_cl}
    f(\theta)= T_0\left(1+\frac{1}{3}\beta^2\right)P_0(\cos \theta) + 2\beta \,T_0P_1(\cos \theta) + \frac{2}{3}\beta^2 \,T_0P_2(\cos\theta) \,,
\end{equation}
donde identificamos los coeficientes $A_n$ para $n = 0, 1, 2$:
\begin{align}
    A_0 &= T_0\left(1+\frac{1}{3}\beta^2\right) \,, \label{2.2_coef_A0} \\
    A_1 &= 2\beta \,T_0 \,, \label{2.2_coef_A1} \\
    A_2 &= \frac{2}{3}\beta^2 \,T_0 \,. \label{2.2_coef_A2}
\end{align}

Sabemos que $f(\theta)$, en estricto rigor, debe expresarse como una expansión infinita en la base ortogonal de polinomios de Legendre:
\begin{equation*}
    f(\theta)=\sum_{n=0}^\infty A_n P_n(\cos\theta) \,,
\end{equation*}
pero en este caso se cumple que $A_n = 0$ para todo $n \geq 3$, ya que $f(\theta)$ es un polinomio de segundo grado en $\cos\theta$. Por tanto, también es válido representarla como una suma finita truncada en $n = 2$:

\begin{equation}\label{2.2_suma}
    f(\theta)=\sum_{n=0}^2 A_n P_n(\cos\theta) \,.
\end{equation}

Sustituyendo \eqref{2.2_suma} en la solución general \eqref{2.1_solución_general_edp}, se obtiene:

\begin{equation*}
    \Psi(\theta,t) = \sum_{n=0}^2 A_n P_n(\cos\theta)\exp\left(-\frac{a^2}{r^2} n(n+1)t\right),
\end{equation*}
Al expandir la suma:
\begin{equation*}
    \Psi(\theta,t) = A_0 P_0(\cos\theta) + A_1 P_1(\cos\theta)\exp\left(-\frac{2a^2}{r^2}t\right) + A_2 P_2(\cos\theta)\exp\left(-\frac{6a^2}{r^2}t\right) \,,
\end{equation*}
y reemplazando los coeficientes dados en \eqref{2.2_coef_A0}, \eqref{2.2_coef_A1} y \eqref{2.2_coef_A2}, resulta:

\begin{equation*}
    \Psi(\theta,t) = T_0\left(1 + \frac{1}{3}\beta^2\right) P_0(\cos\theta) + 2\beta T_0 \,P_1(\cos\theta)\exp\left(-\frac{2a^2}{r^2}t\right) + \frac{2}{3}\beta^2 T_0 \,P_2(\cos\theta)\exp\left(-\frac{6a^2}{r^2}t\right) \,.
\end{equation*}

Así, obtenemos una solución particular de la EDP, escrita como combinación de los tres primeros modos de Legendre, que son los únicos que aportan debido al grado del polinomio original $f(\theta)$.


\clearpage
%%%%%%%%%%%%%%%%%%%%%%%%%%%%%%%%%%%%%%%%%%%%%%%%%%%%%%%%%%%%%%%%%%%%%%%%%%%%%%%%%%%%%%%%%%%%%%%%%%%%%%%%%%%%%%%%%%%%%%%%%%%%%%%%%%%%%%%%%%%%%%%%%%%%%%%%%%%%%%%%%%%%%%%%%%%%%%%%%%%%%%%%%%%%%%%%%%%%%%%
\section{Ejercicio 3}\label{sec:3}

Considere una esfera conductora de radio $a$, construida a partir de dos cascarones semiesféricos separados por un pequeño anillo aislante en $z = 0$. Los hemisferios superior e inferior se mantienen a potencial $V_0$ y $-V_0$, respectivamente. Suponga además que el sistema no necesariamente posee simetría axial.

\begin{figure}[h]
    \centering
    \includegraphics[width=0.3\linewidth]{fig1.png}
    \caption{Situación descrita en el ejercicio 3.}
    \label{fig:esfera}
\end{figure}

%%%%%%%%%%%%%%%%%%%%%%%%%%%%%%%%%%%%%%%%%%%%%%%%%%%%%%%%%%%%%%%%%%%%%%%%%%%%%%%%%%%%%%%%%%%%%%%%%%%%%%%%%%%%%%%%%%%%%%%%%%%%%%%%%%%%%%%%%%%%%%%%%%%%%%%%%%%%%%%%%%%%%%%%%%%%%%%%%%%%%%%%%%%%%%%%%%%%%%%
\subsection{Identificación de la ecuación y condiciones de borde físicas} 

Identifique la ecuación y las condiciones de borde físicas que surgen del problema.

\textbf{Solución:} 
 
%%%%%%%%%%%%%%%%%%%%%%%%%%%%%%%%%%%%%%%%%%%%%%%%%%%%%%%%%%%%%%%%%%%%%%%%%%%%%%%%%%%%%%%%%%%%%%%%%%%%%%%%%%%%%%%%%%%%%%%%%%%%%%%%%%%%%%%%%%%%%%%%%%%%%%%%%%%%%%%%%%%%%%%%%%%%%%%%%%%%%%%%%%%%%%%%%%%%%%%

\subsubsection*{-- Sobre identificar la ecuación del sistema:}

Consideremos un sistema de coordenadas esféricas cuyo origen coincide con el centro de la esfera definida en el enunciado, como se muestra en la figura~\ref{fig:esfera}. Dado que la esfera posee radio $a$, esta divide el espacio tridimensional $\mathbb{R}^3$ en dos regiones disjuntas:

\begin{itemize}\label{3.1_región_}
    \item Región interior a la esfera: $\Omega_\text{int} = \{ (r, \theta, \varphi) \in \mathbb{R}^3 : 0 < r < a \}$.
    \item Región exterior a la esfera: $\Omega_\text{ext} = \{ (r, \theta, \varphi) \in \mathbb{R}^3 : r > a \}$.
\end{itemize}

La esfera está conformada por dos cascarones semiesféricos conductores ubicados sobre su superficie, la cual constituye la frontera común entre ambas regiones:

\begin{equation*}
    \partial\Omega_\text{int} = \partial\Omega_\text{ext} = \{ (r, \theta, \varphi) \in \mathbb{R}^3 :  r = a \} \,.
\end{equation*}

Estos cascarones concentran toda la carga del sistema y son, por tanto, la fuente del campo eléctrico $\vec{E}$, el cual se propaga a través de todo el espacio $\mathbb{R}^3$. En las regiones $\Omega_\text{int}$ y $\Omega_\text{ext}$ no hay presencia de cargas libres, por lo que la densidad de carga $\rho$ en estas es nula. En consecuencia, la ley de Gauss en su forma diferencial se reduce a:

\begin{equation}\label{3.1_ley_de_Gauss_usar}
\vec{\nabla} \cdot \vec{E} = 0 \,.
\end{equation}

Cabe decir que la expresión \eqref{3.1_ley_de_Gauss_usar} no es válida sobre la superficie $r = a$, puesto que en ella la densidad de carga $\rho$ es no nula. 

Dado que el sistema está en equilibrio electrostático, el campo eléctrico $\vec{E}$ es conservativo y puede expresarse como el gradiente negativo de un potencial escalar:

\begin{equation}\label{3.1_potencial_campo_electrico}
\vec{E} = -\vec{\nabla} \phi \,.
\end{equation}

En adelante, nos referiremos a $\phi$ simplemente como el \textbf{potencial electrostático}.

\clearpage

Sustituyendo \eqref{3.1_potencial_campo_electrico} en \eqref{3.1_ley_de_Gauss_usar}, se obtiene:

\begin{equation*}
\vec{\nabla} \cdot (-\vec{\nabla} \phi) = 0 
\quad \Rightarrow \quad 
\nabla^2 \phi = 0 \,.
\end{equation*}


Concluimos entonces que el potencial electrostático $\phi$ satisface la ecuación de Laplace en las regiones $\Omega_\text{int}$ y $\Omega_\text{ext}$ del espacio\footnote{Parte del razonamiento presentado se apoya en la exposición conceptual desarrollada en: \emph{HyperPhysics, Georgia State University}, disponible en \url{http://hyperphysics.phy-astr.gsu.edu/hbasees/electric/laplace.html}.}:

\begin{equation}\label{3.1_laplace_esferica}
\nabla^2 \phi(r, \theta, \varphi) = 0 \,.
\end{equation}

\subsubsection*{-- Condiciones de borde y comportamiento físico del potencial}

Como ya se ha mencionado, la esfera está compuesta por dos cascarones semiesféricos conductores, eléctricamente aislados entre sí a lo largo del plano $z = 0$ por un pequeño anillo aislante. Dado que estamos trabajando en coordenadas esféricas, dicho plano —expresado originalmente en coordenadas cartesianas— se representa mediante $\theta = \frac{\pi}{2}$. 

El hemisferio correspondiente a $0 \leq \theta < \frac{\pi}{2}$ se mantiene a un potencial constante $+V_0$, mientras que el hemisferio opuesto, con $\frac{\pi}{2} < \theta \leq \pi$, está a potencial $-V_0$.

Así, el \textbf{potencial sobre la superficie esférica $r = a$} queda definido por la condición de borde:\label{sec:3.1}
\begin{equation}\label{3.1_condición_de_borde} 
\phi(a, \theta, \varphi) =
\begin{cases}
+V_0\,, & 0 \leq \theta < \dfrac{\pi}{2}, \\[6pt]
- V_0\,, & \dfrac{\pi}{2} < \theta \leq \pi.
\end{cases}
\end{equation}

Observamos que la condición de borde \eqref{3.1_condición_de_borde} presenta una discontinuidad en el plano $\theta = \frac{\pi}{2}$. Sin embargo, el potencial electrostático $\phi(r, \theta, \varphi)$ debe ser \textbf{continuo entre las interfases de las regiones $\Omega_\text{int}$ y $\Omega_\text{ext}$}. Esto impone una \textbf{condición de continuidad} del potencial en $r = a$ para cada valor fijo de $\theta \neq \frac{\pi}{2}$, es decir:

\begin{equation*}
\lim_{r \to a^-} \phi(r, \theta, \varphi) = \lim_{r \to a^+} \phi(r, \theta, \varphi) = \phi(a, \theta, \varphi), \quad \text{para } \theta \neq \dfrac{\pi}{2} \,.
\end{equation*}

El campo eléctrico $\vec{E}$ —generado por la distribución de cargas en $r=a$— decae rápidamente con la distancia. En particular, se anula en el infinito:
\begin{equation*}
\lim_{r \to \infty} \vec{E}(r, \theta, \varphi) = 0 \, .
\end{equation*}

Dado la relación $\vec{E} = -\vec{\nabla} \phi$, este comportamiento implica que el potencial también tiende a cero en el infinito. Se impone una \textbf{condición de decaimiento en el infinito}:
\begin{equation*}
\lim_{r \to \infty} \phi(r, \theta, \varphi) = 0 \, .
\end{equation*}

Dado que el potencial electrostático $\phi$ debe ser \textbf{finito en todo el dominio físico}, se requiere que la solución no diverja en el límite $r \to 0$.

\begin{equation*}
\lim_{r \to 0} \phi(r, \theta, \varphi) < \infty \,.
\end{equation*}

Estas cuatro condiciones, en conjunto con la ecuación de Laplace \eqref{3.1_laplace_esferica}, definen completamente el problema de valores de contorno asociado al potencial electrostático generado por los cascarones semiesféricos.

\clearpage
%%%%%%%%%%%%%%%%%%%%%%%%%%%%%%%%%%%%%%%%%%%%%%%%%%%%%%%%%%%%%%%%%%%%%%%%%%%%%%%%%%%%%%%%%%%%%%%%%%%%%%%%%%%%%%%%%%%%%%%%%%%%%%%%%%%%%%%%%%%%%%%%%%%%%%%%%%%%%%%%%%%%%%%%%%%%%%%%%%%%%%%%%%%%%%%%%%%%%%%
\subsection{Planteamiento de la solución general y elección de constantes de separación}

Plantee la solución general al problema, explicando el procedimiento para obtenerla y el motivo de la elección de los valores de las constantes de separación. Considere que el potencial en el interior y en el exterior se comportará de forma distinta en cada región.

\textbf{Solución:} 

%%%%%%%%%%%%%%%%%%%%%%%%%%%%%%%%%%%%%%%%%%%%%%%%%%%%%%%%%%%%%%%%%%%%%%%%%%%%%%%%%%%%%%%%%%%%%%%%%%%%%%%%%%%%%%%%%%%%%%%%%%%%%%%%%%%%%%%%%%%%%%%%%%%%%%%%%%%%%%%%%%%%%%%%%%%%%%%%%%%%%%%%%%%%%%%%%%%%%%%

Queremos resolver la ecuación de Laplace dada en \eqref{3.1_laplace_esferica}. En coordenadas esféricas el operador Laplaciano para una función escalar se expresa como:

\begin{equation*}
    \nabla^2  = 
    \frac{1}{r^2 } \frac{\partial }{\partial r} \left( r^2 \frac{\partial }{\partial r} \right)
    + \frac{1}{r^2 \sin\theta} \frac{\partial}{\partial \theta} \left( \sin\theta \frac{\partial }{\partial \theta} \right)
    + \frac{1}{r^2 \sin^2\theta} \frac{\partial^2}{\partial \varphi^2} \, .
\end{equation*}

Sustituyendo esta expresión en \eqref{3.1_laplace_esferica}, la ecuación de Laplace queda escrita explícitamente como:

\begin{equation}\label{3.2_laplaciano}
    \frac{1}{r^2 } \frac{\partial }{\partial r} \left( r^2 \frac{\partial \phi}{\partial r} \right)
    + \frac{1}{r^2 \sin\theta} \frac{\partial}{\partial \theta} \left( \sin\theta \frac{\partial \phi}{\partial \theta} \right)
    + \frac{1}{r^2 \sin^2\theta} \frac{\partial^2\phi}{\partial \varphi^2} = 0 \, .
\end{equation}

Observamos que la EDP \eqref{3.2_laplaciano} es lineal y depende de tres variables independientes:
    \begin{itemize} 
        \item $r$: coordenada radial. 
        \item $\theta$: coordenada colatitudinal.
        \item $\varphi$: coordenada azimutal.
    \end{itemize}

En coordenadas esféricas, estas variables recorren los dominios:
\begin{equation*}
r \in [0, \infty[\,, \qquad \theta \in [0, \pi]\,, \qquad \varphi \in [0, 2\pi[ \,,
\end{equation*}

generando así todo el espacio $\mathbb{R}^3$.
    
A continuación, abordaremos la resolución de esta ecuación implementando el \textbf{método de separación de variables}, proponiendo una solución \textbf{completamente separable} de la forma:
\begin{equation}\label{3.2_solución_propuesta}
    \phi(r, \theta, \varphi) = R(r)\,\Theta(\theta)\,\Phi(\varphi) \,,
\end{equation}
donde $R$, $\Theta$ y $\Phi$ son funciones auxiliares que dependen únicamente de una de las variables independientes $r$, $\theta$ y $\varphi$, respectivamente.

Sustituimos la solución propuesta en \eqref{3.2_solución_propuesta} dentro de la ecuación de Laplace \eqref{3.2_laplaciano}. Simplificando cada término, se obtiene:

\begin{equation*}
    \frac{\Theta\,\Phi}{r^2} \frac{d}{dr} \left( r^2 \frac{dR}{dr} \right)
    + \frac{R\,\Phi}{r^2 \sin\theta} \frac{d}{d\theta} \left( \sin\theta \frac{d\Theta}{d\theta} \right)
    + \frac{R\,\Theta}{r^2 \sin^2\theta} \frac{d^2\Phi}{d\varphi^2} =0\, .
\end{equation*}

Expresamos explícitamente las derivadas de segundo orden en $R$ y $\Theta$:

\begin{equation*}
    \frac{\Theta\,\Phi}{r^2} \left( r^2 \frac{d^2R}{dr^2} + 2r \frac{dR}{dr} \right)
    + \frac{R\,\Phi}{r^2 \sin\theta} \left( \sin\theta \frac{d^2\Theta}{d\theta^2} + \cos\theta \frac{d\Theta}{d\theta} \right)
    + \frac{R\,\Theta}{r^2 \sin^2\theta} \frac{d^2\Phi}{d\varphi^2}=0 \,.
\end{equation*}

Multiplicamos ambos lados por $r^2$ para simplificar denominadores comunes:

\begin{equation}\label{3.2_desarrollo_1}
\Theta\,\Phi \left( r^2 \frac{d^2R}{dr^2} + 2r \frac{dR}{dr} \right)
+ \frac{R\,\Phi}{\sin\theta} \left( \sin\theta \frac{d^2\Theta}{d\theta^2} + \cos\theta \frac{d\Theta}{d\theta} \right)
+ \frac{R\,\Theta}{\sin^2\theta} \frac{d^2\Phi}{d\varphi^2} = 0 \, .
\end{equation}

Dividimos ambos miembros de la ecuación \eqref{3.2_desarrollo_1} por la solución propuesta $R\,\Theta\,\Phi$:

\begin{equation*}
    \frac{1}{R} \left( r^2 \frac{d^2 R}{dr^2} + 2r \frac{dR}{dr} \right)
    + \frac{1}{\Theta \sin\theta} \left( \sin\theta \frac{d^2 \Theta}{d\theta^2} + \cos\theta \frac{d\Theta}{d\theta} \right)
    + \frac{1}{\Phi \sin^2\theta} \frac{d^2 \Phi}{d\varphi^2} = 0.
\end{equation*}

Aislamos ahora el término dependiente únicamente de la variable radial $r$ en un lado de la ecuación:

\begin{equation}\label{3.2_laplace_propuesta_arreglo_1}
    \frac{1}{R} \left( r^2 \frac{d^2 R}{dr^2} + 2r \frac{dR}{dr} \right)
    =
    -\left[
        \frac{1}{\Theta \sin\theta} \left( \sin\theta \frac{d^2 \Theta}{d\theta^2} + \cos\theta \frac{d\Theta}{d\theta} \right)
        + \frac{1}{\Phi \sin^2\theta} \frac{d^2 \Phi}{d\varphi^2}
    \right] \,.
\end{equation}

Como ambos miembros de la ecuación \eqref{3.2_laplace_propuesta_arreglo_1} dependen de variables distintas e independientes, cada uno debe ser igual a una \textbf{constante de separación}, que denotaremos por $\lambda_1 \in \mathbb{R}$. Entonces:
\begin{equation*}
        \frac{1}{R} \left( r^2 \frac{d^2 R}{dr^2} + 2r \frac{dR}{dr} \right)
    =
    -\left[
        \frac{1}{\Theta \sin\theta} \left( \sin\theta \frac{d^2 \Theta}{d\theta^2} + \cos\theta \frac{d\Theta}{d\theta} \right)
        + \frac{1}{\Phi \sin^2\theta} \frac{d^2 \Phi}{d\varphi^2}
    \right] = \lambda_1 \,.
\end{equation*}

Al restar $\lambda_1$ en cada miembro de la igualdad anterior se obtiene el siguiente sistema de ecuaciones diferenciales:
\begin{align}
    \frac{1}{R} \left( r^2 \frac{d^2 R}{dr^2} + 2r \frac{dR}{dr} \right) - \lambda_1 &= 0\,, \label{3.2_edo_radial} \\[4pt]
    \frac{1}{\Theta \sin\theta} \left( \sin\theta \frac{d^2 \Theta}{d\theta^2} + \cos\theta \frac{d\Theta}{d\theta} \right)
    + \frac{1}{\Phi \sin^2\theta} \frac{d^2 \Phi}{d\varphi^2} + \lambda_1 &= 0\,. \label{3.2_edo_angular}
\end{align}

Consideremos ahora la EDP \eqref{3.2_edo_angular}, que depende de las variables angulares $\theta$ y $\varphi$. Aislamos el término que depende exclusivamente de la variable azimutal $\varphi$, lo que conduce a:

\begin{equation}\label{3.2_laplace_propuesta_arreglo_2}
    \frac{1}{\Phi} \frac{d^2 \Phi}{d \varphi^2}
    =
    - \frac{\sin\theta}{\Theta} \left( \sin\theta \frac{d^2 \Theta}{d \theta^2} + \cos\theta \frac{d \Theta}{d \theta} \right)
    - \lambda_1 \sin^2 \theta.
\end{equation}

Dado que los lados de la igualdad \eqref{3.2_laplace_propuesta_arreglo_2} dependen de variables distintas, ambos deben ser iguales a una segunda \textbf{constante de separación} $\lambda_2 \in \mathbb{R}$. Esto conduce a un segundo sistema de ecuaciones diferenciales ordinarias:

\begin{align}
    \frac{d^2 \Phi}{d \varphi^2} + \lambda_2 \Phi &= 0 \,, \label{3.2_edo_phi} \\[4pt]
    \frac{d^2 \Theta}{d \theta^2} + \frac{1}{\tan\theta} \frac{d \Theta}{d \theta}  + \left( \lambda_1 - \frac{\lambda_2}{\sin^2\theta} \right) \Theta &= 0 \,, \label{3.2_edo_theta}
\end{align}

donde en la ecuación \eqref{3.2_edo_theta} hemos reorganizado los términos para obtener una forma explícita de la misma. 

A continuación resolveremos ambos sistemas de ecuaciones diferenciales:

\subsubsection*{-- Para la EDO \eqref{3.2_edo_phi} asociada a la coordenada azimutal $\varphi$:}

La coordenada azimutal $\varphi$ describe una rotación completa en torno al eje $z$, trazando una circunferencia en el plano $xy$. Por tanto, la función auxiliar $\Phi(\varphi)$ debe satisfacer una \textbf{condición de periodicidad}\footnote{Dado que $\varphi$ tiene dominio en $[0, 2\pi[$, entonces el periodo de la función auxiliar $\Phi(\varphi)$ es $T = 2\pi$.}:
\begin{equation*}
    \Phi(\varphi) = \Phi(\varphi + 2\pi) \,.
\end{equation*}

En consecuencia, las soluciones admisibles de la EDO \eqref{3.2_edo_phi} deben ser funciones periódicas de periodo $2\pi$. La solución general para un $\lambda_2 \in \mathbb{R}$ arbitrario es:

\begin{equation*}
    \Phi(\varphi) = A \cos(\sqrt{\lambda_2}\,\varphi) + B \sin(\sqrt{\lambda_2}\,\varphi),
\end{equation*}

donde $A$ y $B$ son constantes complejas arbitrarias. Para que esta función sea periódica, se requiere que $\sqrt{\lambda_2} = m$ con $m \in \mathbb{Z}$, es decir, $\lambda_2 = m^2$. En este caso, la solución se escribe como:

\begin{equation}\label{3.2_solución_Phi_m^2}
    \Phi(\varphi) = A \cos(m\varphi) + B \sin(m\varphi).
\end{equation}

En efecto, al imponer la condición de periodicidad, tenemos:

\begin{equation*}
    \cos(m\varphi + 2\pi m) = \cos(m\varphi), \qquad
    \sin(m\varphi + 2\pi m) = \sin(m\varphi),
\end{equation*}

lo cual se verifica únicamente cuando $m$ es un número entero.

\clearpage

\subsubsection*{-- Para la EDO \eqref{3.2_edo_theta} asociada a la coordenada colatitudinal $\theta$:}

Realizamos el cambio de variable $x = \cos\theta$, con $x \in [-1, 1]$. Este procedimiento ya fue desarrollado previamente en el ítem~\ref{item:cambio-variable-theta} de la sección~\ref{sec:2.1}, al tratar la ecuación diferencial de Legendre estándar. Bajo este cambio de variable, la ecuación \eqref{3.2_edo_theta} se transforma en:

\begin{equation}\label{3.2_legendre_asociada}
(1 - x^2) \frac{d^2 \Theta}{dx^2} - 2x \frac{d \Theta}{dx} + \left( \lambda_1 - \frac{m^2}{1 - x^2} \right) \Theta = 0\,,
\end{equation}

la cual corresponde a la \textbf{ecuación asociada de Legendre}, de orden $m$.

Su solución general se expresa como una combinación lineal de funciones de primera y segunda especie \footnote{Las condiciones físicas del problema determinarán más adelante si la función de segunda especie es admisible como solución.
}:

\begin{equation}\label{3.2_solucion_theta}
\Theta(\theta) = C\,P_n^m(\cos\theta) + D\,Q_n^m(\cos\theta)\,,
\end{equation}

donde $C$ y $D$ son constantes complejas arbitrarias. La dependencia de la constante de separación $\lambda_1$ con el índice $n$ es tal que:

\begin{equation}\label{3.2_lambda_1=n(n+1)}
\lambda_1 = n(n+1), \qquad \text{con } n \in \mathbb{N}_0, \quad n \geq |m| \,.
\end{equation}


\subsubsection*{-- Para la EDO \eqref{3.2_edo_radial} asociada a la coordenada radial $r$:}

Sustituyendo \eqref{3.2_lambda_1=n(n+1)} en la ecuación radial \eqref{3.2_edo_radial}, obtenemos:

\begin{equation}\label{3.2_edo_radial_2}
      r^2\frac{d^2 R}{dr^2} + 2r \frac{dR}{dr} - n(n+1)R = 0\,, 
\end{equation}

la cual corresponde a una ecuación de tipo \textbf{Euler-Cauchy}. Su solución general está dada por:

\begin{equation}\label{3.2_solucion_R}
    R_n(r)=E\,r^n + F\,r^{-(n+1)} \,,
\end{equation}

donde $E$ y $F$ son constantes complejas arbitrarias, y $n$ es un número natural fijo tal que $n \geq |m|$.

\subsubsection*{-- Para la solución general de la ecuación \eqref{3.2_laplaciano}:}

Sustituyendo las soluciones obtenidas en \eqref{3.2_solución_Phi_m^2}, \eqref{3.2_solucion_theta} y \eqref{3.2_solucion_R} en la solución propuesta \eqref{3.2_solución_propuesta}, se obtiene una \textbf{solución particular} de la EDP \eqref{3.2_laplaciano} correspondiente a valores fijos de $n$ y $m$:

\begin{equation*}\label{3.2_solución_particular}
\phi_{n m}(r,\theta,\varphi) = \left( E_{n m}r^n + F_{n m}r^{-(n+1)} \right) \left(C_{n m}P_n^m(\cos\theta)+D_{n m}\,Q_n^m(\cos\theta)\right)\left(A_{n m} \cos(m\varphi) + B_{n m}\sin(m\varphi) \right) \, .
\end{equation*}

La \textbf{solución general} de la ecuación de la EDP \eqref{3.2_laplaciano} se obtiene como la combinación lineal de las soluciones particulares \eqref{3.2_solución_particular}, para cada par de índices $n \in \mathbb{N}_0$ y $m \in \mathbb{Z}$ tales que $|m| \leq n$:

\begin{equation}\label{3.2_solución_general}
\phi(r, \theta, \varphi) = \sum_{n=0}^{\infty} \sum_{m=-n}^{n} \left( E_{n m}r^n + F_{n m}r^{-(n+1)} \right) \left(C_{n m}P_n^m(\cos\theta)+D_{n m}\,Q_n^m(\cos\theta)\right)\left(A_{n m} \cos(m\varphi) + B_{n m}\sin(m\varphi) \right) \,,
\end{equation}

donde $A_{n m}$, $B_{n m}$, $C_{n m}$, $D_{n m}$, $E_{n m}$ y $F_{n m}$ son constantes complejas arbitrarias.

\clearpage

\subsubsection*{-- Sobre la exclusión de soluciones no físicas en $\theta$:}

La solución general \eqref{3.2_solución_general} es válida desde el punto de vista matemático; sin embargo, no necesariamente representa una solución físicamente aceptable para el problema planteado.

En particular, la función auxiliar $\Theta(\theta)$ aparece como solución de la EDO \eqref{3.2_legendre_asociada}, y está dada por una combinación lineal entre funciones asociadas de Legendre de primera y segunda especie, con $x = \cos\theta$:

\begin{equation*}
\Theta(\theta) = C_{n m} P_n^m(\cos\theta) + D_{n m} Q_n^m(\cos\theta)\,.
\end{equation*}

Sin embargo, tal como se advierte en el apunte del curso\footnote{Véase sección sobre la ecuación asociada de Legendre en el apunte de Física Matemática 2.}, las funciones $Q_n^m(x)$ divergen en los extremos del intervalo $x \in [-1,1]$, es decir, en los puntos $\theta = 0$ y $\theta = \pi$. Esto resulta inaceptable desde el punto de vista físico, pues el potencial electrostático $\phi(r,\theta,\varphi)$ debe ser una función continua y finita en todo el dominio espacial $\mathbb{R}^3$.

Para garantizar dicha regularidad angular, debemos excluir los términos que contengan $Q_n^m$. En consecuencia, se impone que la constante $D_{n m}$ sea nula para todo $n$ y $m$:

\begin{equation*}
D_{n m} = 0 \,.
\end{equation*}

La expresión general del potencial queda entonces restringida a:

\begin{equation}\label{3.2_solución_restringida}
\phi(r, \theta, \varphi) = \sum_{n=0}^{\infty} \sum_{m=-n}^{n} \left( E_{n m}r^n + F_{n m}r^{-(n+1)} \right) C_{n m}P_n^m(\cos\theta)\left(A_{n m} \cos(m\varphi) + B_{n m}\sin(m\varphi) \right) \,,
\end{equation}
donde $A_{n m}$, $B_{n m}$, $C_{n m}$, $E_{n m}$ y $F_{n m}$ son constantes complejas arbitrarias.

\subsubsection*{-- Implementación de armónicos esféricos:}

Introducimos los \textbf{armónicos esféricos} $Y_n^m(\theta,\varphi)$, tal como se definen en el apunte del curso\footnote{Véase sección sobre Armónicos Esféricos en el apunte de Física Matemática 2.}:

\begin{equation}\label{3.2_armonico_normalizado}
Y_n^m(\theta, \varphi) =
(-1)^m \sqrt{ \frac{2n+1}{4\pi} \cdot \frac{(n-m)!}{(n+m)!} } \, P_n^m(\cos\theta) \, e^{i m \varphi} \, ,
\end{equation}

donde $P_n^m(\cos\theta)$ son los polinomios asociados de Legendre de primera especie y el índice entero $m$ cumple $|m| \leq n$. 

Para implementar estos armónicos en la solución del potencial \eqref{3.2_solucion_armonicos}, primero reescribimos la función azimutal $\Phi(\varphi)$ como el producto entre una constante y la exponencial compleja con argumento $m\varphi$, para ello definimos:
\begin{equation*}
B_{n m} = i A_{n m} \,.
\end{equation*}

De este modo la expresión que define a la función azimutal adopta de la forma:

\begin{equation}\label{3.2_exponencial_compleja}
A_{n m} \cos(m\varphi) + B_{n m} \sin(m\varphi) = A_{n m} e^{i m \varphi} \,.
\end{equation} 

Sustituyendo \eqref{3.2_exponencial_compleja} en la expresión del potencial obtenida en \eqref{3.2_solución_restringida}, se tiene:

\begin{equation}\label{3.2_solucion_intermedia}
\phi(r, \theta, \varphi) = \sum_{n=0}^{\infty} \sum_{m=-n}^{n} \left( E_{n m} r^n + F_{n m} r^{-(n+1)} \right) G_{n m} P_n^m(\cos\theta) \, e^{i m \varphi} \,,
\end{equation}

donde hemos definido $G_{n m} := C_{n m} A_{n m}$ como una constante compleja. Ahora, si elegimos $G_{n m}$ como el factor de normalización de los armónicos esféricos, esto es,

\begin{equation*}
G_{n m} := (-1)^m \sqrt{ \frac{2n+1}{4\pi} \cdot \frac{(n-m)!}{(n+m)!} } \, ,
\end{equation*}

entonces podemos sustituir \eqref{3.2_armonico_normalizado} en \eqref{3.2_solucion_intermedia}, obteniendo finalmente la solución expresada en términos de armónicos esféricos normalizados:

\begin{equation}\label{3.2_solucion_armonicos}
\phi(r, \theta, \varphi) = \sum_{n=0}^{\infty} \sum_{m=-n}^{n} \left( E_{n m} r^n + F_{n m} r^{-(n+1)} \right) Y_n^m(\theta, \varphi) \,.
\end{equation}

La implementación de los armónicos esféricos no solo permite compactar la expresión de la solución, sino que también facilitará, en la sección \ref{sec:3.3}, la imposición de las condiciones de borde y el análisis de los casos con $m \neq 0$.

\subsubsection*{-- Sobre el comportamiento radial del potencial electrostático en la región interior $\Omega_{\text{int}}$:}

En la región interior $\Omega_\text{int}$, se exige que $\phi$ sea continua y finita en todo su dominio. En particular, se exige que el límite del potencial cuando $r$ tiende a cero sea finito:

\begin{equation*}
\lim_{r \to 0} \phi(r, \theta, \varphi) < \infty \,.
\end{equation*}

Aplicando esta condición a la solución del potencial \eqref{3.2_solucion_armonicos}, observamos que:

\begin{equation*}
\begin{aligned}
     \lim_{r \to 0} \phi(r, \theta, \varphi) 
     &= \lim_{r \to 0} \sum_{n=0}^{\infty} \sum_{m=-n}^{n} 
     \left( E_{n m}r^n + F_{n m}r^{-(n+1)} \right) 
     Y_n^m(\theta, \varphi)
     \\[6pt]
     &= \sum_{n=0}^{\infty} \sum_{m=-n}^{n} \left(
     E_{n m}\cancelto{0}{ r^n} + 
     F_{n m}\cancelto{\infty}{ r^{-(n+1)}} \right)
     Y_n^m(\theta, \varphi)
     \\[6pt]
     &=\infty \,,
\end{aligned}
\end{equation*}

es decir, los términos proporcionales a $r^{-(n+1)}$ divergen cuando $r \to 0$. Para garantizar que se satisfaga la condición de finitud del potencial electrostático en el origen, se debe imponer:

\begin{equation*}
F_{nm} = 0 \quad \text{para todo } n, m.
\end{equation*}

De este modo, la solución del potencial sobre la región $\Omega_{\text{int}}$ ($r<a$) se reduce a:

\begin{equation}\label{3.2_solucion_interior}
\phi(r, \theta, \varphi) = \sum_{n=0}^{\infty} \sum_{m=-n}^{n} E_{nm}  r^n\, Y_n^m(\theta, \varphi) \,.
\end{equation}

\subsubsection*{Sobre el comportamiento radial del potencial electrostático sobre la región exterior $\Omega_{\text{ext}}$:}

En la región exterior $\Omega_\text{ext}$, se espera que el potencial electrostático que sea despreciable a grandes distancias $r$ respecto del centro de la esfera conductora, es decir: 

\begin{equation*}
    \lim_{r \to \infty} \phi(r, \theta, \varphi) = 0 \,.
\end{equation*}

Aplicando esta condición a la solución del potencial \eqref{3.2_solucion_armonicos}, observamos que:

\begin{equation*}
\begin{aligned}
     \lim_{r \to \infty} \phi(r, \theta, \varphi) 
     &= \lim_{r \to \infty} \sum_{n=0}^{\infty} \sum_{m=-n}^{n} 
     \left( E_{n m}r^n + F_{n m}r^{-(n+1)} \right) 
     Y_n^m(\theta, \varphi)
     \\[6pt]
     &= \sum_{n=0}^{\infty} \sum_{m=-n}^{n} \left(
     E_{n m}\cancelto{\infty}{ r^n} + 
     F_{n m}\cancelto{0}{ r^{-(n+1)}} \right)
     Y_n^m(\theta, \varphi)
     \\[6pt]
     &=\infty \,,
\end{aligned}
\end{equation*}

es decir, los términos proporcionales a $r^n$ divergen cuando $r \to \infty$. Para garantizar que se satisfaga la condición de decaimiento del potencial electrostático en el infinito, se debe imponer:

\begin{equation*}
E_{nm} = 0 \quad \text{para todo } n, m.
\end{equation*}

De este modo, la solución del potencial sobre la región $\Omega_{\text{ext}}$ ($r>a$) se reduce a:

\begin{equation}\label{3.2_solucion_exterior}
\phi(r, \theta, \varphi) = \sum_{n=0}^{\infty} \sum_{m=-n}^{n} F_{nm}  r^{-(n+1)}\, Y_n^m(\theta, \varphi) \,.
\end{equation}

\subsubsection*{-- Forma del potencial en todo el espacio:}

Aplicadas las condiciones de finitud en el origen y de decaimiento en el infinito, el potencial electrostático $\phi(r, \theta, \varphi)$ queda definido por tramos según la región del espacio:

\begin{equation}\label{3.2_solucion_tramos_1}
    \phi(r,\theta,\varphi) = 
    \begin{cases}
        \displaystyle\sum_{n=0}^{\infty} \sum_{m=-n}^{n} E_{nm} \, r^n \, Y_n^m(\theta, \varphi)\,, & 0 \leq r < a \,,\\[6pt]
        \displaystyle\sum_{n=0}^{\infty} \sum_{m=-n}^{n} F_{nm} \, r^{-(n+1)} \, Y_n^m(\theta, \varphi)\,, & r > a \,.
    \end{cases}
\end{equation}

Para garantizar la \textbf{continuidad del potencial en la superficie} $r = a$ para todo ángulo $\theta \neq \pi/2$, se impone la siguiente condición:

\begin{equation*}
    \lim_{r \to a^-} \phi(r, \theta,\varphi) = \lim_{r \to a^+} \phi(r, \theta,\varphi) \,,
\end{equation*}
lo que equivale a:
\begin{equation*}
    \sum_{n=0}^{\infty} \sum_{m=-n}^{n} E_{nm} \, a^n \, Y_n^m(\theta, \varphi) 
    = \sum_{n=0}^{\infty} \sum_{m=-n}^{n} F_{nm} \, a^{-(n+1)} \, Y_n^m(\theta, \varphi) \,.
\end{equation*}

Como los armónicos esféricos $Y_n^m(\theta,\varphi)$ forman una base ortonormal sobre la esfera, se igualan coeficientes:

\begin{equation*}
    E_{nm} \, a^n = F_{nm} \, a^{-(n+1)} \,,
\end{equation*}
de donde se deduce:
\begin{equation}\label{3.2_E_nm}
    E_{nm} = F_{nm} \, a^{-(2n+1)} \,.
\end{equation}

Sustituyendo \eqref{3.2_E_nm} en la expresión \eqref{3.2_solucion_tramos_1}, se obtiene el potencial definido en todo el espacio $\mathbb{R}^3$ como:

\begin{equation}\label{3.2_solucion_tramos_final}
    \phi(r,\theta,\varphi) = 
    \begin{cases}
        \displaystyle\sum_{n=0}^{\infty} \sum_{m=-n}^{n} F_{nm} \, a^{-(2n+1)} \, r^n \, Y_n^m(\theta, \varphi)\,, & 0 \leq r < a \,,\\[6pt]
        \displaystyle\sum_{n=0}^{\infty} \sum_{m=-n}^{n} F_{nm} \, r^{-(n+1)} \, Y_n^m(\theta, \varphi)\,, & r > a \,.
    \end{cases}
\end{equation}

Esta forma será utilizada en la sección siguiente al imponer la condición de borde dada sobre la superficie $r = a$.

\clearpage
%%%%%%%%%%%%%%%%%%%%%%%%%%%%%%%%%%%%%%%%%%%%%%%%%%%%%%%%%%%%%%%%%%%%%%%%%%%%%%%%%%%%%%%%%%%%%%%%%%%%%%%%%%%%%%%%%%%%%%%%%%%%%%%%%%%%%%%%%%%%%%%%%%%%%%%%%%%%%%%%%%%%%%%%%%%%%%%%%%%%%%%%%%%%%%%%%%%%%%%
\subsection{Solución particular del problema con condiciones de borde}\label{sec:3.3}

Encuentre la solución particular al problema, imponiendo las condiciones de borde halladas anteriormente.

\textbf{Solución:} 

%%%%%%%%%%%%%%%%%%%%%%%%%%%%%%%%%%%%%%%%%%%%%%%%%%%%%%%%%%%%%%%%%%%%%%%%%%%%%%%%%%%%%%%%%%%%%%%%%%%%%%%%%%%%%%%%%%%%%%%%%%%%%%%%%%%%%%%%%%%%%%%%%%%%%%%%%%%%%%%%%%%%%%%%%%%%%%%%%%%%%%%%%%%%%%%%%%%%%%%

A continuación impondremos las condiciones de borde identificadas en la sección \ref{sec:3.1} a la solución general de la ecuación de Laplace del potencial electrostático en coordenadas esféricas dada en \eqref{3.2_solucion_tramos_final}.

\subsubsection*{Para la condición del valor del potencial sobre la superficie esférica:}

Esta condición especifica el valor del potencial en la superficie de la esfera conductora $r = a$:

\begin{equation}\label{3.3_condicion_borde}
\phi(a, \theta, \varphi) =
\begin{cases}
+V_0, & 0 \leq \theta < \dfrac{\pi}{2}, \\[4pt]
-V_0, & \dfrac{\pi}{2} < \theta \leq \pi.
\end{cases}
\end{equation}

Evaluando la solución \eqref{3.2_solucion_tramos_final} en $r = a$, obtenemos la expresión del potencial sobre la superficie esférica:

\begin{equation*}
\begin{aligned}
    \phi(a,\theta,\varphi) &= 
    \begin{cases}
        \displaystyle\sum_{n=0}^{\infty} \sum_{m=-n}^{n} F_{nm} \, a^{-(2n+1)} \cdot a^n \, Y_n^m(\theta, \varphi)\,, & 0 \leq r < a \,,\\[6pt]
        \displaystyle\sum_{n=0}^{\infty} \sum_{m=-n}^{n} F_{nm} \cdot a^{-(n+1)} \, Y_n^m(\theta, \varphi)\,, & r > a \,.
    \end{cases} \\[10pt]
    &= 
    \begin{cases}
        \displaystyle\sum_{n=0}^{\infty} \sum_{m=-n}^{n} F_{nm} \cdot a^{-(n+1)} \, Y_n^m(\theta, \varphi)\,, & 0 \leq r < a \,,\\[6pt]
        \displaystyle\sum_{n=0}^{\infty} \sum_{m=-n}^{n} F_{nm} \cdot a^{-(n+1)} \, Y_n^m(\theta, \varphi)\,, & r > a \,.
    \end{cases} \\[10pt]
    &= \sum_{n=0}^{\infty} \sum_{m=-n}^{n} F_{nm} \cdot a^{-(n+1)} \, Y_n^m(\theta, \varphi) \,.
\end{aligned}
\end{equation*}


Obtenemos:

\begin{equation}\label{3.3_solucion_r=a}
\phi(a, \theta,\varphi) = \sum_{n=0}^{\infty} \sum_{m=-n}^{n} F_{nm}  a^{-(n+1)} Y_n^m(\theta, \varphi) \, ,
\end{equation}

lo que constituye una \textbf{serie de Laplace} sobre el intervalo $\theta \in [0, \pi]$. Así, los coeficientes $F_{n m}a^{-(n+1)}$ se obtienen mediante:

\begin{equation}\label{3.3_coeficientes_F_nm}
F_{n m} \, a^{-(n+1)} =  \int_{0}^{2\pi} \int_{0}^{\pi} \phi(a, \theta,\varphi)\, \left[ Y_n^m(\theta,\varphi) \right]^* \sin \theta\, d\theta \,d\varphi \,.
\end{equation}

Donde $\left[Y_n^m(\theta,\varphi)\right]^*$ denota el conjugado complejo del armónico esférico definido en \eqref{3.2_armonico_normalizado}:

\begin{equation}\label{3.3_armonico_esferico_conjugado}
    \left[ Y_n^m(\theta,\varphi) \right]^*=
(-1)^m \sqrt{ \frac{2n+1}{4\pi} \cdot \frac{(n-m)!}{(n+m)!} } \, P_n^m(\cos\theta) \, e^{-i m \varphi}
\end{equation}

Procedemos a imponer la condición de borde \eqref{3.3_condicion_borde} en \eqref{3.3_coeficientes_F_nm}, obtenemos:

\begin{equation}\label{3.3_coeficientes_F_nm_condiciones_de_borde}
F_{n m} \, a^{-(n+1)} =  \int_{0}^{2\pi} \left( \int_{0}^{\frac{\pi}{2}} V_0\, \left[ Y_n^m(\theta,\varphi) \right]^* \sin \theta\, d\theta - \int_{\frac{\pi}{2}}^{\pi} V_0\, \left[ Y_n^m(\theta,\varphi) \right]^* \sin \theta\, d\theta\right)\,d\varphi \,.
\end{equation}

Despejando $F_{nm}$ en la ecuación \eqref{3.3_coeficientes_F_nm_condiciones_de_borde}, obtenemos:

\begin{equation}\label{3.3_F_nm_condiciones_de_borde}
F_{n m} =  a^{n+1}\,V_0\, \int_{0}^{2\pi} \left( \int_{0}^{\frac{\pi}{2}} \left[ Y_n^m(\theta,\varphi) \right]^* \sin \theta\, d\theta - \int_{\frac{\pi}{2}}^{\pi} \left[ Y_n^m(\theta,\varphi) \right]^* \sin \theta\, d\theta\right)\,d\varphi \,.
\end{equation}

Sustituyendo \eqref{3.3_armonico_esferico_conjugado} en \eqref{3.3_F_nm_condiciones_de_borde}:

\begin{equation*}
\begin{aligned}
F_{n m} &=  a^{n+1}\,V_0\, \int_{0}^{2\pi} \Bigg( \int_{0}^{\frac{\pi}{2}} \left[
(-1)^m \sqrt{ \frac{2n+1}{4\pi} \cdot \frac{(n-m)!}{(n+m)!} } \, P_n^m(\cos\theta) \, e^{-i m \varphi} \right] \sin \theta\, d\theta \\
& - \int_{\frac{\pi}{2}}^{\pi} \left[ 
(-1)^m \sqrt{ \frac{2n+1}{4\pi} \cdot \frac{(n-m)!}{(n+m)!} } \, P_n^m(\cos\theta) \, e^{-i m \varphi} \right] \sin \theta\, d\theta\Bigg)\,d\varphi \\[6pt]
&=  a^{n+1}\,V_0\, \int_{0}^{2\pi} \Bigg( (-1)^m \sqrt{ \frac{2n+1}{4\pi} \cdot \frac{(n-m)!}{(n+m)!} }\, e^{-i m \varphi} \int_{0}^{\frac{\pi}{2}} 
 \, P_n^m(\cos\theta) \sin \theta\, d\theta \\
& - (-1)^m \sqrt{ \frac{2n+1}{4\pi} \cdot \frac{(n-m)!}{(n+m)!} }\, e^{-i m \varphi}\int_{\frac{\pi}{2}}^{\pi}  P_n^m(\cos\theta) \sin \theta\, d\theta\Bigg)\,d\varphi \\[6pt]
&=  a^{n+1}\,V_0\int_{0}^{2\pi} (-1)^m \sqrt{ \frac{2n+1}{4\pi} \cdot \frac{(n-m)!}{(n+m)!} }\, e^{-i m \varphi}\left( \int_{0}^{\frac{\pi}{2}} P_n^m(\cos\theta)\sin \theta\, d\theta - \int_{\frac{\pi}{2}}^{\pi} P_n^m(\cos\theta)\sin \theta\, d\theta\right)\,d\varphi \\[6pt]
&=   a^{n+1}\,V_0 \,(-1)^m \sqrt{ \frac{2n+1}{4\pi} \cdot \frac{(n-m)!}{(n+m)!} }
\left( \int_{0}^{2\pi} e^{-im\varphi} \, d\varphi \right) 
\left( \int_{0}^{\frac{\pi}{2}} P_n^m(\cos\theta) \sin \theta\, d\theta + \int_{\pi}^{\frac{\pi}{2}} P_n^m(\cos\theta) \sin \theta\, d\theta\right)\,.
\end{aligned}
\end{equation*}

Obtenemos:

\begin{equation}\label{3.3_F_mn_azimutal_separada}
F_{n m} =   a^{n+1}\,V_0 \,(-1)^m \sqrt{ \frac{2n+1}{4\pi} \cdot \frac{(n-m)!}{(n+m)!} }
\left( \int_{0}^{2\pi} e^{-im\varphi} \, d\varphi \right) 
\left( \int_{0}^{\frac{\pi}{2}} P_n^m(\cos\theta) \sin \theta\, d\theta + \int_{\pi}^{\frac{\pi}{2}} P_n^m(\cos\theta) \sin \theta\, d\theta\right)\,,
\end{equation}

donde antes de continuar resolveremos la integral en $\varphi$, de modo que obtenemos:

\begin{equation*}
\begin{aligned}
    \int_{0}^{2\pi} e^{-im\varphi} \, d\varphi 
    &= \Big[-\frac{1}{im}e^{-im\varphi}\Big]_0^{2\pi} \\[4pt]
    &= -\frac{1}{im}+\frac{1}{im}\\[6pt]
    &= 0 \,, 
\end{aligned}
\end{equation*}
para todo $m \in \mathbb{Z}$ tal que $m \neq 0$. 

Esto implica que el coeficiente $F_{n m}$ en \eqref{3.3_F_mn_azimutal_separada} se anula para todo $m \neq 0$. En consecuencia, la serie de Laplace del potencial electrostático dada por \eqref{3.3_solucion_r=a} no recibe contribución de los términos con $m \neq 0$.

Analicemos ahora el coeficiente dado en \eqref{3.3_F_mn_azimutal_separada} para el caso $m = 0$. En esta situación, se obtienen los siguientes resultados parciales:

\begin{itemize}
    \item \textbf{Para la integral en $\varphi$:}
    \begin{equation}\label{3.3_integral_varphi_m=0}
        \int_{0}^{2\pi} d\varphi = 2\pi \,.
    \end{equation}

    \item \textbf{Para el polinomio asociado de Legendre}, con $x = \cos\theta$, y a partir de la \textbf{fórmula de Rodrigues}:
    \begin{equation*}
        P_n^m(x) = \frac{1}{2^n n!}(1 - x^2)^{\frac{m}{2}} \, \frac{d^{n+m}}{dx^{n+m}}(x^2 - 1)^n \,,
    \end{equation*}
    En este caso:
        \begin{equation*}
            \begin{aligned}
               P_n^{0}(x) &= \frac{1}{2^nn!}(1 - x^2)^{\frac{0}{2}} \, \frac{d^{n+0}}{dx^{n+0}}(x^2-1)^n \\
                &= \frac{1}{2^nn!} \, \frac{d^{n}}{dx^{n}}(x^2-1)^n \\[4pt]
                &=P_n(x) \,.
            \end{aligned}
        \end{equation*}
    Se recupera así el polinomio de Legendre estándar:
    \begin{equation}\label{3.3_polinomio_de_legendre_m=0}
        P_n^0(\cos\theta)=P_n(\cos\theta) \,.
    \end{equation}
    \item \textbf{Para el factor de normalización de los armónicos esféricos}, se obtiene:
    \begin{equation}\label{3.3_factor_de_normalización_m=0}
        (-1)^0 \sqrt{ \frac{2n+1}{4\pi} \cdot \frac{(n-0)!}{(n+0)!} } = \sqrt{ \frac{2n+1}{4\pi} } \,.
    \end{equation}
\end{itemize}

Sustituyendo los resultados \eqref{3.3_integral_varphi_m=0}, \eqref{3.3_polinomio_de_legendre_m=0} y \eqref{3.3_factor_de_normalización_m=0} en la expresión \eqref{3.3_F_mn_azimutal_separada}, se obtiene:

\begin{equation}\label{3.3_F_n_azimutal_separada}
F_n = 2\pi \, a^{n+1} V_0 \, \sqrt{ \frac{2n+1}{4\pi} } 
\left( \int_{0}^{\frac{\pi}{2}} P_n(\cos\theta) \sin\theta\, d\theta + \int_{\pi}^{\frac{\pi}{2}} P_n(\cos\theta) \sin\theta\, d\theta \right) \,,
\end{equation}

donde se ha escrito $F_n$ como el valor particular del coeficiente $F_{n m}$ correspondiente al caso $m = 0$.

Por otra parte, conocemos la relación: 
\begin{equation}\label{3.3_relación}
\int_{-1}^{0} P_n(x)\, dx = (-1)^n \int_{0}^{1} P_n(x)\, dx \, .
\end{equation}
Aplicamos ahora el cambio de variable $x = \cos \theta$, donde $x \in [-1,1]$ y $\theta\in[0,\pi]$, y $dx = -\sin \theta\, d\theta$. El intervalo se transforma como sigue:
\begin{align*}
x= 1 \quad &\Rightarrow \quad \theta = 0 \, , \\
x = 0 \quad &\Rightarrow \quad \theta = \frac{\pi}{2} \, , \\
x = -1 \quad &\Rightarrow \quad \theta = \pi \, . \\
\end{align*}
Aplicando este cambio en \eqref{3.3_relación}, obtenemos:
\begin{equation}\label{3.3_relación_en_theta}
\int_{\pi}^{\frac{\pi}{2}} P_n(\cos\theta)\, \sin \theta\, d\theta = (-1)^n \int_{\frac{\pi}{2}}^{0} P_n(\cos\theta)\, \sin \theta\, d\theta \, .
\end{equation}

Sustituyendo la relación \eqref{3.3_relación_en_theta} en la expresión de los coeficientes \eqref{3.3_F_n_azimutal_separada}, obtenemos:

\begin{equation*}
\begin{aligned}
    F_n&= 2\pi \, a^{n+1}\,V_0 \, \sqrt{ \frac{2n+1}{4\pi} }  \left( \int_{0}^{\frac{\pi}{2}} P_n(\cos \theta) \sin \theta\, d\theta 
    + (-1)^n \int_{\frac{\pi}{2}}^{0} P_n(\cos \theta)\, \sin \theta\, d\theta \right) \\[4pt]
    &= 2\pi \, a^{n+1}\,V_0 \, \sqrt{ \frac{2n+1}{4\pi} }  \left( \int_{0}^{\frac{\pi}{2}} P_n(\cos \theta) \sin \theta\, d\theta 
    - (-1)^n \int_{0}^{\frac{\pi}{2}} P_n(\cos \theta)\, \sin \theta\, d\theta \right) \\[4pt]
    &= 2\pi \, a^{n+1}\,V_0 \, \sqrt{ \frac{2n+1}{4\pi} }  \left[ 1 - (-1)^n \right] \int_{0}^{\frac{\pi}{2}} P_n(\cos \theta) \sin \theta\, d\theta \,.
\end{aligned}
\end{equation*}

Así, los coeficientes quedan expresados como:

\begin{equation}\label{3.3_integrales_general}
F_n = 2\pi \, a^{n+1}\,V_0 \,\sqrt{ \frac{2n+1}{4\pi} } \left[ 1 - (-1)^n \right] \int_{0}^{\frac{\pi}{2}} P_n(\cos \theta) \sin \theta\, d\theta \,.
\end{equation}

Observamos que el factor que acompaña a la integral depende de la paridad de $n$:

\begin{equation*}
1 - (-1)^n = \left\{
    \begin{aligned}
        &0, &\quad \text{si } n \text{ es par,} \\
        &2, &\quad \text{si } n \text{ es impar.}
    \end{aligned}
\right.
\end{equation*}

En consecuencia, para todo $n$ par se anula el coeficiente $F_n$, y por tanto también su contribución al potencial electrostático. Esto implica que, en la solución final, sólo intervienen los términos correspondientes a valores impares de $n$.

Para dichos valores, podemos simplificar el coeficiente $F_n$ como:

\begin{equation}\label{3.3_integrales_general_n_impar}
F_n = 4\pi \, a^{n+1}\,V_0 \,\sqrt{ \frac{2n+1}{4\pi} }  \int_{0}^{\frac{\pi}{2}} P_n(\cos \theta)\, \sin \theta\, d\theta \,.
\end{equation}

Por otra parte, la función del potencial electrostático \eqref{3.2_solucion_tramos_final} para $m=0$ es independiente de $\varphi$, con $n$ impar, adopta la forma:

\begin{equation}\label{3.3_solucion_tramos_final_m=0}
    \phi(r,\theta) = 
    \begin{cases}
        \displaystyle\sum_{\substack{n=1 \\ n \text{ impar}}}^{\infty}  F_{n} \, a^{-(2n+1)} \, r^n \, \sqrt{ \frac{2n+1}{4\pi} } P_n(\cos\theta)\,, & 0 \leq r < a \,,\\[6pt]
        \displaystyle\sum_{\substack{n=1 \\ n \text{ impar}}}^{\infty}  F_{n} \, r^{-(n+1)} \, \sqrt{ \frac{2n+1}{4\pi} } P_n(\cos\theta)\,, & r > a \,.
    \end{cases}
\end{equation}

Sustituyendo la expresión \eqref{3.3_F_n_azimutal_separada} en \eqref{3.3_solucion_tramos_final_m=0}, se concluye que la solución del problema, bajo las condiciones de borde impuestas y restringida al caso $m = 0$, está dada por:

\begin{equation}\label{3.3_solucion_tramos_final}
    \phi(r,\theta) = 
    \begin{cases}
        \displaystyle\sum_{\substack{n=1 \\ n \text{ impar}}}^{\infty}  V_0  \, \left(\frac{r}{a}\right)^n P_n(\cos\theta)(2n+1)\int_{0}^{\frac{\pi}{2}} P_n(\cos \theta)\, \sin \theta\, d\theta \,, & 0 \leq r < a \,,\\[8pt]
        \displaystyle\sum_{\substack{n=1 \\ n \text{ impar}}}^{\infty}  V_0 \, \left(\frac{a}{r}\right)^{-(n+1)} P_n(\cos\theta)(2n+1)\int_{0}^{\frac{\pi}{2}} P_n(\cos \theta)\, \sin \theta\, d\theta\,, & r > a \,.
    \end{cases}
\end{equation}

Observemos el siguiente factor presente en las expresiones de \eqref{3.3_solucion_tramos_final}:

\begin{equation}\label{3.3_factor_integral}
    (2n+1)\int_{0}^{\frac{\pi}{2}} P_n(\cos \theta)\, \sin \theta\, d\theta = \int_{0}^{\frac{\pi}{2}} (2n+1)P_n(\cos \theta)\, \sin \theta\, d\theta \,.
\end{equation}

Aplicando en \eqref{3.3_factor_integral} el cambio de variable $x = \cos\theta$, el cual ya fue introducido anteriormente, se obtiene:

\begin{equation}\label{3.3_factor_integral_en_x}
    \int_{0}^{\frac{\pi}{2}} (2n+1)P_n(\cos \theta)\, \sin \theta\, d\theta = \int_{1}^{0} (2n+1)P_n(x)\, dx \,.
\end{equation}

Utilizando en \eqref{3.3_factor_integral_en_x} la relación de recurrencia para los polinomios de Legendre:

\begin{equation*}
    (2n+1)P_n(x) = \frac{d}{dx}P_{n+1}(x) - \frac{d}{dx}P_{n-1}(x) \,,
\end{equation*}

se sigue que:

\begin{equation*}
    \begin{aligned}
        \int_{1}^{0} (2n+1)P_n(x)\, dx &= \int_{1}^{0} \left( \frac{d}{dx}P_{n+1}(x) - \frac{d}{dx}P_{n-1}(x) \right) dx \\
        &= \int_{1}^{0} \frac{d}{dx}P_{n+1}(x)\, dx - \int_{1}^{0} \frac{d}{dx}P_{n-1}(x)\, dx \\
        &= \int_{1}^{0} \frac{d}{dx}P_{n+1}(x)\, dx + \int_{0}^{1} \frac{d}{dx}P_{n-1}(x)\, dx \\
        &= \Big[ P_{n+1}(x) \Big]_{1}^{0} + \Big[ P_{n-1}(x) \Big]_{0}^{1} \\[4pt]
        &= P_{n+1}(0) - P_{n+1}(1) + P_{n-1}(1) - P_{n-1}(0) \,.
    \end{aligned}
\end{equation*}

Por lo tanto, se obtiene:

\begin{equation}\label{3.3_factor_integral_resuelto}
     (2n+1)\int_{0}^{\frac{\pi}{2}} P_n(\cos \theta)\, \sin \theta\, d\theta = P_{n+1}(0) - P_{n+1}(1) + P_{n-1}(1) - P_{n-1}(0) \,.
\end{equation}

Para evaluar este resultado explícitamente, analizamos los valores de los polinomios de Legendre en los puntos $x = 0$ y $x = 1$:

\begin{itemize}
    \item Para $x = 0$, los polinomios toman valores determinados por:
    \begin{equation}\label{eq:valor_Pn_0}
        P_n(0) =
        \begin{cases}
        0\,, & \text{si $n$ es impar} \\[8pt]
        \displaystyle (-1)^{\frac{n}{2}} \frac{n!}{2^n \left( \left(\frac{n}{2}\right)! \right)^2} \,, & \text{si $n$ es par}
        \end{cases}
    \end{equation}

    \item Para $x = 1$, se tiene que los polinomios están normalizados, por lo que:
    \begin{equation*}
        P_n(1) = 1
    \end{equation*}
    para todo $n \in \mathbb{N}_0$. En consecuencia, se cumple que:
    \begin{equation}\label{3.3_diferencia}
        P_{n-1}(1) - P_{n+1}(1) = 0 \,, \quad \text{para todo } n \in \mathbb{N} \,.
    \end{equation}
\end{itemize}

Dado que en \eqref{3.3_solucion_tramos_final} la suma se restringe a valores impares de $n$, se tiene que tanto $n-1$ como $n+1$ son pares. En consecuencia, los polinomios de Legendre correspondientes evaluados en el origen no se anulan. Por lo tanto, considerando lo establecido en \eqref{3.3_diferencia}, el factor \eqref{3.3_factor_integral_resuelto} puede simplificarse del siguiente modo:
\begin{equation}\label{3.3_factor_integral_desarrollo}
     (2n+1)\int_{0}^{\frac{\pi}{2}} P_n(\cos \theta)\, \sin \theta\, d\theta = P_{n+1}(0) - P_{n-1}(0) \,.
\end{equation}

Sustituyendo \eqref{3.3_factor_integral_desarrollo} en la expresión del potencial \eqref{3.3_solucion_tramos_final}, obtenemos:

\begin{equation}
    \phi(r,\theta) = 
    \begin{cases}
        \displaystyle\sum_{\substack{n=1 \\ n \text{ impar}}}^{\infty}  V_0  \left(\frac{r}{a}\right)^n P_n(\cos\theta) \Big[P_{n+1}(0) - P_{n-1}(0)\Big] \,, & 0 \leq r < a \,,\\[8pt]
        \displaystyle\sum_{\substack{n=1 \\ n \text{ impar}}}^{\infty}  V_0  \left(\frac{a}{r}\right)^{n+1} P_n(\cos\theta) \Big[P_{n+1}(0) - P_{n-1}(0)\Big] \,, & r > a \,.
    \end{cases}
\end{equation}

De este modo, se ha determinado la solución del potencial electrostático en todo el espacio $\mathbb{R}^3$ generado por una distribución compuesta por dos cascarones semisféricos, dispuestos sobre una esfera de radio $a$ y mantenidos a potenciales de igual magnitud $V_0$ pero de signo opuesto.

\clearpage
%%%%%%%%%%%%%%%%%%%%%%%%%%%%%%%%%%%%%%%%%%%%%%%%%%%%%%%%%%%%%%%%%%%%%%%%%%%%%%%%%%%%%%%%%%%%%%%%%%%%%%%%%%%%%%%%%%%%%%%%%%%%%%%%%%%%%%%%%%%%%%%%%%%%%%%%%%%%%%%%%%%%%%%%%%%%%%%%%%%%%%%%%%%%%%%%%%%%%%%

\end{document}
